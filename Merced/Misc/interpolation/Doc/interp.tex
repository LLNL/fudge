\documentclass[11pt]{article}
\usepackage{epic}
\usepackage{eepic}
\textheight=21cm
\textwidth=15cm
\hoffset=0pt
\oddsidemargin=20pt

\begin{document}

\centerline{Interpolation of probability densities in ENDF and ENDL}

\centerline{Gerald Hedstrom}

\centerline{28 December 2005}

\section{Summary}
Suppose that we are given two probability densities
$p_0(E')$ and $p_1(E')$ for the energy $E'$ of an outgoing particle,
$p_0(E')$ corresponding to energy $E_0$ of the incident particle
and $p_1(E')$ corresponding to incident energy $E_1$.
If $E_0 < E_1$, our problem is how to define $p_\alpha(E')$ for 
intermediate incident energies 
\begin{equation}
  E_\alpha = (1 - \alpha)E_0 + \alpha E_1
  \label{intermediateE}
\end{equation}
with $0 < \alpha < 1$.
In this note I consider three ways to do it.
I begin
with unit-base interpolation, which is
standard in ENDL and is sometimes used in ENDF.
I then describe the equiprobable bins used by some
Monte Carlo codes.   I close with a
discussion of
interpolation by
corresponding-points, which is commonly used in ENDF.

\section{Why this interpolation is an issue}
What we have to be careful of is to ensure that our interpolation
method preserves the normalization
\begin{equation}
  \int p(E, E') \, dE' = 1
  \label{norm}
\end{equation}
for all incident energies $E_0 < E < E_1$.

To see why the normalization (\ref{norm}) might be a problem
consider the following ``reasonable'' example.  Suppose that
the energy densities are flat for two incident energies
$E_0 < E_1$,
\[
  p_0(E') = 1, \quad 0 \le E' \le 1
\]
and
\[
  p_1(E') = 1/2, \quad 0 \le E' \le 2.
\]
See Fig.~1.

\begin{figure}
% picture of 2 rectangles
\unitlength=20mm
$$
\begin{picture}(2,1)(0,0)
\thicklines
% the axes
  \put(-0.05,0){\line(1,0){2.15}}
  \put(0,-0.05){\line(0,1){1.15}}
  \put(1, -0.5){$E'$}
  \put(-0.45, 0.8){$p$}
% labels and curves
  \put (-0.05, 0.97){\line(1,0){1.05}}
  \put (-0.2, 1){$1$}
  \put (-0.05, 0.5){\line(1,0){0.05}}
  \put (-0.4, 0.47){$1/2$}
  \dashline{0.1}(0, 0.5)  (2, 0.5)
  \put (1,-0.05){\line(0,1){1.05}}
  \put (1, -0.2){$1$}
  \put (2,-0.05){\line(0,1){0.55}}
  \put (2, -0.2){$2$}
  \put (0.6, 1.1){$p_0$}
  \put (1.6, 0.6){$p_1$}
\end{picture}
$$
\vglue 40pt
\centerline{{\bf Fig.~1.}  Distributions for $E_0$ and $E_1$.}

\end{figure}

We don't want to take a simple average of these densities.
At incident energy
\[
  E_{1/10} = \frac{1}{10}(9E_0 + E_1),
\]
for example, because this would give the density shown in Fig.~2.
The trouble is that for incident energy close to $E_0$ we
immediately get a positive probability density on the whole
interval $1 < E' < 2$.

\begin{figure}
% straight interpolation
\unitlength=20mm
$$
\begin{picture}(2,1)(0,0)
\thicklines
% the axes
  \put(-0.05,0){\line(1,0){2.15}}
  \put(0,-0.05){\line(0,1){1.15}}
  \put(1, -0.5){$E'$}
  \put(-0.45, 0.6){$p$}
% labels and curves
  \put (-0.05, 1){\line(1,0){0.05}}
  \put (-0.2, 1){$1$}
  \put (0, 0.95){\line(1,0){1}}
  \put (1, 0.05){\line(1,0){1}}
  \put (1,-0.05){\line(0,1){1}}
  \put (1, -0.2){$1$}
  \put (2,-0.05){\line(0,1){0.1}}
  \put (2, -0.2){$2$}
\end{picture}
$$
\vglue 40pt
\centerline{{\bf Fig.~2.}  Straight interpolation.}

\end{figure}

Another bad idea is to mindlessly interpolate both the probability
densities and the lengths of the $E'$ intervals.  For the example
shown in Fig.~1 this would give
\[
  p_{1/2}(E') = 3/4, \quad 0 \le E' \le 3/2.
\]
at
\[
  E_{1/2} = \frac{1}{2}(E_0 + E_1),
\]
This density has the wrong normalization
\[
  \int p_{1/2}(E') \, dE' = 9/8.
\]

In this note I illustrate three better interpolation
methods using as an example the probability densities shown 
in Fig.~3,
\begin{equation}
  p_0(E') = \left\{
    \begin{array}{ll}
       4E', & \hbox{for $0 \le E' < 1/2$}, \\
       4(1 - E'), & \hbox{for $1/2 < E' \le 1$}
    \end{array}
  \right.
  \label{p0def}
\end{equation}
and
\begin{equation}
  p_1(E') = \left\{
    \begin{array}{ll}
      E'/2, & \hbox{for $0 \le E' < 1$}, \\
      {\displaystyle \frac{2}{3} - \frac{E'}{6},}
        & \hbox{for $1 < E' \le 4$}.
    \end{array}
  \right.
  \label{p1def}
\end{equation}
The motivation for this example, besides being simple, is that the 
higher-energy incident particle gives a higher-energy peak in
the probability density
for the energy of the outgoing particle, and it also gives a longer tail.

\begin{figure}
% the 2 distributions to interpolate
\unitlength=1.5mm
$$
\begin{picture}(40,20)(0,0)
\thicklines
% the axes
  \put(-0.5,0){\line(1,0){41.5}}
  \put(0,-0.5){\line(0,1){21.5}}
  \put(20, -5){$E'$}
  \put(-4.5, 13){$p$}
% the curves
  \dashline{1}(0,0)  (10,  5)  (40, 0)
  \path(0,0)  (5, 20)  (10,0)
% labels
\thinlines
  \put (-0.5, 5){\line(1,0){0.5}}
  \put (-4.5, 4.5){$1/2$}
  \put (-0.5, 20){\line(1,0){0.5}}
  \put (-2.5, 19.5){$2$}
  \put (10,-0.5){\line(0,1){0.5}}
  \put (9.5, -3){1}
  \put (40,-0.5){\line(0,1){0.5}}
  \put (39.5, -3){4}
  \put (8, 12){$p_0$}
  \put (25, 4){$p_1$}
\end{picture}
$$
\vglue 40pt
\centerline{{\bf Fig.~3.}  Example distributions}

\end{figure}

\section{Unit-base interpolation}
Unit-base interpolation is defined in the Omega manual~\cite{Omega}
on pages VI-19 to VI-22.
The idea behind unit-base interpolation is that the total
probability remains~1 if the interpolation is done on a fixed interval
$0 \le \eta \le 1$.  Therefore, if we have a probability density 
$p(E')$ defined on
an energy interval $E'_{\rm min} \le E' \le E'_{\rm max}$, we make the change
of variable 
\begin{equation}
  E' = E'_{\rm min} + (E'_{\rm max} - E'_{\rm min}) \eta
  \label{Emap}
\end{equation}
 with $0 \le \eta \le 1$,
and we work with the scaled density
\begin{equation}
  \Pi(\eta) = (E'_{\rm max} - E'_{\rm min}) 
     p( E'_{\rm min} + (E'_{\rm max} - E'_{\rm min}) \eta ).
  \label{pmap}
\end{equation}
This mapping maintains the normalization
\[
  \int_0^1 \Pi(\eta) \, d\eta = 
    \int_{E'_{\rm min}}^{E'_{\rm max}} p(E')\, dE' = 1.
\]

I now show that the unit-base mapping (\ref{pmap}) gets the
normalization right under interpolation.  Suppose that we
have two probability densities $\Pi_0(\eta)$ and $\Pi_1(\eta)$
both on the interval $0 \le \eta \le 1$ and both normalized
\[
  \int_0^1 \Pi_j(\eta) \, d\eta = 1, \quad j = 0, 1.
\]
For $0 < \alpha < 1$ we define the interpolated unit-base probability as
\begin{equation}
  \Pi_\alpha(\eta) = (1 - \alpha) \Pi_0(\eta) + \alpha \Pi_1(\eta).
  \label{unitbase}
\end{equation}
It is clear that
\[
  \int_0^1 \Pi_\alpha(\eta) \, d\eta = (1 - \alpha) + \alpha = 1.
\]
So, the normalization does come out right.  To get the interpolated
probability density $p_\alpha(E')$ it remains only to interpolate
the energy intervals $(E'_{\rm min}, E'_{\rm max})$ and to
invert the mapping~(\ref{pmap}).  For the densities $p_0$ and $p_1$
of (\ref{p0def}-\ref{p1def}) shown in Fig.~3 the unit-base
interpolation at $\alpha = 1/2$ is given in Fig.~4.

\begin{figure}
% unit-base interpolation
\unitlength=20mm
$$
\begin{picture}(2.5,1)(0,0)
\thicklines
% the axes
  \put(-0.05,0){\line(1,0){2.65}}
  \put(0,-0.05){\line(0,1){1}}
  \put(1.25, -0.5){$E'$}
  \put(-0.3, 1){$p$}
% the curve
  \path(0,0) (0.625, 0.6)  (1.25,  0.667)  (2.5,  0)
% labels
\thinlines
  \put (-0.05, 0.667){\line(1,0){0.05}}
  \put (-0.45, 0.66){$2/3$}
  \put (-0.05, 0.6){\line(1,0){0.05}}
  \put (-0.45, 0.5){$3/5$}
  \put (0.625,-0.05){\line(0,1){0.05}}
  \put (0.575, -0.25){$5/8$}
  \put (1.25,-0.05){\line(0,1){0.05}}
  \put (1.2, -0.25){$5/4$}
  \put (2.5,-0.05){\line(0,1){0.05}}
  \put (2.45, -0.25){$5/2$}
\end{picture}
$$
\vglue 40pt
\centerline{{\bf fig.~4.}  Interpolated by unit base}

\end{figure}

In order to understand why the graph in Fig.~4 looks as it does,
it is useful to investigate unit-base interpolation in the
original $(E, E')$ energy space.  It is clear that the length
of the $E'$ energy interval
\[
  L = E'_{\rm max} - E'_{\rm min}
\]
plays a determining role in the mapping~(\ref{pmap}).  Let
$L_0$ denote the length of the $E'$ energy interval for
the probability density~$p_0$ and $L_1$ the length of 
the $E'$ energy interval for~$p_1$.  At the intermediate
incident incident energy $E_\alpha$ (\ref{intermediateE})
the length is
\[
   L_\alpha = (1 - \alpha)L_0 + \alpha L_1.
\]
With this notation the unit-base interpolation (\ref{unitbase})
takes the form
\begin{equation}
  L_\alpha p_\alpha( E'_\alpha ) =
    (1 - \alpha) L_0 p_0( E'_0 ) + \alpha L_1 p_1( E'_1 ).
  \label{Eunitbase}
\end{equation}

The energies $E'_\alpha$ for $0 \le \alpha \le 1$ used in
(\ref{Eunitbase}) are those
obtained from (\ref{Emap}) for fixed~$\eta$,
\[
  E'_\alpha = E'_{{\rm min},\alpha} + L_\alpha \eta.
\]
For the densities $p_0$ and $p_1$ of (\ref{p0def}-\ref{p1def})
this interpolation is to be done along the diagonal lines in
Fig.~5.  In this figure the top of the trapezoid $T_1$ is
generated by the discontinuity in the derivative of $p_1$ at $E' = 1$,
and the top of trapezoid $T_2$ is generated by the discontinuity in 
the derivative of $p_0$ at $E' = 1/2$.  For $p_{1/2}$ shown
in Fig.~4 the region $0 < E' < 5/8$ lies in trapezoid $T_1$, 
the region $5/8 < E' < 5/4$ lies in trapezoid~$T_2$, and
the region $5/4 < E' < 5/2$ lies in trapezoid~$T_3$.  It is also
significant that for each of the trapezoids $T_1$, $T_2$, and $T_3$
the ratios of the lengths of the two vertical sides is the same,
namely $L_1/L_0$.

\begin{figure}
% unit-base viewed in energy space
\unitlength=15mm
$$
\begin{picture}(3,4)(0,0)
\thicklines
% the axes
  \put(-0.1,0){\line(1,0){3.3}}
  \put(0,-0.1){\line(0,1){4.1}}
  \put(1.5, -0.6){$E$}
  \put(-0.6, 2.5){$E'$}
% the E lines
  \put(1,-0.05){\line(0,1){1.05}}
  \put(3,-0.05){\line(0,1){4.05}}
% labels
  \put (-0.05, 0.5){\line(1,0){0.05}}
  \put (-0.5, 0.45){$1/2$}
  \put (-0.05, 1){\line(1,0){0.05}}
  \put (-0.25, 0.95){$1$}
  \put (-0.05, 2){\line(1,0){0.05}}
  \put (-0.25, 1.95){$2$}
  \put (-0.05, 4){\line(1,0){0.05}}
  \put (-0.25, 3.95){$4$}
  \put (0.9, -0.35){$E_0$}
  \put (2.9, -0.35){$E_1$}
  \put (2.1, 0.25){$T_1$}
  \put (2.1, 1){$T_2$}
  \put (2.1, 1.8){$T_3$}
% the curves
  \path(1,0.25) (3,1)
  \path (1, 0.5) (3, 2)
  \path (1, 1) (3, 4)
\end{picture}
$$
\vglue 40pt
\centerline{{\bf Fig.~5.}  Unit-base interpolation in energy space}

\end{figure}

\section{Equiprobable bins}
For a probability density
$p(E')$ on an energy range $E_{\rm min} <= E' <= E_{\rm max}$
suppose that we want $N$ equiprobable bins, with $N$ a positive
integer.
These bins are the intervals delimited by the numbers $\epsilon_j$
such that
\begin{equation}
  \int_{E_{\rm min}}^{\epsilon_j}
    p(E') \, dE' = \frac{j}{N}
  \label{Ebins}
\end{equation}
for $j = 0, 1, \ldots, N$.  It is clear that $\epsilon_0 = E'_{\rm min}$,
and it follows from the normalization that $\epsilon_N = E'_{\rm max}$.

{\bf Warning.}  The condition on the probability density is only
that $p(E') \ge 0$ on the energy range $E_{\rm min} <= E' <= E_{\rm max}$.
If it should happen that $p(E') = 0$ on a subinterval, then the
values $\epsilon_j$ defined by (\ref{Ebins}) may not be uniquely
determined.  This could occur, for example, if $p(E')$ is the probability
density of a set of discrete gamma lines.  It is not a good idea to use
interpolation by equiprobable bins is such situations.

For interpolation by equiprobable bins between a probability
density $p_0(E')$ at incident energy $E_0$ and density
$p_1(E')$ at incident energy $E_1$ we use (\ref{Ebins})
to calculate the bin edges $\epsilon_j(0)$ for $p_0$ and
$\epsilon_j(1)$ for $p_1$.  For an intermediate incident
energy
\[
  E_\alpha = (1 - \alpha) E_0 + \alpha E_1
\]
with $0 < \alpha < 1$ we use interpolated bin edges
\[
  \epsilon_j(\alpha) = (1 - \alpha) \epsilon_j(0) +
     \alpha \epsilon_j(1).
\]
On the interval $\epsilon_{j-1}(\alpha) < E' < \epsilon_j(\alpha)$
we define the probability density $p_\alpha(E')$ to be
\[
  p_\alpha(E') = \frac{1}{N( \epsilon_j(\alpha) - \epsilon_{j-1}(\alpha) ) }.
\]
This value is chosen to make
\[
  \int_{\epsilon_{j-1}(\alpha)}^{\epsilon_j(\alpha)} p_\alpha(E') \, dE' =
    \frac{1}{N}.
\]

With $N = 8$ the equiprobable bins for the probability density
$p_0(E')$ of (\ref{p0def}) are shown in Fig.~6, and the equiprobable
bins for the density $p_1(E')$ of (\ref{p1def}) are shown in Fig.~7.
The interpolated equiprobable bins for $p_{1/2}(E')$ are shown in 
Fig.~8.

\begin{figure}
% equiprobable bins for p_0
\unitlength=20mm
$$
\begin{picture}(1,2)(0,0)
\thicklines
% the axes
  \put(-0.05,0){\line(1,0){1.15}}
  \put(0,-0.05){\line(0,1){2.05}}
  \put(0.5, -0.5){$E'$}
  \put(-0.4, 1){$p$}
% the curve
  \path(0,0) (0.5, 2) (1,  0)
% labels
  \put (1,-0.05){\line(0,1){0.05}}
  \put (0.98, -0.25){1}
  \put (-0.05, 2){\line(1,0){0.05}}
  \put (-0.25, 1.98){$2$}
% the bins
  \put(0, 0.5){\line(1,0){0.25}}
  \put(0.25, 1.20711){\line(1,0){0.103553}}
  \put(0.353553, 1.57313){\line(1,0){0.0794593}}
  \put(0.433013, 1.86603){\line(1,0){0.0669873}}
  \put(0.5, 1.86603){\line(1,0){0.0669873}}
  \put(0.566987, 1.57313){\line(1,0){0.0794593}}
  \put(0.646447, 1.20711){\line(1,0){0.103553}}
  \put(0.75, 0.5){\line(1,0){0.25}}
\thinlines
  \put(0.25,0){\line(0,1){1.20711}}
  \put(0.353553,0){\line(0,1){1.57313}}
  \put(0.433013,0){\line(0,1){1.86603}}
  \put(0.5,0){\line(0,1){1.86603}}
  \put(0.566987,0){\line(0,1){1.86603}}
  \put(0.646447,0){\line(0,1){1.57313}}
  \put(0.75,0){\line(0,1){1.20711}}
  \put(1,0){\line(0,1){0.5}}
\end{picture}
$$
\vglue 40pt
\centerline{{\bf Fig.~6.}  Equiprobable bins for $p_0$}

\end{figure}

\begin{figure}
% equiprobable bins for p_1
\unitlength=15mm
$$
\begin{picture}(4,0.5)(0,0)
\thicklines
% the axes
  \put(-0.05,0){\line(1,0){4.15}}
  \put(0,-0.05){\line(0,1){0.6}}
  \put(2, -0.5){$E'$}
  \put(-0.6, 0.25){$p$}
% the curve
  \path(0,0) (1, 0.5) (4,  0)
% labels
  \put (-0.05, 0.5){\line(1,0){0.05}}
  \put (-0.5, 0.45){$1/2$}
  \put (1,-0.05){\line(0,1){0.05}}
  \put (0.95, -0.25){$1$}
  \put (4,-0.05){\line(0,1){0.05}}
  \put (3.95, -0.25){$4$}
% the bins
  \put(0, 0.176777){\line(1,0){0.707107}}
  \put(0.707107, 0.426777){\line(1,0){0.292893}}
  \put(1, 0.478218){\line(1,0){0.261387}}
  \put(1.26139, 0.432342){\line(1,0){0.289123}}
  \put(1.55051, 0.380901){\line(1,0){0.328169}}
  \put(1.87868, 0.321114){\line(1,0){0.38927}}
  \put(2.26795, 0.2464){\line(1,0){0.507306}}
  \put(2.77526, 0.102062){\line(1,0){1.22474}}
\thinlines
  \put(0.707107,0){\line(0,1){0.426777}}
  \put(1,0){\line(0,1){0.478218}}
  \put(1.26139,0){\line(0,1){0.478218}}
  \put(1.55051,0){\line(0,1){0.432342}}
  \put(1.87868,0){\line(0,1){0.380901}}
  \put(2.26795,0){\line(0,1){0.321114}}
  \put(2.77526,0){\line(0,1){0.2464}}
  \put(4,0){\line(0,1){0.102062}}
\end{picture}
$$
\vglue 40pt
\centerline{{\bf Fig.~7.}  Equiprobable bins for $p_1$}

\end{figure}

\begin{figure}
% interpolated equiprobable bins
\unitlength=15mm
$$
\begin{picture}(2.5,1)(0,0)
\thicklines
% the axes
  \put(-0.05,0){\line(1,0){2.6}}
  \put(0,-0.05){\line(0,1){1.1}}
  \put(1.25, -0.5){$E'$}
  \put(-0.45, 0.5){$p$}
% labels
  \put (-0.05, 1){\line(1,0){0.05}}
  \put (-0.25, 0.98){$1$}
  \put (2.5,-0.05){\line(0,1){0.05}}
  \put (2.45, -0.3){$2.5$}
% the continuuous limit
  \path (0, 0) (0.677, 0.739) (0.694, 0.739) (0.728, 0.738) (0.762, 0.735)
  \path (0.762, 0.735) (0.796, 0.731) (0.83, 0.726) (0.865, 0.719) (0.9, 0.712)
  \path (0.9, 0.712) (0.935, 0.703) (0.971, 0.694) (1.007, 0.684)(1.025, 0.678)
  \path (1.025, 0.678) (2.5, 0 )
% the bins
  \put(0, 0.261204){\line(1,0){0.478553}}
  \put(0.478553, 0.630602){\line(1,0){0.198223}}
  \put(0.676777, 0.733468){\line(1,0){0.170423}}
  \put(0.8472, 0.70203){\line(1,0){0.178055}}
  \put(1.02526, 0.63266){\line(1,0){0.197578}}
  \put(1.22283, 0.533357){\line(1,0){0.234364}}
  \put(1.4572, 0.40926){\line(1,0){0.30543}}
  \put(1.76263, 0.169521){\line(1,0){0.737372}}
\thinlines
  \put(0.478553,0){\line(0,1){0.630602}}
  \put(0.676777,0){\line(0,1){0.733468}}
  \put(0.8472,0){\line(0,1){0.733468}}
  \put(1.02526,0){\line(0,1){0.70203}}
  \put(1.22283,0){\line(0,1){0.63266}}
  \put(1.4572,0){\line(0,1){0.533357}}
  \put(1.76263,0){\line(0,1){0.40926}}
  \put(2.5,0){\line(0,1){0.169521}}
\end{picture}
$$
\vglue 40pt
\centerline{{\bf Fig.~8.}  Equiprobable bins for $p_{1/2}$}

\end{figure}

The $(E, E')$ energy space version of this interpolation is
illustrated in Fig.~9.  In fact, we are doing unit-base
interpolation within each trapezoid in this figure, but
now $L_0$ and $L_1$ in equation (\ref{Eunitbase})
refer to the lengths of the vertical edges of the trapezoids.
Hence, the ratio $L_1/L_0$ differs
from one trapezoid to another.

\begin{figure}
% equiprobable bins in energy space
\unitlength=15mm
$$
\begin{picture}(3,4)(0,0)
\thicklines
% the axes
  \put(-0.1,0){\line(1,0){3.3}}
  \put(0,-0.1){\line(0,1){4.1}}
  \put(1.5, -0.6){$E$}
  \put(-0.6, 2){$E'$}
% the E lines
  \put(1,-0.05){\line(0,1){1.05}}
  \put(3,-0.05){\line(0,1){4.05}}
% labels
  \put (-0.05, 1){\line(1,0){0.05}}
  \put (-0.25, 0.95){$1$}
  \put (-0.05, 4){\line(1,0){0.05}}
  \put (-0.25, 3.95){$4$}
  \put (0.9, -0.3){$E_0$}
  \put (2.9, -0.3){$E_1$}
% the curves
  \path(1, 0.25)  (3, 0.707)
  \path(1, 0.354)  (3, 1)
  \path(1, 0.433)  (3, 1.261)
  \path(1, 0.5)  (3, 1.551)
  \path(1, 0.567)  (3, 1.879)
  \path(1, 0.646)  (3, 2.268)
  \path(1, 0.75)  (3, 2.775)
  \path(1, 1)  (3, 4)
\end{picture}
$$
\vglue 40pt
\centerline{{\bf Fig.~9.}  The view in energy space}

\end{figure}

It is interesting to investigate the continuum limit of
equiprobable bin interpolation as $N \to \infty$.  For interpolation
of the
densities $p_0(E')$ of (\ref{p0def}) and $p_1(E')$ of (\ref{p1def})
this limit is shown as the continuous curve in Fig.~8.  In this
discussion we need to be careful of the mathematical terminology.
Up to this point we have worked only with {\em probability densities}
$p(E')$, namely, nonnegative functions such that
\[
  \int_{E'_{\rm min}}^{E'_{\rm max}} p(E') \, dE' = 1.
\]
The analysis of equiprobable bin interpolation is based on the
corresponding {\em distribution function} $F(E')$ defined by
the integral
\[
  F(E') = \int_{E'_{\rm min}}^{E'} p(E'') \, dE''.
\]
In fact, it follows from (\ref{Ebins}) that the equiprobable bin 
edges $\epsilon_j$ for $j = 0, 1, \ldots, N$ are obtained from 
the inverse to the distribution 
function,
\[
  \epsilon_j = F^{-1}\left(
     \frac{j}{N}
   \right).
\]
In view of the warning given above, I assume in this discussion
that $F(E')$ is strictly increasing
\begin{equation}
  F(E'_2) > F(E'_1) \quad
    \hbox{whenever $E'_{\rm min} \le E'_1 < E'_2 \le E'_{\rm max}$}.
  \label{onetoone}
\end{equation}
This condition implies that $F^{-1}(y)$ is uniquely defined for
$0 \le y \le 1$.

With this background, we see that the continuum version of
interpolation by equiprobable bins proceeds by the following
steps.
\begin{enumerate}
  \item Given probability densities $p_0(E')$ for
     incident energy $E_0$ and $p_1(E')$ for
     incident energy $E_1$, form the corresponding
     distribution functions $F_0(E')$ and $F_1(E')$.
  \item Solve the equations $y = F_0(E')$ and $y = F_1(E')$
     to determine the inverse distribution functions
     $E' = F_0^{-1}(y)$ and $E' = F_1^{-1}(y)$ on $0 \le y \le 1$.
  \item For $0 < \alpha < 1$ the interpolated inverse distribution 
     function $F_\alpha^{-1}(y)$ is definded as
\[
   F_\alpha^{-1}(y) = ( 1 - \alpha ) F_0^{-1}(y) +
    \alpha F_1^{-1}(y).
\]
  \item Get the interpolated distribution function $y = F_\alpha(E')$
    by solving the equation $E' = F_\alpha^{-1}(y)$.
  \item Differentiate to obtain the interpolated probability
    density
\[
    p_\alpha(E') = \frac{d}{dE'} F_\alpha(E').
\]
\end{enumerate}

Interpolation by equiprobable bins is simply a finite-difference
approximation to this algorithm.

Let's illustrate these ideas by applying them to the densities
$p_0$ and $p_1$ of (\ref{p0def}-\ref{p1def}).  The distribution
function $F_0(E')$ for $p_0(E')$ is obtained by integrating
(\ref{p0def}),
\[
  F_0(E') = \left\{
    \begin{array}{ll}
      2{E'}^2 & \hbox{for $0 \le E' \le 1/2$}, \\
      -1 + 4E' -2{E'}^2 & \hbox{for $1/2 \le E' \le 1$}.
    \end{array}
  \right.
\]
Note that in the equation $y = F_0(E')$ the interval
 $0 \le E' \le 1/2$ corresponds to $0  \le y \le 1/2$ and
$1/2 \le E' \le 1$ corresponds to $1/2 \le y \le 1$.
In inverting $y = F_0(E')$ we take the physically
relevant square roots,
\begin{equation}
  E' = F_0^{-1}(y) = \left\{
    \begin{array}{ll}
        {\displaystyle  \sqrt{ \frac{y}{2} } }
          & \hbox{for $0 \le y \le 1/2$}, \\
        {\displaystyle 1 - \sqrt{ \frac{1-y}{2} } }
          & \hbox{for $1/2 \le y \le 1$}.
    \end{array}
  \right.
  \label{F0inverse}
\end{equation}

If we apply the same operations to the probability density
$p_1(E')$ given by (\ref{p1def}), we get the distribution
function
\[
  F_1(E') = \left\{
    \begin{array}{ll}
      {\displaystyle  \frac{{E'}^2}{4} }
         & \hbox{for $0 \le E' \le 1$}, \\
      {\displaystyle   -\frac{1}{3} + \frac{2E'}{3} - \frac{{E'}^2}{12} }
         & \hbox{for $1 \le E' \le 4$}.
    \end{array}
  \right.
\]
For $y = F_1(E')$ the interval $0 \le E' \le 1$ corresponds to
$0 \le y \le 1/4$ and the interval $1 \le E' \le 4$ corresponds 
to $1/4 \le y \le 1$.  The inverse distribution function is
given by
\begin{equation}
  E' = F_1^{-1}(y) = \left\{
    \begin{array}{ll}
      \sqrt{ 2y} & \hbox{for $0 \le y \le 1/4$}, \\
         4 - \sqrt{ 3(1-y) } & \hbox{for $1/4 \le y \le 1$}.
    \end{array}
  \right.
  \label{F1inverse}
\end{equation}

Let us interpolate half way between these inverse distribution functions,
\begin{equation}
  F_{1/2}^{-1}(y) = \frac{1}{2} ( F_0^{-1}(y) + F_1^{-1}(y) ).
 \label{Finverse}
\end{equation}
It is clear from (\ref{F0inverse}-\ref{F1inverse}) that this
function is defined on the interval $0 \le y \le 1$ and that
it is given by 3 different formulas, depending on whether
$0 \le y \le 1/4$ or $1/4 \le y \le 1/2$ or $1/2 \le y \le 1$.
Let us examine these regions one at a time.

On the interval $0 \le y \le 1/4$ equation (\ref{Finverse})
takes the form
\begin{equation}
  E' = \left(
    1 + \frac{\sqrt{2}}{4}
  \right)
  \sqrt{y}.
  \label{firstinv}
\end{equation}
The range of $E'$ values is from
\[
  E' = 0 \quad \hbox{for} \quad y = 0
\]
to
\[
  E' = \frac{4 + \sqrt{2}}{8} \quad \hbox{for} \quad y = \frac{1}{4}.
\]
Upon solving (\ref{firstinv}) for $y$, we get the distribution function
\[
  F_{1/2}(E') = \left(
    \frac{4}{4 + \sqrt{2}}
   \right)^2
    {E'}^2
   \quad \hbox{for} \quad 
  0 \le E' \le \frac{4 + \sqrt{2}}{8}.
\]
The probability density is the derivative
\[
  p_{1/2}(E') = \left(
    \frac{4}{4 + \sqrt{2}}
   \right)^2
    2E'
   \quad \hbox{for} \quad 
  0 \le E' \le \frac{4 + \sqrt{2}}{8}.
\]

The algebra is more complicated for $1/4 \le y \le 1/2$ because
from (\ref{Finverse}) we obtain the equation
\[
  F_{1/2}^{-1}(y) = \frac{1}{2} \sqrt{ \frac {y}{2} } + 2
   - \sqrt{ 3(1 - y) }.
\]
In particular, $y = 1/2$ corresponds to
\[
  E' = F_{1/2}^{-1}(1/2) = \frac{9}{4} - \frac{\sqrt{3}}{2}.
\]
It is possible to solve $E' = F_{1/2}^{-1}(y)$ for $y = F_{1/2}(E')$,
but I choose not to do it.

For $1/2 \le y \le 1$ we find from (\ref{Finverse}) that
\[
  E' = F_{1/2}^{-1}(y) = \frac{5}{2} -
    \left(
      \frac { \sqrt{2} }{4} + \sqrt{3}
    \right)
   \sqrt{ 1 - y }.
\]
With the notation
\[
  \beta = \frac { \sqrt{2} }{4} + \sqrt{3},
\]
we can easily solve this equation for $y$ to get
\[
  y = F_{1/2}(E') = \frac{1}{\beta^2}
  \left(
    \frac{5}{2} - E'
  \right)^2
\]
for 
\[
  \frac{9}{4} - \frac{\sqrt{3}}{2} \le E' \le \frac{5}{2}.
\]
The derivative of $F_{1/2}(E')$ is the probability density
\[
  p_{1/2}(E') = \frac{2}{\beta^2}
  \left(
    \frac{5}{2} - E'
  \right).
\]

\section{Interpolation by corresponding points}
The ENDF documentation on interpolation by corresponding points
\cite{ENDF} page~34
is not entirely clear, so I give here my interpretation.

In the method of corresponding points the probability densities
$p(E, E')$ are piecewise linear in the energy $E'$ of the outgoing
particle, and the same number $N$ of data points $(E', p)$
are to be given for each incident energy~$E$.  Interpolation by
corresponding points is by unit-base interpolation (\ref{Eunitbase})
between adjacent points in the table.  But the lengths $L_0$
and $L_1$ used in the interpolation are local.

\begin{figure}
% corresponding points in energy space
\unitlength=15mm
$$
\begin{picture}(3,4)(0,0)
\thicklines
% the axes
  \put(-0.1,0){\line(1,0){3.3}}
  \put(0,-0.1){\line(0,1){4.2}}
  \put(1.5, -0.6){$E$}
  \put(-0.6, 2){$E'$}
% the E lines
  \put(1,-0.05){\line(0,1){1.05}}
  \put(3,-0.05){\line(0,1){4.05}}
% labels
  \put(0.9, -0.3){$E_0$}
  \put(2.9, -0.3){$E_1$}
  \put(-0.05,0.5){\line(1,0){0.05}}
  \put(-0.5, 0.45){$1/2$}
  \put(-0.05,1){\line(1,0){0.05}}
  \put(-0.25, 0.95){$1$}
  \put(-0.05,4){\line(1,0){0.05}}
  \put(-0.25, 3.95){$4$}
  \put(2, 0.3){$T_1$}
  \put(2, 1.4){$T_2$}
% the curves
  \path(1, 0.5)  (3, 1)
  \path(1, 1)  (3, 4)
\end{picture}
$$
\vglue 40pt
\centerline{{\bf Fig.~10.}  Corresponding points for our example}

\end{figure}


I illustrate the process with a simple example with $N = 3$
based on the probability densities (\ref{p0def}-\ref{p1def})
shown in Fig.~3.  The corresponding data points are the
vertices of the triangles in Fig.~3, and they are given
in Table~1.

\begin{table}
$$
\vbox{
  \offinterlineskip
  \hrule
  \tabskip=0pt
  \halign{    % start with the table format
    \vrule height 14pt depth 3pt #      \tabskip=5pt &
    \hfil #\hfil&                       % N column
    \vrule #&
    \hfil$#$\hfil  &               % first E' column
    \hfil$#$\hfil  &               % first p column
    \vrule #&
    \hfil$#$\hfil  &               % second E' column
    \hfil$#$\hfil  &               % second p column
    \vrule #              \tabskip=0pt  \cr  % end of table format
   depth 7pt &&& \multispan{2}\hfil$E = E_0$\hfil&&
        \multispan{2}\hfil$E = E_1$\hfil&\cr
   \noalign{\hrule}
   depth 7pt & $n$&& \hfil E'&  \hfil p &&
                  \hfil E'&  \hfil p & \cr
   \noalign{\hrule}
    & 1&& 0 & 0&& 0 & 0& \cr  % first point
    & 2&& 1/2 & 2&& 1 & 1/2& \cr % second point
 depth 7pt   & 3&& 1 & 0&& 4 & 0& \cr % third point
   \noalign{\hrule}
  }
}
$$
\vglue 10pt
\centerline{{\bf Table 1.}  Corresponding points}
\end{table}

These corresponding points determine trapezoids $T_1$ and $T_2$ in
energy space, as displayed in Fig.~10.  Let's see what the
corresponding points interpolation gives along the bottom of the
trapezoid~$T_2$ at the incident energy
\[
  E_{1/2} = \frac{1}{2}
   (E_0 + E_1).
\]
The lengths of the vertical sides of $T_2$ are
\[
  L_0 = \frac{1}{2} \quad \hbox{and} \quad
  L_1 = 3,
\]
and the average length is
\[
  L_{1/2} = \frac{7}{4}.
\]
Unit-base interpolation (\ref{Eunitbase}) in $T_2$ therefore takes 
the form
\[
   \frac{7}{4} p_{1/2}(3/4) =
     \frac{1}{2} \left(
       \frac{1}{2}2 + 3\frac{1}{2}
     \right).
\]
Consequently, we find that
\[
  p_{1/2}(3/4) = \frac{5}{7}
\]
as viewed from $T_2$.

The value of $p_{1/2}(3/4)$ as viewed from $T_1$ is different
because for $T_1$ the ratio of the lengths of the vertical sides
\[
  \frac{L_1}{L_0} = 2
\]
is different from the corresponding ratio for $T_2$
\[
  \frac{L_1}{L_0} = 6.
\]
This explains the jump discontinuity shown in Fig.~11.
The full set of interpolated values is given in Table~2.

\begin{figure}
% interpolation by corresponding points
\unitlength=20mm
$$
\begin{picture}(2.5,1)(0,0)
\thicklines
% the axes
  \put(-0.05,0){\line(1,0){2.65}}
  \put(0,-0.05){\line(0,1){1.2}}
  \put(1.25, -0.5){$E'$}
  \put(-0.5, 1.2){$p$}
% the curve
  \path(0, 0)  (0.75, 1)
  \path(0.75, 0.71) (2.5, 0)
\thinlines
% labels
  \put (-0.05, 1){\line(1,0){0.05}}
  \put (-0.25, 0.98){$1$}
  \put (-0.05, 0.71){\line(1,0){0.05}}
  \put (-0.45, 0.68){$5/7$}
%  \put (0.75,-0.05){\line(0,1){0.05}}
  \put (0.7, -0.25){$3/4$}
  \put (2.5,-0.05){\line(0,1){0.05}}
  \put (2.45, -0.25){$5/2$}
% jump discontinuity
  \put (0.75, -0.05){\line(0,1){1.05}}
\end{picture}
$$
\vglue 40pt
\centerline{{\bf Fig.~11.}  Interpolation by corresponding points}

\end{figure}

\begin{table}
$$
\vbox{
  \offinterlineskip
  \hrule
                                 \tabskip=0pt
  \halign{    % start with the table format
    \vrule height 14pt depth 3pt #      \tabskip=5pt &
    \hfil #\hfil&                       % N column
    \vrule #&
    \hfil$#$\hfil  &               % first E' column
    \hfil$#$\hfil  &               % first p column
    \vrule #&
    \hfil$#$\hfil  &               % second E' column
    \hfil$#$\hfil  &               % second p column
    \vrule #&
    \hfil$#$\hfil  &               % third E' column
    \hfil$#$\hfil  &               % third p column
    \vrule #              \tabskip=0pt  \cr  % end of table format
   depth 7pt &&& \multispan{2}\hfil$E = E_0$\hfil&&
        \multispan{2}\hfil$E = E_{1/2}$\hfil&&
        \multispan{2}\hfil$E = E_1$\hfil&\cr
   \noalign{\hrule}
  depth 7pt & $n$&&  E'& p &&  E' & p && E' & p & \cr
   \noalign{\hrule}
    & 1&& 0 & 0&&  0 & 0&& 0 & 0& \cr  % first point
 depth 7pt    & 2&& 1/2 & 2&&  3/4 & 1&& 1 & 1/2& \cr  % second point
   \noalign{\hrule}
    & 2&& 1/2 & 2&&  3/4 & 5/7&& 1 & 1/2& \cr  % second point
 depth 7pt   & 3&& 1 & 0&&  5/2 & 0&& 4 & 0& \cr  % third point
   \noalign{\hrule}
  }
}
$$
\vglue 10pt
\centerline{{\bf Table 2.}  Interpolation by corresponding points}
\end{table}

Another problem with interpolation by corresponding points
is that the lengths of the data columns may not be the same.
For example, it is not at all clear how to pair up the data
points in Table~3.  Table~3 was obtained by inserting an extra
point into the last 2 columns of Table~1.

\begin{table}
$$
\vbox{
  \offinterlineskip
  \hrule
                                 \tabskip=0pt
  \halign{    % start with the table format
    \vrule height 14pt depth 3pt #      \tabskip=5pt &
    \hfil #\hfil&                       % N column
    \vrule #&
    \hfil$#$\hfil  &               % first E' column
    \hfil$#$\hfil  &               % first p column
    \vrule #&
    \hfil$#$\hfil  &               % second E' column
    \hfil$#$\hfil  &               % second p column
    \vrule #              \tabskip=0pt  \cr  % end of table format
   depth 7pt &&& \multispan{2}\hfil$E = E_0$\hfil&&
        \multispan{2}\hfil$E = E_1$\hfil&\cr
   \noalign{\hrule}
   depth 7pt & $n$&&  E'& p &&  E'&  p & \cr
   \noalign{\hrule}
    & 1&& 0 & 0&& 0 & 0& \cr  % first point
    & 2&& 1/2 & 2&& 1 & 1/2& \cr % second point
    & 3&& 1 & 0&& 3 & 1/6& \cr % third point
 depth 7pt   & 4&&  & && 4 & 0& \cr % third point
   \noalign{\hrule}
  }
}
$$
\vglue 10pt
\centerline{{\bf Table 3.}  Unmatched points}
\end{table}

My interpretation of interpolation by corresponding points
is to extend the short list by repeating the last pair, as
in Table~4.

\begin{table}
$$
\vbox{
  \offinterlineskip
                                 \tabskip=0pt
  \halign{    % start with the table format
    \vrule height 14pt depth 3pt #      \tabskip=5pt &
    \hfil #\hfil&                       % N column
    \vrule #&
    \hfil$#$\hfil  &               % first E' column
    \hfil$#$\hfil  &               % first p column
    \vrule #&
    \hfil$#$\hfil  &               % second E' column
    \hfil$#$\hfil  &               % second p column
    \vrule #&
    \hfil$#$\hfil  &               % third E' column
    \hfil$#$\hfil  &               % third p column
    \vrule #              \tabskip=0pt  \cr  % end of table format
   \noalign{\hrule}
   depth 7pt &&& \multispan{2}\hfil$E = E_0$\hfil&&
        \multispan{2}\hfil$E = E_{1/2}$\hfil&&
        \multispan{2}\hfil$E = E_1$\hfil&\cr
   \noalign{\hrule}
   depth 7pt & $n$&& E' & p && E'& p && E'& p & \cr
   \noalign{\hrule}
    & 1&& 0 & 0&&  0 & 0&& 0 & 0& \cr  % first point
 depth 7pt    & 2&& 1/2 & 2&&  3/4 & 1&& 1 & 1/2& \cr  % second point
   \noalign{\hrule}
    & 2&& 1/2 & 2&&  3/4 & 4/5&& 1 & 1/2& \cr  % second point
 depth 7pt   & 3&& 1 & 0&&  2 & 2/15&& 3 & 1/6& \cr  % third point
   \noalign{\hrule}
    & 3&& 1 & 0&&  2 & 1/6&& 3 & 1/6& \cr  % second point
 depth 7pt   & 4&& 1 & 0&&  5/2 & 0&& 4 & 0& \cr  % third point
   \noalign{\hrule}
  }
}
$$
\vglue 10pt
\centerline{{\bf Table 4.}  Extended table.}
\end{table}

This extension is shown in energy space in Fig.~12.  One peculiarity
of extending the corresponding points by a triangle, as I have done
with~$T_3$ is that from the $T_3$ side of the line from $A$ to $B$
the probability density $p_\alpha(E')$ is the constant value that it
takes at $B$.

\begin{figure}
\include{fig12}
\end{figure}

The lack of a formal extension rule for different numbers of
corresponding points is significant, because every ENDF/B-VII
and JEFF evaluation which calls for interpolation by corresponding
points happens to have different numbers of points.

\section{Comments}
The principal disadvantage of unit-base interpolation is that
its division of the domain in energy space into trapezoids with
equal ratios of the vertical sides $L_1/L_0$ as in Fig.~5 may
not be consistent with the physics of the probability densities.
Interpolation by equiprobable bins makes more sense to me physically,
but it also has difficulties.  As seen in Fig.~8, the continuum
limit gives a very reasonable interpolation, but for any finite
number of bins the approximation is poor in regions of low
probability density.  Another drawback of equiprobable bins
is that the process is undefined if $p(E') = 0$ on an interval 
interior to its domain of definition.
This is a violation of the condition (\ref{onetoone})
which is necessary for the method to work.

The primary difficulty with the method of interpolation by
corresponding points is the question of what to do when the
number of points is not the same.  I have made what seems to me
to be a reasonable guess.  Another aspect to be aware of is that
the method generates discontinuities in the interpolated probability
density as in Fig.~11 when the ratio $L_1/L_0$ changes from one
energy trapezoid to the next.


\begin{thebibliography}{9}

\bibitem{Omega} R. J. Howerton, R.~E. Dye, P.~C. Giles,
J.~R. Kimlinger, S.~T. Perkins, and E.~F. Plechaty,
``Omega: a Cray~1 executive code for LLNL nuclear data
libraries'',
Report UCRL-50400 Vol.~25, 
Lawrence Livermore National Laboratory, Livermore, California,
1983.

\bibitem{ENDF} M. Herman, ed.,
``ENDF-102 ENDF-6 data formats and procedures for the evaluated
nuclear data file ENDF-VII'', 
Informal report BNL-NCS-44945-01/04-Rev.,
National Nuclear Data Center,
Brookhaven National Laboratory, Upton, New York. 2005.

\end{thebibliography}

\end{document}
