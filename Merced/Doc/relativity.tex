{ % localize the newcommands
\newcommand{\mzerot}{m_{t,0}}
\newcommand{\mzeroi}{m_{i,0}}
\newcommand{\Ezerot}{m_t}
\newcommand{\Ezeroi}{m_i}
\newcommand{\EzeroR}{m_R}
\newcommand{\Ezeroe}{m_e}
\newcommand{\Tlabin}{T_{i,\textrm{lab}}}
\newcommand{\plabin}{p_{i,\textrm{lab}}}
\newcommand{\Tlabe}{T_{e,\textrm{lab}}}
\newcommand{\plabe}{p_{e,\textrm{lab}}}
\newcommand{\plabea}{p_{e1,\textrm{lab}}}
\newcommand{\plabeb}{p_{e2,\textrm{lab}}}
\newcommand{\plabec}{p_{e3,\textrm{lab}}}
\newcommand{\Tcmi}{T_{i,\textrm{cm}}}
\newcommand{\Tcmt}{T_{t,\textrm{cm}}}
\newcommand{\Tcme}{T_{e,\textrm{cm}}}
\newcommand{\TcmR}{T_{R,\textrm{cm}}}
\newcommand{\pcmi}{p_{i,\textrm{cm}}}
\newcommand{\pcme}{p_{e,\textrm{cm}}}
\newcommand{\pcmea}{p_{e1,\textrm{cm}}}
\newcommand{\pcmeb}{p_{e2,\textrm{cm}}}
\newcommand{\pcmec}{p_{e3,\textrm{cm}}}
%\newcommand{\mucm}{\mu_{\textrm{cm}}}
%\newcommand{\mulab}{\mu_{\textrm{lab}}}

\chapter{Relativistic 2-body problems}
\label{Appendix-relativity}
In this appendix, relativistic 2-body mechanics is examined from the point
of view of computational physics.  That is, the subtraction
nearly equal numbers is avoided as much as is possible.
The analysis starts with a collision of an incident particle with
a stationary target.  This determines the mapping between
the laboratory frame and the center-of-mass frame.  
The appendix closes with a discussion of
emission after the reaction. 

As is customary in discussions of relativity, the units are
such that the speed of light has the value~$c = 1$.

\section{Initial collision}
For this appendix, $E$ is the total energy of
a system
and $p$ its total momentum.  Thus, for a particle with rest mass~$m_0$
and kinetic energy~$T$, it follows that $E = m_0 + T$. 
\textit{The convention $c = 1$ implies that the data must be such that
particle rest masses and kinetic energies must be given in the same units.}
The analysis makes repeated use of the invariance under Lorentz
transformations of the quantity
\begin{equation}
  S_0 = E^2 - p^2.
  \label{spacetime}
\end{equation}

If the system is a single particle
in a frame in which the particle is stationary, then $S_0 = m_0^2$.
Consequently, for a single particle in
any frame Eq.~(\ref{spacetime}) takes the form
\begin{equation}
  m_0^2 = (m_0 + T)^2 - p^2,
 \label{spacetime-1}
\end{equation}
or
\begin{equation}
  p^2  = 2m_0 T + T^2.
  \label{one-p}
\end{equation}
When it is desired to solve Eq.~(\ref{one-p}) for~$T$ corresponding
to a known value of~$p^2$, it is recommended to use the formula
\begin{equation}
  T = \frac{p^2 }
          {m_0 + \sqrt{ m_0^2 + p^2 }}.
 \label{good-T}
\end{equation}
The relation Eq.~(\ref{good-T}) is computationally more reliable
than the more obvious solution of the quadratic equation Eq.~(\ref{one-p})
$$
  T = -m_0 + \sqrt{ m_0^2 + p^2 }.
$$

Consider the application of Eq.~(\ref{spacetime}) to the system
consisting of a moving incident particle and a target 
at rest in
the laboratory frame.  Suppose that the incident
particle has rest mass $\Ezeroi$ and kinetic
energy $\Tlabin$, and let $\Ezerot$ be the rest mass of the
target.  Then it follows from Eq.~(\ref{one-p})
that the initial laboratory-frame momentum is given by
\begin{equation}
  \plabin^2  = 2\Ezeroi \Tlabin + \Tlabin^2.
 \label{p-lab-in}
\end{equation}
Consequently, for the system of consisting of the two particles
in the laboratory frame, the energy-momentum invariant is
$$
  S = (\Ezerot + \Ezeroi + \Tlabin)^2 -
      ( 2\Ezeroi \Tlabin + \Tlabin^2 ),
$$
This expression simplifies to
\begin{equation}
  S = (\Ezeroi + \Ezerot )^2 + 2\Ezerot \Tlabin.
  \label{S-lab}
\end{equation}

The value of $S$ must be the same when this
system of two particles is considered in the center-of-mass
frame.  Denote the center-of-mass kinetic energy of
the incident particle by $\Tcmi$ and its momentum by~$\pcmi$.
Similarly, let the target have center-of-mass kinetic energy $\Tcmt$, 
and its momentum is~$-\pcmi$.  The energy-momentum invariant
for the system is therefore
\begin{equation}
  S = (\Ezeroi + \Tcmi + \Ezerot + \Tcmt)^2,
 \label{spacetime-cm}
\end{equation}
the square of the total energy of the system in the center-of-mass frame.
By using Eq.~(\ref{spacetime-1}) on each of the particles, it
is possible to rewrite this as
$$
  S = \left(
    \sqrt{\Ezeroi^2 + \pcmi^2 } +
       \sqrt{\Ezerot^2 + \pcmi^2 }
   \right)^2.
$$
Upon solving this equation for $\pcmi^2$, it is found that
\begin{equation}
  \pcmi^2 = \frac{ \left[S - (\Ezeroi^2 + \Ezerot^2)\right]^2 -
                 4 \Ezeroi^2 \Ezerot^2 }
                {4 S }.
  \label{p-cm}
\end{equation}
An expression for $\pcmi^2$ in terms of the
laboratory incident kinetic energy $\Tlabin$ is obtained by substituting
in Eq.~(\ref{p-cm}) the value of $S$ given by Eq.~(\ref{S-lab}),
\begin{equation}
  \pcmi^2 = \frac{\Ezerot^2( 2 \Ezeroi \Tlabin + \Tlabin^2)}
              {(\Ezerot + \Ezeroi )^2 + 2\Ezerot \Tlabin}.
  \label{p-cm-1}
\end{equation}
It follows from Eq.~(\ref{one-p}) that this equation may also
be written as
$$
  \pcmi^2 = \frac{\Ezerot^2 \plabin^2 }
                {(\Ezerot + \Ezeroi )^2 + 2\Ezerot \Tlabin}.
$$

\section{Mapping between frames}
Consider a coordinate system in which the momentum
$\plabin$ of the incident particle is in the direction of
the first spatial axis.  The boost from
the laboratory to the center-of-mass frame then takes the
form
$$
  (E_{\text{cm}}, p_{\text{cm}})^T = R (E_{\text{lab}}, p_{\text{lab}})^T
$$
with the matrix
\begin{equation}
   R =
    \begin{bmatrix}
     \cosh \chi & -\sinh \chi & 0 & 0 \\
     -\sinh \chi & \cosh \chi & 0 & 0 \\
     0  & 0 & 1 & 0 \\
     0  & 0 & 0 & 1
  \end{bmatrix}.
  \label{R-3}
\end{equation}

Upon applying the rotation Eq.~(\ref{R-3}) to the target,
it is found that
$$
     \begin{bmatrix}
        \Ezerot + \Tcmt \\
        -|\pcmi| \\
        0 \\
        0
     \end{bmatrix}
      = R
\begin{bmatrix}
        \Ezerot \\
           0  \\
           0 \\
           0
     \end{bmatrix}.
$$
It follows that
\begin{equation}
   \sinh \chi = \frac{|\pcmi|}{\Ezerot}.
 \label{lab-to-cm}
\end{equation}
By using Eq.~(\ref{p-cm-1}), one may conclude that
\begin{equation}
  \sinh \chi = \frac{\sqrt{ 2 \Ezeroi \Tlabin + \Tlabin^2}}
    {\sqrt{( \Ezerot + \Ezeroi )^2 + 2 \Ezerot \Tlabin }}.
  \label{lab-to-cm-2}
\end{equation}
Note that except for incident gammas, $\Tlabin$ is much smaller
than the rest mass $\Ezeroi$, so that $\chi$ is a
small, positive number.

In the next section of this appendix, for 2-body problems
the center-of-mass energy and momentum of the
emitted particle and residual are determined.  In order to boost these 4-vectors
to the laboratory frame, one may use the inverse of the matrix $R$ in
Eq.~(\ref{R-3}), so that
\begin{equation}
  (E_{\text{lab}}, p_{\text{lab}})^T = R^{-1}(E_{\text{cm}}, p_{\text{cm}})^T
 \label{3d-map}
\end{equation}
with
\begin{equation}
   R^{-1} =
    \begin{bmatrix}
     \cosh \chi & \sinh \chi & 0 & 0 \\
     \sinh \chi & \cosh \chi & 0 & 0 \\
     0  & 0 & 1 & 0 \\
     0  & 0 & 0 & 1
  \end{bmatrix}.
  \label{R-3-inv}
\end{equation}

\subsection{Incident photons}
\label{Sec:photon-in}
When the incident particle is a photon, the boost from the 
center-of-mass frame to the laboratory frame must be determined
relativistically, because the mass of the incident particle is zero
but its momentum is nonzero.

In this case, Eq.~(\ref{p-lab-in}) simplifies to
$$
  | \plabin |  =  \Tlabin,
$$
and Eq.~(\ref{lab-to-cm-2}) becomes
$$
  \sinh \chi = \frac{ \Tlabin}
    {\sqrt{ \Ezerot^2 + 2 \Ezerot \Tlabin }}.
$$
It follows that
$$
  \cosh \chi = \frac{ \Ezerot + \Tlabin}
    {\sqrt{ \Ezerot^2 + 2 \Ezerot \Tlabin }}.
$$



\section{Outgoing particles}
Denote by $\Ezeroe$ the rest mass of the
emitted particle and
$\Tcme$ its kinetic energy in the center-of-mass frame.
The convention in \xendl\ is that the energy~$Q$ of the reaction
is specified by the data, and the rest mass  $\EzeroR$ of the residual
is calculated from
\begin{equation}
  \EzeroR = \Ezerot + ( \Ezeroi - \Ezeroe ) - Q.
 \label{Q-2body}
\end{equation}
Let $\TcmR$ be the kinetic energy of the residual in the center-of-mass frame.
In terms of these variables, the energy-momentum invariant
for the system is the square of the total energy
$$
    S = (\Ezeroe + \Tcme + \EzeroR + \TcmR)^2,
$$
with the same value of $S$ as in Eq.~(\ref{spacetime-cm}).
The argument leading to Eq.~(\ref{p-cm}) shows that the
momentum~$\pcme$ of the emitted particle in the center-of-mass frame
has magnitude given by
\begin{equation}
  \pcme^2 = \frac{ \left[S - (\EzeroR^2 + \Ezeroe^2)\right]^2 -
                 4 \EzeroR^2 \Ezeroe^2 }
                {4 S }.
  \label{p-cm-e}
\end{equation}

It is not a good idea to use Eq.~(\ref{p-cm-e}) in a
computation, because of its subtraction of nearly equal
numbers.  It is therefore desirable to do some algebraic manipulation
in order to mitigate this problem as much as possible.
As a first step, Eq.~(\ref{p-cm-e}) is rewritten in the
form
\begin{equation}
  4 S \pcme^2 = \left[S - (\EzeroR + \Ezeroe)^2\right]
              \left[S - (\EzeroR - \Ezeroe)^2\right].
 \label{p-cm-e-1}
\end{equation}
In this expression, the subtraction of nearly equal
numbers is confined to the first factor on the right-hand
side.  For photon emission the two factors are identical.
An analysis of photon emission later, because it offers some
simplifications.

By using the expression for~$S$ in Eq.~(\ref{S-lab}), one
obtains the relation
$$
  S - (\EzeroR + \Ezeroe)^2 =
  (\Ezerot + \Ezeroi)^2 - (\EzeroR + \Ezeroe)^2  + 2\Ezerot \Tlabin.
$$
In terms of the energy $Q$ of the discrete 2-body reaction
and the parameter
\begin{equation}
  M_T = \Ezerot + \EzeroR + \Ezeroi + \Ezeroe,
 \label{def-M}
\end{equation}
it follows that
$$
    S - (\EzeroR + \Ezeroe)^2 =
     M_T Q + 2\Ezerot \Tlabin.
$$
Consequently, it is seen that Eq.~(\ref{p-cm-e}) may be
replaced by
\begin{equation}
  \pcme^2 = \frac{ (M_T Q +  2\Ezerot \Tlabin )
                 ( M_T Q +  2\Ezerot \Tlabin + 4\EzeroR \Ezeroe)}
                {4 S }.
  \label{p-cm-e-OK}
\end{equation}

\textbf{Remark.}
It is clear from Eq.~(\ref{p-cm-e-OK}) that for endothermic
reactions ($Q < 0$), the threshold occurs when the incident
particle has kinetic energy
$$
  \Tlabin = \frac{-M_T Q}{2\Ezerot}.
$$

In Eq.~(\ref{p-cm-e-OK}) there is subtraction of nearly equal
numbers when the kinetic energy $\Tlabin$ of the incident
particle is just above the threshold in endothermic reactions.
That operation is unavoidable in
the analysis of nuclear reactions. 

Now that $\pcme^2$ has been obtained in Eq.~(\ref{p-cm-e-OK}), one may use
Eq.~(\ref{good-T}) to determine the kinetic energy of the
emitted particle in the center-of-mass frame as
\begin{equation}
  \Tcme = \frac{\pcme^2 }
          {\Ezeroe + \sqrt{ \Ezeroe^2 + \pcme^2 }}.
 \label{T-cm-e}
\end{equation}

\subsection{The boost to the laboratory frame}
It is often desired to determine the kinetic energy~$\Tlabe$
and momentum~$\plabe$ of the emitted particle in the laboratory frame
for given direction cosine~$\mucm$ in the center-of-mass frame.
It is possible to use the boost 
Eq.~(\ref{3d-map}) to determine $\plabe$ as follows.  Recall that
the form of Eq.~(\ref{3d-map}) is determined by the requirement that
the first axis of the coordinate system was chosen parallel to~$\plabin$.
Consequently, one has
$$
  \pcmea = \mucm |\pcme|.
$$
If the orientation of the coordinate system is such that
$$
  \pcmec = 0
  \quad \textrm{and} \quad
  \pcmeb \ge 0,
$$
then
$$
  \pcmeb =  |\pcme| \sqrt{ 1 - \mucm^2}.
$$
The momentum components of the boost Eq.~(\ref{3d-map}) then
take the form
\begin{equation*}
\begin{split}
  \plabea &= (\Ezeroe + \Tcme)\sinh \chi +  \mucm |\pcme| \cosh \chi, \\
  \plabeb &= |\pcme| \sqrt{ 1 - \mucm^2}, \\
  \plabec & = 0.\\
\end{split}
\end{equation*}

The magnitude of the momentum in the laboratory frame is
$$
  |\plabe| = \sqrt{ \plabea^2 + \plabeb^2 + \plabec^2 }.
$$
If $|\plabe| = 0$, the direction cosine $\mulab$ in the laboratory
frame is undetermined.  Otherwise, it is given by
$$
  \mulab = \frac{\plabea}{|\plabe|}.
$$
The kinetic energy $\Tlabe$ is calculated from $|\plabe|$
by using Eq.~(\ref{good-T}).

\subsection{Photon emission}
When the emitted particle is a photon, because $\Ezeroe = 0$,
Eqs.~(\ref{p-cm-e-OK}) and (\ref{T-cm-e}) take the simpler form
$$
  E_{e, \textrm{cm}} = \Tcme = |\pcme| =
   \frac{ M_T Q +  2\Ezerot \Tlabin }
                {2 \sqrt{S} }.
$$
For given direction cosine~$\mucm$ in the center-of-mass frame,
the energy component of the boost Eq.~(\ref{3d-map}) gives the
Doppler shift
$$
   E_{e, \textrm{lab}} = E_{e, \textrm{cm}}
     \left( \cosh \chi + \mucm \sinh \chi \right).
$$
The first component of the momentum of the photon in the laboratory
frame is
$$
  \plabea =  E_{e, \textrm{cm}}
     \left( \sinh \chi + \mucm \cosh \chi \right),
$$
so the direction cosine is
$$
  \mulab = \frac{\sinh \chi + \mucm \cosh \chi}
                {\cosh \chi + \mucm \sinh \chi}.
$$

} % end of this appendix
