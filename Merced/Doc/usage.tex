\chapter{Usage of \gettransfer}
\label{Sec:usage}
The command to run \gettransfer\ is\\
  \Input{\gettransfer\ [-inputOption]}{InputFile}\\
The input options are described in Section~\ref{Sec:usage-optional} below.
Most of them may be specified either in the input file or on the
command line, and command line options override those given in the
input file.  For example, to get output with 9 significant figures,
the command line could be\\
\Input{getTransferMatrix\ -datafield\_precision 9}{InputFile}\\
Alternatively, one may insert the line\\
   \Input{datafield\_precision: 9}{}\\
into the input file.  Note the presence of the colon in this line.
The format for identification of data in the input file is\\
  \Input{data identifier:}{value}

For most types of data, the units of energy are arbitrary, but they
must be consistent.  In particular, rest masses of particles must
be in the same units as the energy bins.  As mentioned in the
individual sections on the data, some models require that
energies be given in MeV.

\section{Output file}
The default name of the output file is \texttt{utfil}.  
It may be changed on the command line with\\
  \Input{-output}{OutputFile}\\
It may also be changed with the
line\\
  \Input{output:}{OutputFile}\\
  in the input file.

\section{Form of the input file}
The first line of the input file must be\\
  \Input{xndfgenTransferMatrix: version 1.0}{}\\
This line is followed by information common to all data models.
The file closes with data specific to the particular data model.
Blank lines are ignored.
  
\subsection{Comments}
Comments may be included in the input file in either if two forms.\\
 \Input{Comment:}{This comment is printed to the output file.}\\
 \Input{\#}{On a line, anything after a pound sign is ignored.}\\
 For example, the input file usually contains a comment identifying the
 particles involved in the reaction, e.g.,\\
   \Input{Comment: n1 + C12 --> n1 + C12 outgoing data for n1}{}
 
\subsection{Parallel computing}
The \gettransfer\ code may be compiled to run in parallel if
\textsf{OpenMP} is available on your computer.  In fact, the
default \textsf{Makefile} uses \textsf{OpenMP}.  Compile with
\textsf{Makefile\_serial} to obtain serial code.  

For the parallel code, the number of threads is set to $n$ by the
line\\
  \Input{num\_threads:}{$n$}\\
in the input file.  The default is $n = 0$, which causes the computer
to choose the number of threads.  If the specified $n$ is larger than
the number of available threads, then the code runs on the threads
available.

The parallel code may be forced to run in serial mode by
the command line  option\\
 \Input{-num\_threads~1} {}\\
or by inclusion of\\
 \Input{num\_threads:~1}{}\\
in the input file.

 \subsection{Interpolation flags} \label{interp-flags}
The identifiers for the standard interpolation methods
 given in Section~\ref{Sec:1d-interp} are\\
    \Input{flat}{for histograms}\\
   \Input{lin-lin}{for linear-linear}\\
   \Input{lin-log}{for linear-log}\\
   \Input{log-lin}{for log-linear}\\
   \Input{log-log}{for log-log}\\
The identifiers are incorporated in different ways into the interpolation
flags for simple lists such as reaction cross sections, for probability densities,
and for the Kalbach-Mann $r$ and~$a$ parameters.  The complete
identifiers for the interpolation of the various types of tabulated data are as follows.

\subsubsection{Interpolation flags for simple lists}\label{interp-flags-list}
 For simple lists of data such as $\{E, M\}$ for particle multiplicity~$M$ at
 incident energy~$E$, the interpolation flags are\\
   \Input{Interpolation: }{identifier}\\
with one of the identifiers above.  For example, the command\\
   \Input{Interpolation: lin-lin}{}\\
specifies linear-linear interpolation.

\subsubsection{Interpolation flags for probability densities}\label{interp-flags-probability}
For probability density tables 
\begin{equation}
  \{x, y, \pi(y \mid x) \},
 \label{xypi-data}
\end{equation}
the interpolation
method with respect to~$x$ as discussed in Section~\ref{Sec:2d-interp}
is identified by\\
   \Input{$x$ interpolation:}{identifier interpolation-flag}\\
where the identifier is one of those for simple lists and the
interpolation flag here is one of\\
   \Input{direct }{for direct interpolation with extrapolation}\\
   \Input{unitbase }{for unit-base interpolation}\\
   \Input{cumulativepoints }{for interpolation by cumulative points}\\
Thus, a table of probability densities of outgoing energies $\pi( E' \mid E)$ may
be marked\\
   \Input{Incident energy interpolation: lin-lin cumulativepoints}{}\\
to indicate that interpolation with respect to incident energy~$E$ is to be done
using linear-linear cumulative points as in Section~\ref{Sec:cumProb}.

The method for interpolation of the data in Eq.~(\ref{xypi-data}) with
respect to $y$ is specified by\\
   \Input{$y$ interpolation:}{identifier}\\
with an identifier as in a simple list.  For example, the command\\
   \Input{Outgoing energy interpolation: flat}{}\\
specifies histogram interpolation with respect to the energy of
the outgoing particle.

\subsubsection{Interpolation flags for unscaled interpolation of
Kalbach-Mann data}\label{interp-flags-Kalbach-r}
The methods of interpolation of tables for the Kalbach-Mann 
parameters $r(E', E)$ and $a(E', E)$ with respect to the energy~$E$
of the incident particle are discussed in Section~\ref{Sec:Kalbach-r-interp}.
The options for interpolation flags are\\
   \Input{Incident energy interpolation:}{identifier \texttt{unscaleddirect}}\\
   \Input{Incident energy interpolation:}{identifier \texttt{unscaledunitbase}}\\
   \Input{Incident energy interpolation:}{identifier \texttt{unscaledcumulativepoints}}\\
where the identifier is one of those for simple lists. 
For example, the command denoting linear-linear unscaled unit-base interpolation
with respect to incident energy is\\
   \Input{Incident energy interpolation: lin-lin unscaledunitbase}{}

The method for interpolation of tables of Kalbach-Mann $r$ and~$a$ parameters
with respect to outgoing energy~$E$ is specified by\\
   \Input{Outgoing energy interpolation:}{identifier}\\
with an identifier as in a simple list.  For example, the comand\\
   \Input{Outgoing energy interpolation: flat}{}\\
specifies histogram interpolation with respect to the energy of
the outgoing particle.

\section{Information used by all data models}
The following information is required, but the order
is arbitrary.

\subsection{The data model} \label{data-model}
Identification of the data model.\\
  \Input{Process:}{Identifier of the type of data}\\
These identifiers are specified in the previous sections.

\subsection{Incident energy groups}\label{Ein-bins}
The boundaries of the incident energy groups.\\
  \Input{Projectile's group boundaries: n = $n$}{}\\
This is followed by the $n$ values the incident energy bin
boundaries.  Thus, in units of eV this section may take the form\\
  \Input{Projectile's group boundaries: n = 88}{}\\
  \Input{  1.306800000000e-03  2.090800000000e-02  1.306800000000e-01}{}\\ 
  \Input{  3.345300000000e-01  1.176100000000e+00  2.090800000000e+00}{}\\
   \Input{ \indent $\cdots$}{}\\
  \Input{  2.000000000000e+07 }{}

\subsection{Outgoing energy groups}\label{Eout-bins}
The boundaries of the outgoing energy groups.\\
  \Input{Product's group boundaries: n = $n$}{}\\
This is followed by the $n$ values the outgoing energy bin
boundaries.  The units must be the same as for the incident
energy groups.  A sample input given in eV is\\
  \Input{Product's group boundaries: n = 88}{}\\
  \Input{  1.306800000000e-03  2.090800000000e-02  1.306800000000e-01}{}\\ 
  \Input{  3.345300000000e-01  1.176100000000e+00  2.090800000000e+00}{}\\
   \Input{ \indent $\cdots$}{}\\
  \Input{  2.000000000000e+07 }{}

\subsection{Frames of reference}\label{Reference-frame}
The energy $E$ of the incident particle must be given in the laboratory frame,
as indicated by the command\\
  \Input{Projectile Frame: lab}{}\\
For the outgoing particle, the energy $E'$ and direction cosine~$\mu$
may be given in the laboratory frame with\\
  \Input{Product Frame: lab}{}\\
or the center-of-mass frame as\\
  \Input{Product Frame: CenterOfMass}{}

\subsection{Relativistic kinetics}\label{Sec:relativistic}
For discrete 2-body reactions, 
the code may use either Newtonian or relativistic mechanics in its
computations.  The command to control this option is\\
  \Input{kinetics: Newtonian}{}\\
or\\
  \Input{kinetics: relativistic}{}\\
The default is \texttt{Newtonian} except when the emitted particle is a gamma.

\subsection{Approximate flux}
The Legendre coefficients $\widetilde \phi_\ell(E)$ used as weights in
the integrals Eqs.~(\ref{Inum}) and~(\ref{Ien}).\\
  \Input{Fluxes: n = $n$}{}\\
  \Input{Interpolation:}{interpolation flag}\\
Here, $n$ is the number of incident energies $E$, and
the interpolation flag is one of those for simple lists, Section~\ref{interp-flags-list}.
Note that because of the scaling performed in Eqs.~(\ref{cons_num})
and~(\ref{cons_en}), the units of $\widetilde \phi_\ell(E)$
are arbitrary, but barns are most common.  The computed transfer matrix is unchanged if
$\widetilde \phi_\ell(E)$ is multiplied by a constant.

For each incident energy, the input file has a block specifying
the Legendre coefficients  given as\\
 \Input{ Ein: $E$:}{  n = $n$}\\
and $\widetilde \phi_\ell(E)$ for $n = 0$, 1, \ldots , $n - 1$.
The incident energy $E$ must be in the same units as the
energy groups.
The number of Legendre coefficients given here need not be
consistent with the Legendre order $L$ of the computed transfer
matrix as specified in Section~\ref{Sec:LegendreOrder}.  
If $n - 1 < L$, then the {\gettransfer} code sets
$$
  \widetilde \phi_\ell(E) = \widetilde \phi_{n - 1}(E)
  \quad \text{for $\ell = n$, $n+1$, \ldots, $L$.}
$$

A sample input with $E$ in eV and with all Legendre coefficients
the same is\\
  \Input{Fluxes: n = 2}{}\\
  \Input{Interpolation: lin-lin}{}\\
  \Input{ Ein: 0:  n = 1}{}\\
  \Input{  \indent 8.500000000000e+01}{}\\
  \Input{ Ein: 2.100000000000e+07:  n = 1}{}\\
  \Input{  \indent 8.500000000000e+01}{}

\subsection{Reaction cross section}
As explained in Section~\ref{Sec:gamma-in}, the cross sections for coherent photon
scattering and Compton scattering are computed from the data.
Reaction cross sections are required for all other data models.\\
  \Input{Cross section: n = $n$}{}\\
  \Input{Interpolation:}{interpolation flag}\\
Here, $n$ is the number of pairs $\{E, \sigma(E)\}$,
and the interpolation flag is one of those for simple lists,
Section~\ref{interp-flags-list}.  This is
followed by $n$ pairs of incident energy~$E$ and reaction cross
section~$\sigma( E )$.

A sample of such data with energies in MeV is given by\\
 \Input{Cross section: n = 22}{}\\
 \Input{Interpolation: lin-lin}{}\\
  \Input{  \indent  7.78148000e+00  0.00000000e+00}{}\\
  \Input{  \indent  7.80000000e+00  4.40157000e-04}{}\\
  \Input{  \indent  8.00000000e+00  1.13781000e-02}{}\\
  \Input{  \indent  8.50000000e+00  6.96097100e-02}{}\\
  \Input{  \indent }{ $\cdots$}\\
  \Input{  \indent  2.00000000e+01  9.17094100e-01}{}

\subsection{Multiplicity}
The multiplicity of the outgoing particle must be given if it is
different from~1.  The format is\\
  \Input{Multiplicity: n = $n$}{}\\
  \Input{Interpolation:}{interpolation flag}\\
followed by $n$ pairs $\{E, M(E)\}$.  
The interpolation flag is one of those for simple lists,
Section~\ref{interp-flags-list}.  The units of incident energy
$E$ must be the same as for the energy groups.  
For example, with $E$ in MeV, an $(n, 2n)$ reaction
would typically have\\
  \Input{Multiplicity: n = 2}{}\\
  \Input{Interpolation:}{flat}\\
  \Input{0.0    2.0}{}\\
  \Input{20.0  2.0}{}


\subsection{Model weight}
The model weight $w_r(E)$ is used in the formation of the reaction
kernel $\calK_r(E', \mu \mid E)$ in Eq.~(\ref{def_pi}) and is
discussed in Section~\ref{Sec:model-weight}.  Its value
is usually~1 over the entire range of incident energies in the cross
section data.  If this is not the case, then the model weight is input as\\
  \Input{Weight: n = $n$}{}\\
  \Input{Interpolation: flat}{}\\
followed by $n$ pairs $\{E, w(E)\}$.   For example, the weight
$$
  w(E) = \begin{cases}
    0 & \text{for $0 \le E < 6$,}\\
    1 & \text{for $6 \le E \le 20$,}\\
  \end{cases}
$$
may be specified using the input\\
   \Input{Weight: n = 3}{}\\
  \Input{Interpolation: flat}{}\\
  \Input{0.0    0.0}{}\\
  \Input{6.0    1.0}{}\\
  \Input{20.0  1.0}{}
  
\section{Optional flags, output information}
\label{Sec:usage-optional}
The following options control the output of \gettransfer.
All of them may also be input as command line options.

\subsection{Legendre order of the output}\label{Sec:LegendreOrder}
The Legendre order of the matrices Eqs.~(\ref{Inum}) and~(\ref{Ien})
computed by \gettransfer\ is set by the command\\
  \Input{outputLegendreOrder: n = $L$}{}\\
The default is $L = 3$.

\subsection{Numerical precision of the output}
The number of significant figures for the output data is set by\\
  \Input{datafield\_precision: n = $n$}{}\\
The default is $n = 8$.

\subsection{Conservation flag}\label{Sec:conserveFlag}
This flag determines whether the \gettransfer\ code computes integrals
   Eq.~(\ref{Inum}) for the number-conserving transfer matrix,
   integrals  Eq.~(\ref{Ien}) for the energy-conserving matrix, or both.
   The options are\\
   \Input{Conserve: number}{}\\
for the integrals Eq.~(\ref{Inum})\\
   \Input{Conserve: energy}{}\\
for the integrals Eq.~(\ref{Ien})\\
   \Input{Conserve: both}{}\\
for the both integrals.  The default is \texttt{both} for most types of data.

\subsection{Consistency check}
If the integrals Eq.~(\ref{Inum}) for the number-preserving transfer matrix
are computed, it is possible to check the consistency as in Eq.~(\ref{rowSum}).
With the option\\
  \Input{check\_row\_sum: true}{}\\
both sides of Eq.~(\ref{rowSum}) are printed, along with their differences
and relative differences.  This information is not printed if the option is \texttt{false}.
The default is \texttt{false}.

To scale the integrals Eq.~(\ref{Inum}) so as to enforce the identity
Eq.~(\ref{rowSum}), use the option\\
  \Input{scale\_rows: true}{}\\
In this case, the integrals Eq.~(\ref{Ien}) are also scaled.  The default
is \texttt{true}, scale the integrals.

\section{Optional inputs, quadrature methods}\label{Sec:QuadratureMethods}
For the integrals Eqs.~(\ref{Inum}) and~(\ref{Ien}) and their equivalents
in the center-of-mass frame, the quadrature methods may be set
by commands of the form\\
  \Input{\textrm{Variable} quadrature method:}{Method}\\
The `Variable' in this line is any of the following.\\
  \Input{Ein}{for integrals with respect to incident energy~$E$,}\\
  \Input{Eout}{for integrals with respect to outgoing energy~$\Elab'$ or~$\Ecm'$,}\\
  \Input{mu}{for integrals with respect to direction cosine~$\mulab$ of ~$\mucm$.}\\
  \Input{-}{Omission of this parameter gives the same quadrature method for all integrals.}\\
The options for the `Method' in the command to set the quadrature method
are\\
  \Input{adaptive}{for adaptive 2nd-order Gaussian quadrature,}\\
  \Input{square root}{for adaptive 1st-order Gaussian quadrature with weight $\sqrt{1 - \mu}$,}\\
  \Input{Gauss2}{for non-adaptive 2nd-order Gaussian quadrature,}\\
  \Input{Gauss4}{for non-adaptive 4th-order Gaussian quadrature,}\\
  \Input{Gauss6}{for non-adaptive 6th-order Gaussian quadrature.}\\
The default for most data models is\\
  \Input{quadrature method: adaptive}{}\\
to use adaptive 2nd-order Gaussian  quadrature for all integrals.
The exceptions are\\
  \Input{mu quadrature method: square root}{}\\
for the integrals over direction cosine $\mulab$ with data for
coherent scattering and Compton scattering.  The non-adaptive options
are primarily used in debugging.

\section{Optional inputs, numerical tolerances}
The user may reset the tolerances for convergence of the adaptive quadrature
and for determination of the equality of two floating-point numbers.

\subsection{Convergence of adaptive quadrature}
The adaptive quadrature routine produces an estimate $\calI$ of
the integral, along with an estimate $e_\calI$ of the error.  The process of
successive subdivision stops when
$$
  | e_\calI | < \epsilon_a + \epsilon_r | \calI |.
$$
The absolute quadrature tolerance is set by the command\\
  \Input{abs\_quad\_tol:}{$\epsilon_a$}\\
The default is $\epsilon_a = \texttt{1.0e-8}$.  To set the relative
quadrature tolerance, use\\
  \Input{quad\_tol:}{$\epsilon_r$}\\
The default is $\epsilon_r = \texttt{1.0e-4}$. 

There is also a limit on the total number of intervals used in
adaptive quadrature\\
  \Input{max\_divisions: n = $n$}{}\\
If this limit is exceeded, the adaptive quadrature routine returns
the current estimate and prints a warning that this result may be
inaccurate.

\subsection{Near equality of floating-point numbers}
In comparisons of floating-point numbers $x_1$ and~$x_2$, the code treats 
them as essentially equal if
$$
  | x_1 - x_2 | \le \delta_a + \delta_r \min( | x_1 |, | x_2 | ).
$$
Here, the absolute tolerance $\delta_a$ is set by\\
  \Input{abs\_tol:}{$\delta_a$}\\
The default value is $\delta_a = \texttt{2.0e-14}$ and is appropriate
when energies are measured in MeV.  It should be scaled accordingly,
when other energy units are used.  The relative tolerance
$\delta_r$ is set using\\
  \Input{E\_tol:}{$\delta_r$}\\
The default value is $\delta_r = \texttt{1.0e-9}$.

\section{Physical constants}
The coding for coherent photon scattering and Compton scattering
discussed in Section~\ref{Sec:gamma-in} requires the values of several physical
constants.  These are input as follows.

\subsection{Conversion from $\text{\AA}^{-1}$ to energy}\label{Sec:cm-to-Mev}
In order to convert the energy of photons from inverse wavelength
to energy, multiply by $ch$.  This parameter is set by
the command\\
  \Input{inverseWaveLengthToEnergyFactor:}{$ch$}.

\subsection{Thompson scattering cross section}\label{Sec:Thompson-xs}
The Thompson scattering cross section $\sigma_T$ in 
Eqs.~(\ref{sigma-whole-atom}) and~(\ref{Compton-xs}) is
set by\\
  \Input{ThompsonScattering:}{$\sigma_T$}\\
The default value is $\sigma_T = \texttt{0.6652448}$ barns.

\subsection{Electron rest mass}\label{Sec:electron-mass}
The rest mass $m_e$ of the electron in Eq.~(\ref{Compton-relative-E}) is
set by the command\\
  \Input{electron mass:}{$m_e$}.

\subsection{Neutron rest mass}\label{Sec:neutron-mass}
The rest mass of the neutron $m_n$ is used in Eq.~(\ref{Kalbach-photo})
by the Kalbach-Mann model of photo-nuclear reactions.
Its units are MeV, and its value is set by the command\\
  \Input{m\_neutron:}{$m_n$}\\
Its default value is $m_n = \texttt{939.565653471}\,$MeV.

\section{Errors and warning messages}
These options control the printing of informational messages,
warnings, and fatal errors.  To set which messages are printed,
use the command\\
  \Input{message\_level: n = $n$}{}\\
The effect of this option is:
$$
  \texttt{message\_level} =
    \begin{cases}
       0& \text{print all messages},\\
       1& \text{print only warnings and errors},\\
       2& \text{print only severe errors; these cause exits anyway}.
    \end{cases}
$$
The default value is 0, print all messages.

It is also possible to turn off all messages with the command\\
  \Input{skip\_logging:}{\texttt{true}}\\
The default value is \texttt{false}.

\section{Model-dependent information}\label{model-info}
The remainder of the input file consists of data required by the model. 
