\chapter{Legendre expansions of energy-angle probability densities
in the laboratory frame}
\label{Sec:Legendre-lab}
Another representation of joint energy-angle probability densities
$\pi(E', \mu \mid E)$ in \xendl\ is as a table of the Legendre 
coefficients $\pi_\ell(E' \mid E)$ in the expansion
\begin{equation}
  \pi( E', \mu \mid E) = \sum_\ell
  \left(
    \ell + \frac{1}{2}
  \right)
  \pi_\ell( E' \mid E) P_\ell( \mu ).
 \label{piLegendre}
\end{equation}
Here, $E$ denotes the energy of the incident particle in the
laboratory frame.  For the outgoing particle, the energy $E'$ and 
direction cosine $\mu$ may be given in either center-of-mass or laboratory
coordinates.  The treatment of laboratory-frame data is discussed in this section, 
center-of-mass data in the next.  Data given in the laboratory frame is
much easier to deal with because no boost is involved.

This type of data is ordered according to
\begin{equation}
\{ E, \{ E', \{ \pi_\ell(E' \mid E) \} \} \}.
 \label{ENDF-I4}
\end{equation}
All of the data for the lowest incident energy $E$
is given first, ordered according to outgoing energy~$E'$.  For
given values of $E$ and $E'$, the data consist of Legendre coefficients
$\pi_\ell(E', \mid E)$.  Note that for this data format, the number
of Legendre coefficients may vary, depending on the energies $E$ and~$E'$.

The {\gettransfer} code also handles data for Legendre expansions of
energy-angle probability densities in the {\ENDL} format~\cite{Omega},
\begin{equation}
\{ \ell, \{ E, \{ E', \pi_\ell(E', \mid E) \} \} \}.
 \label{ENDL-I4}
\end{equation}
That is, the $\ell = 0$ data are given first, ordered according to
incident energy~$E$.  The data then consist of pairs
$\{ E', \pi_\ell(E', \mid E) \}$ for given $\ell$ and $E$.
%This data format is deprecated, however.

\section{Computation of the transfer matrices for data in the laboratory frame}
The calculation of the transfer
matrices for laboratory-frame data proceeds as follows.
In terms of $\pi_\ell(\Elab'   \mid E)$,
the integral Eq.~(\ref{Inum}) for the number-preserving transfer matrix
takes the form
\begin{equation}
   \Inum_{g,h,\ell} =
     \int_{\calE_g} dE \, \sigma ( E ) M(E) w(E) \widetilde \phi_\ell(E)
   \int_{\calE_h' } d\Elab'   \, \pi_\ell(\Elab'   \mid E),
 \label{InumI4}
\end{equation}
and Eq.~(\ref{Ien}) for the energy-preserving transfer matrix becomes
\begin{equation}
   \Ien_{g,h,\ell} =
     \int_{\calE_g} dE \, \sigma ( E ) M(E) w(E) \widetilde \phi_\ell(E) 
     \int_{\calE_h' } d\Elab'   \, \pi_\ell(\Elab'   \mid E) \Elab'  .
 \label{IenI4}
\end{equation}

Computation of the integrals Eqs.~(\ref{InumI4}) and~(\ref{IenI4})
depends on the type of interpolation used with respect to the
energy $E$ of the incident particle, and the procedures are
exactly the same as for integration in Eqs.~(\ref{InumI4-0}) 
and~(\ref{IenI4-0}) of the isotropic energy probability densities
$\pi_0(\Elab'   \mid E)$.  Thus, if unit-base interpolation is to
be used for $\pi_\ell(\Elab'   \mid E)$, then the map Eq.~(\ref{unitbaseMap})
converts 
the integrals Eqs.~(\ref{InumI4}) and~(\ref{IenI4}) to the form
\begin{equation}
   \Inum_{g,h,\ell} =
     \int_{\calE_g} dE \, \sigma ( E ) M(E) w(E) \widetilde \phi_\ell(E) 
   \int_{\widehat\calE_h' } d\widehat \Elab'   \,
     \widehat\pi_\ell(\widehat \Elab'   \mid E)
 \label{InumhatI4}
\end{equation}
and
\begin{equation}
   \Ien_{g,h,\ell} =
     \int_{\calE_g} dE \, \sigma ( E ) M(E) w(E) \widetilde \phi_\ell(E) 
   \int_{\widehat\calE_h' } d\widehat \Elab'   \,
     \widehat\pi_\ell(\widehat \Elab'   \mid E) \Elab'  .
 \label{IenhatI4}
\end{equation}
In these intergrals $\widehat\calE_h'$ denotes result of mapping the 
outgoing energy bin $\calE_h'$ with the transformation Eq.~(\ref{unitbaseMap}).
Furthermore, $\Elab'  $ in Eq.~(\ref{IenhatI4}) is to be obtained from $\widehat \Elab'  $
using the inverse unit-base mapping Eq.~(\ref{unitbaseInvert}).

The geometrical considerations involved in integrating 
Eqs.~(\ref{InumhatI4}) and~(\ref{IenhatI4}) over the incident energy bin~$\calE_g$
and the mapped outgoing energy bin~$\widehat\calE_h'$ are illustrated
in Figure~\ref{Fig:unit-base-region}.

\section{Form of the input file for Legendre coefficient data in the laboratory frame}
These data may be input in either of two forms, the format 
in Eq.~(\ref{ENDF-I4}) from \ENDF\ with all Legendre
coefficients given together at each incident energy $E$ and outgoing energy~$E'$
or that  in Eq.~(\ref{ENDL-I4}) with
one Legendre order at a time.
For both formats, all energies must be in the same units as the energy groups.

\subsection{Input of all Legendre coefficients together}\label{Sec:ENDF-I4-data}
For energy-angle tables in the standard format of Eq.~(\ref{ENDF-I4}), 
the Section~\ref{data-model} line in the input
file to identify the data is\\
      \Input{Process: Legendre energy-angle data}{}\\
and the model-dependent data in Section~\ref{model-info} consists of the
Legendre coefficients $\pi_\ell(E' \mid E)$ in Eq.~(\ref{ENDF-I4}) at incident energies~$E$
and outgoing energies~$E'$.

The format for the Legendre coefficient data in
Section~\ref{model-info} given at $K$ values of $E$ is\\
  \Input{Product Frame: lab}{}\\
  \Input{Legendre data by incident energy:}{$n = K$}\\
  \Input{Incident energy interpolation:}{probability interpolation flag}\\
  \Input{Outgoing energy interpolation:}{list interpolation flag}\\
where the interpolation flag for incident energy is one for probability density
tables as in Section~\ref{interp-flags-probability}, and that for outgoing energy 
is for a simple list.
These lines are followed by $K$ sections of the form\\
  \Input{ Ein: $E$:}{\texttt{n = $J_k$}}\\
for $J_k$ outgoing energies $E$.  For each value of $E$ there is
data\\
  \Input{  Eout: $E'$:}{\texttt{n = $L$}}\\
with Legendre coefficients $\pi_\ell(E' \mid E)$ for $\ell = 0$, 1, \ldots\ , $L - 1$.

An example of these data with energies in MeV is\\
  \Input{Legendre data by incident energy:  n = 26}{}\\
   \Input{Incident energy interpolation: lin-lin cumulativepoints}{}\\
  \Input{Outgoing energy interpolation: flat}{}\\
   \Input{Ein: 1.140200e+01:  n = 2}{}\\
  \Input{  Eout: 0.000000e+00:  n = 5}{}\\
 \Input{ \indent  1.000000e+11}{}\\
 \Input{ \indent  0.000000e+00}{}\\
 \Input{ \indent  0.000000e+00}{}\\
   \Input{ \indent  0.000000e+00}{}\\
   \Input{ \indent  0.000000e+00}{}\\
    \Input{ Eout: 1.000000e-11:  n = 5}{}\\
   \Input{ \indent  0.000000e+00}{}\\
   \Input{ \indent  0.000000e+00}{}\\
   \Input{ \indent  0.000000e+00}{}\\
   \Input{ \indent  0.000000e+00}{}\\
   \Input{ \indent  0.000000e+00}{}\\
\Input{}{$\cdots$}\\
   \Input{Ein: 2.000000e+01:  n = 27}{}\\
    \Input{ Eout: 0.000000e+00:  n = 5}{}\\
   \Input{ \indent  4.179200e-02}{}\\
   \Input{ \indent  0.000000e+00}{}\\
   \Input{ \indent  4.179200e-06}{}\\
   \Input{ \indent  0.000000e+00}{}\\
   \Input{ \indent  3.395500e-07}{}\\
 \Input{ \indent}{ etc.}
 
 \subsection{Input of one Legendre coefficient at a time}\label{sec:ENDL-I4}
For data given one Legendre coefficient at a time as in Eq.~(\ref{ENDL-I4}),
the line in Section~\ref{data-model} of the input
file identifying the data is\\
      \Input{Process: Legendre EEpP data transfer matrix}{}\\
The first lines in the data for Section~\ref{model-info} are\\
  \Input{Product Frame: lab}{}\\
 \Input{LEEpPData:}{$n = L$}\\
where $L$ is the number of Legendre coefficients, one greater than the
order of the Legendre expansion.  The interpolation flags as
in Section~\ref{interp-flags-probability} are\\
   \Input{Incident energy interpolation:}{probability interpolation flag}\\
  \Input{Outgoing energy interpolation:}{list interpolation flag}\\
The interpolation flag for incident energy is one for probability density
tables as in Section~\ref{interp-flags-probability}, and that for outgoing energy 
is for a simple list.

The data are then given in $L$ sections, each of the form\\
   \Input{ order: l = $\ell$:}{\texttt{n = $K$}}\\
where $K$ is the number of incident energies.
For each incident energy $E'$ there is a block of data\\
    \Input{ Ein:}{$E$: \quad \texttt{n = $J_k$}}\\
for $J_k$ pairs of values of outgoing energy $E'$ and
Legendre coefficient $\pi_\ell(E' \mid E)$.  For energies measured in MeV,
these data may look like\\
  \Input{LEEpPData: n = 4}{}\\
  \Input{Incident energy interpolation: lin-lin unitbase}{}\\
  \Input{Outgoing energy interpolation: lin-lin}{}\\
  \Input{order: l = 0: n = 10}{}\\
   \Input{Ein:  3.350000000000e+00 : n = 3}{}\\
  \Input{ \indent   3.716500000000e-01   0.000000000000e+00}{}\\
  \Input{ \indent   3.716800000000e-01   2.857140000000e+04}{}\\
  \Input{ \indent   3.717200000000e-01   0.000000000000e+00}{}\\
  \Input{Ein:  4.460200000000e+00 : n = 2}{}\\
  \Input{ \indent   1.238900000000e-01   1.008970000000e+00}{}\\
   \Input{ \indent  1.115000000000e+00   1.008970000000e+00}{}\\
 \Input{$\cdots$}{}\\
  \Input{Ein:  2.000000000000e+01 : n = 2}{}\\
     \Input{ \indent  1.699600000000e-02  1.232820000000e-01}{}\\
     \Input{ \indent  8.128500000000e+00  1.232820000000e-01}{}\\
   \Input{order: l = 1: n = 10}{}\\
    \Input{Ein:  3.350000000000e+00 : n = 3}{}\\
     \Input{ \indent  3.716500000000e-01  0.000000000000e+00}{}\\
     \Input{ \indent  3.716800000000e-01  2.690500000000e+04}{}\\
     \Input{ \indent  3.717200000000e-01  0.000000000000e+00}{}\\
\Input{$\cdots$}{}\\
    \Input{order:l = 3: n = 10}{}\\
    \Input{Ein:  3.350000000000e+00 : n = 3}{}\\
     \Input{ \indent  3.716500000000e-01  0.000000000000e+00}{}\\
     \Input{ \indent  3.716800000000e-01  2.690500000000e+04}{}\\
     \Input{ \indent  3.717200000000e-01  0.000000000000e+00}{}\\
\Input{$\cdots$}{}\\
  \Input{Ein:  2.000000000000e+01 : n = 28}{}\\
     \Input{ \indent  1.699600000000e-02  1.172400000000e-01}{}\\
     \Input{ \indent  3.283800000000e-02 -8.646000000000e-03}{}\\
     \Input{ \indent  4.868100000000e-02 -3.589400000000e-02}{}\\
     \Input{ \indent  6.452400000000e-02 -4.528500000000e-02}{}\\
     \Input{ \indent  8.036700000000e-02 -4.921500000000e-02}{}\\
     \Input{ \indent  1.120500000000e-01 -5.186400000000e-02}{}\\
    \Input{ \indent }{$\cdots$}\\
    \Input{ \indent  7.082900000000e+00  7.783200000000e-02}{}\\
     \Input{ \indent  8.128500000000e+00  1.172400000000e-01}{}

 
 