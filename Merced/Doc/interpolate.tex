\chapter{Interpolation of the data}
\label{Sec:interpolate}
The data in \xendl\ representing the probability density
$\pi(E', \mu \mid E ) = \pi_r(E', \mu \mid E )$ in the integrals (\ref{Inum}) and~(\ref{Ien})
are given in various forms.  In the case of tabulated data, intermediate
values must be obtained via some sort of interpolation.  Interpolation
with respect to one independent variable is described first, followed by a discussion of
the 2-dimensional case.  In \xendl\ full 3-dimensional interpolation
of $\pi(E', \mu \mid E )$ data is reduced to a sequence of 2-dimensional
interpolations.

\section{Interpolation methods for a single variable}\label{Sec:1d-interp}
For the sake of having a specific application,
the discussion here is given in terms of tables of data $\{E_i, f(E_i)\}$,
with values $E_i$ of the energy of the outgoing particle as independent variable.
These ideas are applicable to one dimension for any tabular data.
The types of interpolation method used in \xendl\ 
for such tables are: histogram, linear-linear, 
log-linear, linear-log, and log-log.  The algorithms for interpolation
of $F(E)$ on an interval $E_0 < E < E_1$ with given $f(E_0)$ and~$f(E_1)$
are as follows.  In these definitions it is assumed that the argument of a
logarithm is positive.

\subsection{Histograms}
For histogram interpolation set
$$
  f(E) = f(E_0) \quad \text{for $E_0 \le E < E_1$}.
$$

\subsection{Linear-linear}
 For linear-linear interpolation set
\begin{equation}
   \alpha = \frac{E - E_0}{E_1 - E_0}
 \label{def-alpha}
\end{equation}
and take
$$
  f(E) = (1 - \alpha)f(E_0) + \alpha f(E_1)
   \quad \text{for $E_0 \le E \le E_1$}.
$$

\subsection{Log-linear}
For log-linear interpolation take $\alpha$ as in Eq.~(\ref{def-alpha}),
and set
$$
 \log f(E) = (1 - \alpha)\log f(E_0) + \alpha \log f(E_1)
   \quad \text{for $E_0 \le E \le E_1$}.
$$
This relation may also be written as
\begin{equation}
  f(E) = f(E_0)^{1 - \alpha} f(E_1)^\alpha.
 \label{loglin-interp}
\end{equation}

\subsection{Linear-log}
 For linear-log interpolation set
\begin{equation}
   \alpha' = \frac{\log(E / E_0)}{\log( E_1 / E_0 )}
 \label{def-alpha-prime}
\end{equation}
and take
$$
  f(E) = (1 - \alpha')f(E_0) + \alpha' f(E_1)
   \quad \text{for $E_0 \le E \le E_1$}.
$$

\subsection{Log-log}
For log-log interpolation take $\alpha'$ as in Eq.~(\ref{def-alpha-prime}),
and set
$$
 \log f(E) = (1 - \alpha')\log f(E_0) + \alpha' \log f(E_1)
   \quad \text{for $E_0 \le E \le E_1$}.
$$
This is equivalent to
\begin{equation}
  f(E) = f(E_0)^{1 - \alpha'} f(E_1)^{\alpha'}.
 \label{loglog-interp}
\end{equation}

\textbf{Remark.}
With log-linear interpolation written in the form of Eq.~(\ref{loglin-interp})
and log-log interpolation written as Eq.~(\ref{loglog-interp}), it is permitted
that $f(E_0) = 0$ or $f(E_1) = 0$.  These cases all lead to the result that
$f(E) = 0$ for $E_0 < E < E_1$, however.

\section{Interpolation methods for probability densities}\label{Sec:2d-interp}
In order to explain the methods for interpolation of probability densities, it suffices to
consider a table of values $\pi( E' \mid E )$
\begin{equation}
  \{ E'_{j,k}, \pi(E'_{j,k} \mid E_k )\} \quad \text{for $j = 0$, 1, $\ldots\,$, $J_k$}
  \label{EPtable}
\end{equation}
given at values of the incident energy~$E_k$, for
$k = 0$,~1, $\ldots\,$,~$K$.
In Eq.~(\ref{EPtable}) it is required that the outgoing energies
be ordered
\begin{equation}
  E'_{0,k} < E'_{1,k} \le E'_{2,k} \le \cdots \le E'_{J_k-1,k} < E'_{J_k,k}.
  \label{Eout-order}
\end{equation}

The condition Eq.~(\ref{Eout-order}) permits the data of
Eq.~(\ref{EPtable}) to have equal consecutive intermediate
outgoing energies $E'_{j-1,k} = E'_{j,k}$, so that the probability
density~$\pi(E' \mid E_k)$ may have a jump discontinuity there.
Jump discontinuities are not allowed at the end points $E' = E'_{0,k}$
and~$E' = E'_{J_k,k}$.  In Eq.~(\ref{EPtable}) the possibility 
of three or more consecutive equal outgoing energies may be ruled out,
because all but the first and last would be redundant. 
The convention adopted here is that the value of $\pi(E' \mid E_k)$
at a discontinuity is the second data value
$$
 \pi(E' \mid E_k) = \pi(E'_{j,k} \mid E_k) \quad
  \text{if $E' = E'_{j-1,k} = E'_{j,k}$.}
$$

For fixed incident energy $E_k$, the rules for interpolation of 
$\pi(E' \mid E_k )$ in outgoing energy~$E'$ are as given
in Section~\ref{Sec:1d-interp}.  The following types
of interpolation with respect to $E$ are discussed in subsequent subsections:
\begin{enumerate}
 \item direct interpolation.
 \item unit-base interpolation,
 \item interpolation using cumulative points.
\end{enumerate}

The method referred to here as ``interpolation by cumulative points''
is closely related to ``interpolation by corresponding energies'' as described
in the \ENDF\ manual~\cite{ENDFB}.  
For a more-detailed discussion of 2-dimensional interpolation methods,
see the reference~\cite{interpolation}.

For a discussion of interpolation of data Eq.~(\ref{EPtable}), it suffices
to consider interpolation between incident energies $E_0$ and~$E_1$.
Thus, it is desired to interpolate to incident energy~$E$ with
$E_0 < E < E_1$ the data
\begin{equation}
\begin{split}
  \{ E'_{j,0}, \pi(E'_{j,0} \mid E_0 )\} &\quad \text{for $j = 0$, 1, $\ldots\,$, $J_0$}, \\
  \{ E'_{j,1}, \pi(E'_{j,1} \mid E_1 )\} &\quad \text{for $j = 0$, 1, $\ldots\,$, $J_1$}.
  \label{EPtables}
\end{split}
\end{equation}
The ideas presented apply equally well to interpolation of data
in Eq.~(\ref{EPtable}) between any consecutive pair of
incident energies $E_{k-1} < E_k$.

The methods of 2-dimensional interpolation are described in turn.

\subsection{Direct interpolation}\label{Sec:direct-interp}
It is common to do direct interpolation for interpolating tables of angular probability
density~$\pi( \mu \mid E)$ with respect to incident energy~$E$, because the
range of direction cosines is usually $-1 \le \mu \le 1$.  For example, in order
to determine the value of $\pi( \mu \mid E)$ for $E_0 < E < E_1$ from data
Eq.~(\ref{EPtables}), one first interpolates in $\mu$
at fixed incident energies to obtain $\pi( \mu \mid E_0)$ and~$\pi( \mu \mid E_1)$.
One then obtains the value of $\pi( \mu \mid E)$ by interpolating between
$\pi( \mu \mid E_0)$ and~$\pi( \mu \mid E_1)$.

The trouble with the application of direct interpolation to tables of 
energy distributions
 is that the range of outgoing energy~$E'$
usually depends on the incident energy~$E$.
Thus, for the data in Eq.~(\ref{EPtables}), the ranges of outgoing energies
are given by
\begin{equation}
\begin{split}
  \Eminzero' =  E'_{0,0} \quad \text{and} \quad
           \Emaxzero' = E'_{J_0,0} \quad \text{for $E = E_0$}, \\
  \Eminone' =  E'_{0,1} \quad \text{and} \quad
           \Emaxone' = E'_{J_1,1} \quad \text{for $E = E_1$}.
  \label{Eout-ranges}
\end{split}
\end{equation}

\textbf{Remark.}
In the definition of the range of outgoing energies Eq.~(\ref{Eout-ranges}),
it is natural to expect that the data in Eq.~(\ref{EPtables}) are such that
for each incident energy $E_k$ with $k = 0,$~1, $\ldots\,$,~$K$, the probability
density $\pi(E' \mid E_k )$ is not equal to zero on the entire lowest outgoing
energy range $E'_{0,k} < E' < E'_{1,k}$ or highest outgoing energy range
$E'_{J_k - 1,k} < E' < E'_{J_k,k}$.  That is, 
Eq.~(\ref{Eout-ranges}) ought to give the
actual range of outgoing energies.
Some nuclear data libraries, e.~g.,
\ENDFdata~\cite{ENDFdata}, have
data of the form Eq.~(\ref{EPtable}) which imply that
$\pi(E' \mid E_k ) = 0$ on the lowest or highest outgoing energy ranges.
The sample input data given in
Section~\ref{Sec:isotropic-table-lab} illustrates the problem.

It is convenient to describe the process of direct interpolation using
notation of set theory, with the sets
\begin{equation}
\begin{split}
  \calA_0 = \{ E': \Eminzero' \le E' \le
           \Emaxzero' \}, \\
  \calA_1 = \{ E':   \Eminone' \le E' \le
           \Emaxone' \}.
  \label{def-calA01}
\end{split}
\end{equation}
The union of these two sets is denoted by
\begin{equation}
   \calA_X = \calA_0 \cup \calA_1,
 \label{def-calA-X}
\end{equation}
and the intersection is denoted by
\begin{equation}
   \calA_T = \calA_0 \cap \calA_1,
 \label{def-calA-T}
\end{equation}

There are two obvious interpretations of direct interpolation of
the data in Eq.~(\ref{EPtables}) when the outgoing energy ranges
differ, $\calA_0 \ne \calA_1$.  One may do \textit{direct interpolation
with extrapolation} or \textit{direct interpolation
with truncation}.   Linear-linear versions of these methods are
described here. 

For direct interpolation with extrapolation the probability densities
$\pi(E' \mid E_0 )$ and $\pi(E' \mid E_1 )$ constructed from the
tables in Eq.~(\ref{EPtables})  are extrapolated to
\begin{equation}
  \pi_X(E' \mid E_0 ) = \begin{cases}
    \pi(E' \mid E_0 ) &\quad \text{for $E'$ in $\calA_0$,} \\
    0 &\quad \text{for $E'$ in $\calA_X \setminus \calA_0$,}
  \end{cases}
  \label{def-pi-X0}
\end{equation}
and
\begin{equation}
  \pi_X(E' \mid E_1 ) = \begin{cases}
    \pi(E' \mid E_1 ) &\quad \text{for $E'$ in $\calA_1$,} \\
    0 &\quad \text{for $E'$ in $\calA_X \setminus \calA_1$.}
  \end{cases}
  \label{def-pi-X1}
\end{equation}
For direct interpolation
to incident energy $E$ with $E_0 < E < E_1$,
the proportionality factor $q$ is  defined as
\begin{equation}
  q = \frac{ E - E_0}{ E_1 - E_0 }.
  \label{def-q}
\end{equation}
In linear-linear direct interpolation with extrapolation,
the interpolant is taken to be
\begin{equation}
  \pi_X(E' \mid E ) = ( 1 - q )\, \pi_X(E' \mid E_0 ) + q\, \pi_X(E' \mid E_1 )
  \label{def-pi-X}
\end{equation}
for $E'$ in the set~$\calA_X$.

The method of direct interpolation with truncation differs from
that using extrapolation, in that this method uses the truncated
probability densities
\begin{equation}
 \begin{split}
   \pi_T(E' \mid E_0 )& = C_0 \pi(E' \mid E_0 ), \\
   \pi_T(E' \mid E_1 )& = C_1 \pi(E' \mid E_1 )
  \end{split}
 \label{def-pi-T01}
\end{equation}
for outgoing energy $E'$ in the set~$\calA_T$.  Here, $C_0$ 
and~$C_1$ are normalization constants such that
$$
   \int_{\calA_T} dE' \, \pi_T(E' \mid E_0 ) = 1
   \quad \text{and} \quad
   \int_{\calA_T} dE' \, \pi_T(E' \mid E_1 ) = 1.
$$
For linear-linear direct interpolation with truncation of the data in Eq.~(\ref{EPtables})
to incident energy $E$ with $E_0 < E < E_1$, the factor~$q$
is chosen as in Eq.~(\ref{def-q}), and the interpolant is
$$
  \pi_T(E' \mid E ) = ( 1 - q )\, \pi_T(E' \mid E_0 ) + q\, \pi_T(E' \mid E_1 )
$$
for $E'$ in the set~$\calA_T$.

\textbf{Remarks.}
The \ENDFdata\ data~\cite{ENDFdata} contains many instances in which
linear-linear direct interpolation is specified, but the \ENDF\ manual~\cite{ENDFB}
says nothing about how to deal with differences in range of outgoing
energies.  Both versions can be expected to produce violation of
energy conservation.  The {\gettransfer} code
currently uses direct interpolation with extrapolation.

\subsection{Unit-base interpolation}\label{Sec:unitBase}
Only the linear-linear version of unit-base interpolation is discussed here.
The first step in unit-base interpolation is the construction of the
range of energies of the outgoing particle.  
The minimum and maximum
outgoing energies for the data in Eq.~(\ref{EPtables})
are given by Eq.~(\ref{Eout-ranges}).
For incident energy~$E$ with $E_0 < E < E_1$, the factor~$q$
is taken as in Eq.~(\ref{def-q}), and
 the minimum and maximum outgoing energies
are given by
\begin{equation}
 \begin{split}
   \Emin'& = ( 1 - q )\Eminzero' + q \Eminone', \\
   \Emax' & = ( 1 - q )\Emaxzero' + q \Emaxone'.
  \label{EoutRange}
 \end{split}
\end{equation}

The interpolated probability density $\pi(E' \mid E)$ must
satisfy the normalization condition
\begin{equation}
  \int_{\Emin'}^{\Emax'} dE' \, \pi(E' \mid E) = 1.
 \label{probabilityNorm}
\end{equation}
One way to ensure this is to first map the outgoing energy ranges
Eq.~(\ref{Eout-ranges}) to unit base $0 \le \Ehat' \le 1$
and to scale the probability densities Eq.~(\ref{EPtables})
accordingly.  Thus, for the data in Eq.~(\ref{EPtables}) at
incident energy~$E_0$, set
\begin{equation}
  \Ehat' = \frac{ E' - \Eminzero'}{ \Emaxzero' - \Eminzero'}
  \label{unit-base-map}
\end{equation}
and scale the probability density
\begin{equation}
  \pihat(\Ehat' \mid E_0) = ( \Emaxzero' - \Eminzero' )\pi(E' \mid E_0).
  \label{unitbaseMap}
\end{equation}
For incident energy $E_1$, the outgoing energy is scaled
as
\begin{equation}
    \Ehat' = \frac{ E' - \Eminone'}{ \Emaxone' - \Eminone'},
  \label{unit-base-map1}
\end{equation}
and the probability density
is scaled to define the unit-base probability
density 
\begin{equation}
  \pihat(\Ehat' \mid E_1) = ( \Emaxone' - \Eminone' )\pi(E' \mid E_1)
  \label{unitbaseMap1}
\end{equation}
for $0 \le \Ehat' \le 1$.

If linear-linear interpolation with respect to incident energy is
desired, the proportionality factor $q$ defined in
Eq.~(\ref{def-q}) is used to linearly interpolate between
$\pihat(\Ehat' \mid E_0)$ and $\pihat(\Ehat' \mid E_1)$
by setting
\begin{equation}
  \pihat(\Ehat' \mid E) = (1 - q)\,\pihat(\Ehat' \mid E_0) +
   q \,\pihat(\Ehat' \mid E_1)
 \label{unitbaseInterp}
\end{equation}
for $0 \le \Ehat' \le 1$. 

Finally, in order to define the interpolated probability
density $\pi(E' \mid E)$, invert the mappings Eq.~(\ref{unit-base-map})
and~Eq.~(\ref{unitbaseMap}).  Specifically, with $\Emin'$ and
$\Emax'$ as in Eq.~(\ref{EoutRange}), set
\begin{equation}
   E' = \Emin' + ( \Emax' - \Emin')\Ehat'
 \label{range-inv}
\end{equation}
and take
\begin{equation}
  \pi(E' \mid E) = \frac{\pihat(\Ehat' \mid E)}{ \Emax' - \Emin' }.
 \label{unitbaseInvert}
\end{equation}

Unit-base interpolation is ordinarily not used with tables of
angular probability densities~$\pi( \mu \mid E)$, because the
range of direction cosines is usually~$-1 \le \mu \le 1$.  One
may want to use it for a table with forward emission given in
the laboratory frame, however.

\subsection{Interpolation by cumulative points}\label{Sec:cumProb}
The method of interpolation by cumulative points that is
used in the code \gettransfer\ is proposed in~\cite{interpolation},
and it is a modification of interpolation by corresponding energies as described
in the \ENDF\ manual~\cite{ENDFB}.  
Interpolation by corresponding energies requires the selection of
$N$ equiprobable energy bins, so the result depends on the value of~$N$.
It is shown in~\cite{interpolation} that for data Eq.~(\ref{EPtable}) which are
histogram with respect to outgoing energy $E'$, interpolation by cumulative 
points is equivalent to interpolation by corresponding energies with~$N = \infty$.
The \gettransfer\ code therefore uses interpolation by cumulative points
whenever the data specify interpolation by corresponding energies.

One objection to unit-base interpolation is that the mapping
(\ref{unitbaseMap}) depends only
on the range of outgoing energies.
One can often get a better approximation to the physics if the
interpolation method incorporates
knowledge of the local behavior of each $\pi( E' \mid E_k)$
in Eq.~(\ref{EPtable}).
One method of doing so is based on the cumulative
probability function
\begin{equation}
  \Pi( E' \mid E_k) = \int_{E'_{k,\text{min}}}^{E'} dx\, \pi( x \mid E_k)
  \label{cumProb}
\end{equation}
for $k = 0$,~1, $\ldots\,$,~$K$.

\textbf{Constraint.}
Interpolation by cumulative points requires that $\Pi( E' \mid E_k)$
be strictly increasing with respect to~$E'$ for $k = 0$,~1, $\ldots\,$,~$K$.  
That is, the 
condition that
\begin{equation}
    \Pi( E'_2 \mid E_k) > \Pi( E'_1 \mid E_k)
  \label{monotone}
\end{equation}
is imposed
for every $E'_1$ and $E'_2$ in the outgoing energy
range 
$$
  E'_{0,k} \le E'_1 < E'_2 \le E'_{J_k,k}.
$$

Because the data $\pi( E' \mid E_k)$ consist of probability densities,
it follows that $\pi( E' \mid E_k) \ge 0$.  If the interpolation with respect to
outgoing energy $E'$ is log-linear or log-log, the monotonicity condition
(\ref{monotone}) implies that $\pi(E'_{j,k} \mid E_k ) > 0$ for all data
points $E'_{j,k}$ in Eq.~(\ref{EPtable}), and for histograms only the
highest outgoing energies $E'_{k,\text{max}}$ may have zero probability density.
For linear-linear and linear-log interpolation, it is permitted that
$\pi(E'_{j,k} \mid E_k ) = 0$ at local values of $E'_{j,k}$ but not for
 two consecutive outgoing energies $E'_{j-1,k}$ and~$E'_{j,k}$.

As with unit-base interpolation, it is sufficient to describe interpolation
by cumulative points between the incident energies $E_0$ and~$E_1$.
For incident energy~$E_0$ compute the cumulative probabilities
at the data points in Eq.~(\ref{EPtables})
\begin{equation}
  y_{j,0} = \Pi( E'_{j,0} \mid E_0) \quad
  \text{for  $j = 0,$ 1, $\ldots\,$, $J_0$}.
 \label{cum-prob0}
\end{equation}
Analogously, for $E = E_1$ determine the cumulative probabilities
at the data points in Eq.~(\ref{EPtables})
\begin{equation}
  y_{j,1} = \Pi( E'_{j,1} \mid E_1) \quad
  \text{for  $j = 0,$ 1, $\ldots\,$, $J_1$}.
  \label{cum-prob1}
\end{equation}
Form the union of the two sets
$$
  \{ Y_\ell \} = \{ y_{j,0} \} \cup \{ y_{j,1} \}.
$$
The $Y_\ell$ values are then ordered with removal of duplicates, so that
\begin{equation}
  Y_0 = 0 < Y_1 < \cdots < Y_{L-1} < Y_L = 1.
 \label{order-cum-prob}
\end{equation}

Interpolation by cumulative points consists of a sequence of unit-base 
interpolations on subintervals.  These subintervals are obtained as
follows.  For incident energy~$E_0$ and each cumulative probability~$Y_\ell$
in Eq.~(\ref{order-cum-prob}), the outgoing 
energies~$\widetilde E'_{\ell,0}$ are computed such that
$$
  \Pi( \widetilde E'_{\ell,0} \mid E_0) = Y_\ell \quad
  \text{for $\ell = 0,$ 1, $\ldots\,$, $L$}.
$$
Note that the construction ensures that each of the original
data points $E'_{j,0}$ in Eq.~(\ref{EPtables}) is one of the
$\widetilde E'_{\ell,0}$ values.  The interval~$\calB_0(\ell)$ is defined as
\begin{equation}
 \begin{split}
  \calB_0(\ell) &= \{ E': \widetilde E'_{\ell-1,0} \le E' < \widetilde E'_{\ell,0} \}
  \quad \text{for $\ell = 1,$ 2, $\ldots\,$, $L-1$}, \\
  \calB_0(L) &= \{ E': \widetilde E'_{L-1,0} \le E' \le \widetilde E'_{L,0} \}
 \end{split}
 \label{def-B0-ell}
\end{equation}
The intervals~$\calB_1(\ell)$ for the data Eq.~(\ref{EPtables}) at
incident energy~$E_1$ and $\ell = 1,$ 2, $\ldots\,$, $L$ are defined
in a similar manner.

Interpolation by cumulative points is accomplished by doing
a sequence of unit-base interpolations between $\pi( E' \mid E_0)$
on the interval~$\calB_0(\ell)$ and $\pi( E' \mid E_1)$ on~$\calB_1(\ell)$
for $\ell = 1,$ 2, $\ldots\,$, $L$.

A more detailed discussion of interpolation of probability data by
the method of cumulative points may be found in the note~\cite{interpolation}.

Because interpolation by cumulative points depends on the
detailed behavior of the probability densities, the method may also be useful
for interpolation of angular probability densities~$\pi( \mu \mid E)$.

\section{Unscaled interpolation of Kalbach-Mann data}\label{Sec:Kalbach-r-interp}
The above discussion pertains to the interpolation of tables of probability densities,
for which maintenance of the norm condition Eq.~(\ref{probabilityNorm})
is essential.  The parameter~$r(\Ecm', E)$ in Eq.~(\ref{Kalbach-eta}) for the
Kalbach-Mann model of double-differential data is given as tables
depending on the energy $E$ of the incident particle and the energy $\Ecm'$ of
the outgoing particle in the center-of-mass frame, and it has the different
constraint,
\begin{equation}
  0 \le r \le 1.
 \label{Kalbach-r-constraint}
\end{equation}

Again, it suffices to describe interpolation between Kalbach-Mann $r$
data between tables at incident energies $E_0$ and~$E_1$ with
$E_0 < E_1$.  As in Eq.~(\ref{def-calA01}),
consider the sets $\calA_0$ of outgoing energies
at $E = E_0$ and $\calA_1$ at $E = E_1$.  For unscaled direct interpolation
with extrapolation, take $\calA_X = \calA_0 \cup \calA_1$ as in
Eq.~(\ref{def-calA-X}), so that the extrapolated $r$ parameter is
\begin{equation*}
  r_X(E', E_0 ) = \begin{cases}
    r(E', E_0 ) &\quad \text{for $E'$ in $\calA_0$,} \\
    0 &\quad \text{for $E'$ in $\calA_X \setminus \calA_0$,}
  \end{cases}
%  \label{def-r-X0}
\end{equation*}
and
\begin{equation*}
  r_X(E', E_1 ) = \begin{cases}
    r(E', E_1 ) &\quad \text{for $E'$ in $\calA_1$,} \\
    0 &\quad \text{for $E'$ in $\calA_X \setminus \calA_1$.}
  \end{cases}
%  \label{def-r-X1}
\end{equation*}
Then for $E_0 < E < E_1$, for $q$ as in Eq.~(\ref{def-q}) and for
$E'$ in the set~$\calA_X$, the linear-linear form of unscaled direct interpolation
with extrapolation becomes as in Eq.~(\ref{def-pi-X}),
\begin{equation}
  r_X(E', E ) = ( 1 - q )\, r_X(E', E_0 ) + q\, r_X(E', E_1 ).
  \label{r-direct-extrapolation}
\end{equation}
The extrapolation version of direct interpolation of the Kallbach-Mann
$r$ parameter as in Eq.~(\ref{r-direct-extrapolation}) is implemented in
the \gettransfer\ code.

For unscaled direct interpolation of the Kalbach-Mann $r$ parameter with
truncation, the outgoing energy $E'$ is restricted to the common 
domain $\calA_T = \calA_0 \cap \calA_1$, and there is no change
of scale analogous to that used for probability densities in
Eq.~(\ref{def-pi-T01}).  Thus, the truncated Kalbach-Mann $r$ parameters
for incident energies $E_0$ and~$E_1$ are
\begin{equation*}
  r_T(E', E_0 ) = \begin{cases}
    r(E', E_0 ) &\quad \text{for $E'$ in $\calA_T$,} \\
    0 &\quad \text{for $E'$ in $\calA_0 \setminus \calA_T$,}
  \end{cases}
%  \label{def-r-T0}
\end{equation*}
and
\begin{equation*}
  r_X(E', E_1 ) = \begin{cases}
    r(E', E_1 ) &\quad \text{for $E'$ in $\calA_T$,} \\
    0 &\quad \text{for $E'$ in $\calA_1 \setminus \calA_T$.}
  \end{cases}
%  \label{def-r-T1}
\end{equation*}
The linear-linear version of unscaled direct
interpolation with truncation is
\begin{equation}
  r_T(E', E ) = ( 1 - q )\, r_T(E', E_0 ) + q\, r_T(E', E_1 )
  \label{r-direct-truncation}
\end{equation}
with $E'$ restricted to~$\calA_T$.
The \gettransfer\ code does not currently implement unscaled direct
interpolation with truncation of the Kalbach-Mann $r$ parameter
given in Eq.~(\ref{r-direct-truncation}).

There is also an unscaled version of unit-base interpolation with
Eqs.~(\ref{unitbaseMap}) and~(\ref{unitbaseMap1}) replaced by
\begin{equation*}
 \begin{split}
  \widehat r(\Ehat', E_0) &= r(E', E_0), \\
  \widehat r(\Ehat', E_1) &= r(E', E_1), \\
 \end{split}
\end{equation*}
for $0 \le \Ehat' \le 1$ with $\Ehat'$ as in Eq.~(\ref{unit-base-map})
for $E = E_0$ and as in Eq.~(\ref{unit-base-map1})
for $E = E_1$.  For linear-linear unscaled unit-base interpolation
to incident energy $E$ with $E_0 < E < E_1$, interpolate the
minimal and maximal outgoing energies as in Eq.~(\ref{EoutRange}),
interpolate $\widehat r$ using
$$
    \widehat r(E', E ) = ( 1 - q )\, \widehat r(E', E_0 ) + q\, \widehat r(E', E_1 ),
$$
and invert the unit-base map using Eq.~(\ref{range-inv})
and 
$$
  r(E', E) = \widehat r(\Ehat', E)
$$
for $\Emin' \le E' \le \Emax'$.

When the energy probability density $\pi_E( E' \mid E )$ in Eq.~(\ref{Kalbach-prob})
is interpolated using the method of cumulative points, the interpolated values of
$r(E', E)$ in Eq.~(\ref{Kalbach-eta}) is obtained using the method of
unscaled cumulative points defined as follows.  The method uses the outgoing energy
ranges $\calB_0( \ell )$ given in Eq.~(\ref{def-B0-ell}) for $\pi_E( E' \mid E )$ at
$E = E_0$ and the corresponding $\calB_1( \ell )$ at $E = E_1$, and it does unscaled
unit-base interpolation of $r( E', E)$ between $\calB_0( \ell )$ and $\calB_1( \ell )$
in sequence for $\ell = 1,$ 2, $\ldots\,$, $L$.


