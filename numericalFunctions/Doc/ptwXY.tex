\documentclass[11pt]{article}
\setlength{\textwidth}{6.5 in}
\setlength{\oddsidemargin}{-0. in}
\setlength{\evensidemargin}{\oddsidemargin}
\usepackage{epic}
\usepackage{eepic}
\usepackage[usenames]{color}

\newlength{\argumentOffset}
\setlength{\argumentOffset}{0 in}

\newlength{\argumentNameWidth}
\newlength{\argumentNameOffset}
\newcommand{\setargumentNameLengths}[1]{
    \settowidth{\argumentNameWidth}{\tt #1: }
    \setlength{\argumentNameOffset}{\linewidth}
    \addtolength{\argumentNameOffset}{-1.\argumentNameWidth}
    \addtolength{\argumentNameOffset}{-0.05\linewidth}
}

\newcommand{\highlight}[1]{{\bf #1}}
\newcommand{\CallingCSub}[3]{\settowidth{\argumentOffset}{\tt #1 }
    \vskip 0.1 in\noindent{\bf C declaration: #3} \\
    \parbox[t]{\linewidth}{
        \hskip 0.05\linewidth \parbox[t]{0.95\linewidth}{\raggedright \sloppy \tt #1 #2}
    }
\vskip 0.1 in}
\newcommand{\CallingC}[2]{\CallingCSub{#1}{#2}{}}
\newcommand{\CallingCLimited}[2]{\CallingCSub{#1}{#2}{ --- This routine is not intended for general use. --- }}
\newcommand{\addArgument}[1]{\hfill \\ \hskip \argumentOffset #1}
\newcommand{\argumentBox}[2]{\noindent\parbox{1.\linewidth}{\parbox[t]
    {0.05\linewidth}{\hfill}\parbox[t]{\argumentNameWidth}{\tt #1:}\parbox[t]{\argumentNameOffset}{#2}}\hfill}

\title{Manual for the numerical functions package\footnote{
This work was performed under the auspices of the U.S. Department of Energy by
Lawrence Livermore National Laboratory under contract \#W-7405-ENG-48.}}
\author{{Bret R. Beck}\\Lawrence Livermore National Laboratory\\UCRL-}

\begin{document}
\setargumentNameLengths{interpolation}

\maketitle

\pagebreak
\tableofcontents
\listoftables
\pagebreak

\section{Introduction}
The \highlight{ptwXY} C object (i.e., an instance of C typedef \highlight{ptwXYPoints}) and supporting C routines are designed to 
handle point-wise interpolative data representing a function of one independent variable
(i.e., y = f(x)). That is, a \highlight{ptwXY} object consist of an 
list of pairs ($x_i,y_i$) where $x_{i} < x_{i+1}$ and $y_i = f(x_i)$. Henceforth, the \highlight{ptwXY} C object
and supporting C routines are called the \highlight{ptwXY} model. Routines supporting common operations 
on the points of f(x) are included in this package. As example, a function exists to add two \highlight{ptwXY} instances returning the sum
as a \highlight{ptwXY} instances (i.e., $h(x) = f(x) + g(x)$ where $f(x)$, $g(x)$ and $h(x)$ 
are all \highlight{ptwXY} objects). The main intent for developing this library is for a fast XY math object for LLNL's FUDGE package
which manipulates nuclear data (e.g., it can be used to add cross section from different reactions). However,
this library may be useful for other packages.

As example of the usage of \highlight{ptwXY} objects consider the data in Table~\ref{B1Pop}. This table list the male and female populations of a
bird species on an island for several census years. In this example, $x_i$ represents
a census year and $y_i$ represents the population for that year. Note that the male population is not
given for the year 1885. A portion of a simple C routine to put the male and female 
populations into \highlight{ptwXY} objects and add them together to get the total population is:

\begin{verbatim}

#include <ptwXY.h>
#define nPairs 7

    double maleData[2 * (nPairs-1)] =   { 1871, 1212, 1883, 1215, 1889,  
        51, 1895, 11, 1905,  9, 1915,  9 };
    double femaleData[2 * nPairs] =     { 1871, 1231, 1883, 1241, 1885, 
        621, 1889, 229, 1895, 31, 1905, 23, 1915, 21 };
    ptwXYPoints *males, *females, *total;
    ptwXY_interpolation linlin = ptwXY_interpolationLinLin;
    nfu_status status;

    males = ptwXY_new( linlin, 5, 1e-3, nPairs, 4, &status, 0 );
    ptwXY_setXYData( males, nPairs - 1, maleData );

    females = ptwXY_new( linlin, 5, 1e-3, nPairs, 4, &status, 0 );
    ptwXY_setXYData( females, nPairs, femaleData );

    total = ptwXY_add_ptwXY( males, females, &status );

    printf( "\nMale population\n" );
    printf( "  Year | Count\n" );
    printf( " ------+------\n" );
    ptwXY_simplePrint( males, " %5.0f | %5.0f\n" );

    printf( "\nFemale population\n" );
    printf( "  Year | Count\n" );
    printf( " ------+------\n" );
    ptwXY_simplePrint( females, " %5.0f | %5.0f\n" );

    printf( "\nTotal population\n" );
    printf( "  Year | Count\n" );
    printf( " ------+------\n" );
    ptwXY_simplePrint( total, " %5.0f | %5.0f\n" );
\end{verbatim}

\begin{table}
\begin{center}
\begin{tabular}{|c|r|r|}  \hline
    year    & Male  & Female \\ \hline \hline
    1871    & 1212  & 1231  \\ \hline
    1883    & 1215  & 1241  \\ \hline
    1885    &  ---  & 621   \\ \hline
    1889    & 51    & 229   \\ \hline
    1895    & 11    & 31    \\ \hline
    1905    & 9     & 23    \\ \hline
    1915    & 9     & 21    \\ \hline
\end{tabular}
\end{center}
\caption{Males and females population of a bird species on an island for the years census were taken. There was
no census taken of the male population in 1885.\label{B1Pop}}
\end{table}

\noindent
The output of this code - compressed into fewer lines - is:

\begin{verbatim}
 Output from first  | Output from second  | Output from third
 ptwXY_simplePrint | ptwXY_simplePrint  | ptwXY_simplePrint
--------------------+---------------------+-------------------
  Male population   |  Female population  |  Total population
    Year | Count    |    Year | Count     |    Year | Count
   ------+------    |   ------+------     |   ------+------
    1871 |  1212    |    1871 |  1231     |    1871 |  2443
    1883 |  1215    |    1883 |  1241     |    1883 |  2456
    1889 |    51    |    1885 |   621     |    1885 |  1448
    1895 |    11    |    1889 |   229     |    1889 |   280
    1905 |     9    |    1895 |    31     |    1895 |    42
    1915 |     9    |    1905 |    23     |    1905 |    32
                    |    1915 |    21     |    1915 |    30
\end{verbatim}

\noindent
In this example no error checking is shown. The routine \highlight{ptwXY\_new} allocates memory for
a new \highlight{ptwXYPoints} object, initializes it and returns a pointer to the object. The first 
argument of this routine in an interpolation flag. For all other arguments see Section~\ref{ptwXYnewSec} The
routine \highlight{ptwXY\_setXYData} takes a pointer to a \highlight{ptwXYPoints} object as its first argument and copies 
the list of doubles given by the third argument into the \highlight{ptwXYPoints} object's internal memory, deleting
any data currently in the object. 
The second argument is the number of pairs of points
in the third argument's data. 

The routine \highlight{ptwXY\_add\_ptwXY} takes a \highlight{ptwXYPoints} object
as its first and second arguments and returns a new \highlight{ptwXYPoints} object that is the sum. 
The summed object's $x$ values are a union between the $x$ value's of the operants.  As can be seen
from the example, this routine interpolates to fill in missing data for either data set. That is, the male population
was linear-linear interpolated to give 827 for the year 1885.

\subsection{Important concepts}
This section describes several important concepts and rules that the \highlight{ptwXY} model is build on.

\subsubsection{accuracy}
\subsubsection{Mutual domains}
Most routines that have two or more \highlight{ptwXYPoints} instances as input (e.g., ptwXY\_add\_ptwXY, ptwXY\_groupThreeFunctions) require
that their domains be mutual. This section explains why mutual domains are needed and what a mutual domain is.

Consider the two point-wise linear-linear interpolable functions
\begin{verbatim}
        f1 = (1,1), (9,3)
        f2 = (1,2), (9,4).
\end{verbatim}
where a point-wise function with $n$ points is written as
\begin{eqnarray}
        (x_0,y_0), \ (x_1,y_1), \ (x_2,y_2), \ ... \ (x_{n-1},y_{n-1}) \nonumber
\end{eqnarray}
and each point is the pair $(x_i,y_i)$ with $x_i < x_{i+1}$.
Each of these functions contains only two points. The first has domain $1 \le x \le 9$ with the y-value 
going from 1 to 3 and can be represented symbolically as $y = f1(x) = ( x - 1 ) / 4 + 1$ for the domain $1 \le x \le 9$. 
The second has the same domain with the y-value going from 2 to 4 and can be represented symbolically as $y = f2(x) = ( x - 1 ) / 4 + 2$. 
The rule that should be implemented for adding these two functions is clear and is
$s(x) = ( x - 1 ) / 2 + 3 = f1(x) + f2(x)$ or in point-wise form
\begin{verbatim}
        (1,3), (9,7) = f1 + f2
\end{verbatim}
For the domain $1 \le x \le 9$, the point-wise linear-linear interpolable sum and the symbolic sum yield the
same results. For example, both yield $s(3) = 4$. Figure~\ref{sum_f1_f2} graphically shows $f1$, $f2$ and $f1 + f2$.
\begin{figure}
\setlength{\unitlength}{1 cm}
\begin{center}
\begin{picture}(10, 8)(0, 0)
    \color{Black}
% yAxis
    \drawline(0, 0)(0, 8) \drawline(10, 0)(10, 8)
    \put( -0.3,  0 ){0}
    \put( -0.3,  0.9 ){1} \drawline(0, 1)(.2, 1)
    \put( -0.3,  1.9 ){2} \drawline(0, 2)(.2, 2)
    \put( -0.3,  2.9 ){3} \drawline(0, 3)(.2, 3)
    \put( -0.3,  3.9 ){4} \drawline(0, 4)(.2, 4)
    \put( -0.3,  4.9 ){5} \drawline(0, 5)(.2, 5)
    \put( -0.3,  5.9 ){6} \drawline(0, 6)(.2, 6)
    \put( -0.3,  6.9 ){7} \drawline(0, 7)(.2, 7)
    \put( -0.3,  7.9 ){8} \drawline(0, 8)(.2, 8)

% xAxis
    \drawline(0, 0)(10, 0)  \drawline(0, 8)(10, 8)
    \put( -0.1,  -0.4 ){0}
    \put(  0.9,  -0.4 ){1} \put(1, 0){\line( 0, 1 ){.2}}
    \put(  1.9,  -0.4 ){2} \put(2, 0){\line( 0, 1 ){.2}}
    \put(  2.9,  -0.4 ){3} \put(3, 0){\line( 0, 1 ){.2}}
    \put(  3.9,  -0.4 ){4} \put(4, 0){\line( 0, 1 ){.2}}
    \put(  4.9,  -0.4 ){5} \put(5, 0){\line( 0, 1 ){.2}}
    \put(  5.9,  -0.4 ){6} \put(6, 0){\line( 0, 1 ){.2}}
    \put(  6.9,  -0.4 ){7} \put(7, 0){\line( 0, 1 ){.2}}
    \put(  7.9,  -0.4 ){8} \put(8, 0){\line( 0, 1 ){.2}}
    \put(  8.9,  -0.4 ){9} \put(9, 0){\line( 0, 1 ){.2}}
    \put(  9.9,  -0.4 ){10} \put(10, 0){\line( 0, 1 ){.2}}

% lines
    \color{Red} \drawline(1, 1)(9, 3)
    \color{Green} \drawline(1, 2)(9, 4)
    \color{Blue} \drawline(1, 3)(9, 7)

\end{picture}
\caption{The red curve is f1, the green curve is f2 and the blue curve is $\rm{f1} + \rm{f2}$.
    \label{sum_f1_f2}}
\end{center}
\end{figure}


Now consider the point-wise linear-linear interpolable function
\begin{verbatim}
        f3 = (3,1), (7,3).
\end{verbatim}
This function also contains only two points and has domain $3 \le x \le 7$ with a y-value going of 1 to 3. This function
can be represented symbolically as $y = f3(x) = ( x - 3 ) / 2 + 1$. The rule that should be implemented for 
adding $f1$ and $f3$ is not obvious. For example, one could implement the rule which makes a union of the x-values 
in $f1$ and $f3$ (i.e., 1, 3, 7 and 9),
interpolate each function onto these points using 0 where the function is not defined and then add the y-values.
Let $f1'$ and $f3'$ be the functions $f1$ and $f3$ with the union points and the y-values filled in. In
point-wise representation, $f1'$ and $f3'$ are
\begin{verbatim}
        f1' = (1,1), (3,1.5), (7,2.5), (9,3)
        f3' = (1,0), (3,1),   (7,3),   (9,0).
\end{verbatim}
The sum resulting from this rule is then
\begin{verbatim}
        (1,1), (3,2.5), (7,5.5), (9,3) = f1' + f3' = f1 + f3.
\end{verbatim}
and is shown as the blue curve in Fgure~\ref{sum_f1_f3}. The blue curve is not vary appealing in part because for this addition rule the
sum of $f3$ with $f1$ makes the assumption that $f3(x) = ( x - 1 ) / 2$ for $1 \le x \le 3$.
But what is worse, this rule does not guarantee the associativity rule for addition. To see this, consider the three 
linear-linear point-wise functions
\begin{verbatim}
        g1 = (1,0), (1,1),        (10,10)
        g2 = (1,0),               (10,10)
        g3 =               (2,1), (10,1).
\end{verbatim}
Note that $g1$ and $g2$ represent the same function.
The addition $(g1 + g2) + g3$ is
\begin{verbatim}
        (0,0), (1,2), (2,5), (10,21) = ( g1 + g2 ) + g3
\end{verbatim}
while the addition $g1 + (g2 + g3)$ is
\begin{verbatim}
        (0,0), (1,2.5), (2,5), (10,21) = g1 + ( g2 + g3 ).
\end{verbatim}
\begin{figure}
\setlength{\unitlength}{1 cm}
\begin{center}
\begin{picture}(10, 6)(0, 0)
    \color{black}
% yAxis
    \drawline(0, 0)(0, 6) \drawline(10, 0)(10, 6)
    \put( -0.3,  0 ){0}
    \put( -0.3,  0.9 ){1} \drawline(0, 1)(.2, 1)
    \put( -0.3,  1.9 ){2} \drawline(0, 2)(.2, 2)
    \put( -0.3,  2.9 ){3} \drawline(0, 3)(.2, 3)
    \put( -0.3,  3.9 ){4} \drawline(0, 4)(.2, 4)
    \put( -0.3,  4.9 ){5} \drawline(0, 5)(.2, 5)
    \put( -0.3,  5.9 ){6} \drawline(0, 6)(.2, 6)

% xAxis
    \drawline(0, 0)(10, 0)  \drawline(0, 6)(10, 6)
    \put( -0.1,  -0.4 ){0}
    \put(  0.9,  -0.4 ){1} \put(1, 0){\line( 0, 1 ){.2}}
    \put(  1.9,  -0.4 ){2} \put(2, 0){\line( 0, 1 ){.2}}
    \put(  2.9,  -0.4 ){3} \put(3, 0){\line( 0, 1 ){.2}}
    \put(  3.9,  -0.4 ){4} \put(4, 0){\line( 0, 1 ){.2}}
    \put(  4.9,  -0.4 ){5} \put(5, 0){\line( 0, 1 ){.2}}
    \put(  5.9,  -0.4 ){6} \put(6, 0){\line( 0, 1 ){.2}}
    \put(  6.9,  -0.4 ){7} \put(7, 0){\line( 0, 1 ){.2}}
    \put(  7.9,  -0.4 ){8} \put(8, 0){\line( 0, 1 ){.2}}
    \put(  8.9,  -0.4 ){9} \put(9, 0){\line( 0, 1 ){.2}}
    \put(  9.9,  -0.4 ){10} \put(10, 0){\line( 0, 1 ){.2}}

% lines
    \color{red} \drawline(1, 1)(9, 3)
    \color{green} \drawline(3, 1)(7, 3)
    \color{blue} \drawline(1, 2)(3, 3.5) \drawline(3, 3.5)(7, 5.5)  \drawline(7, 5.5)(9, 3)

    \color{black} \drawline(1,1)(2.99999,1.4999975) \drawline(2.99999,1.4999975)(3,2.5) \drawline(3,2.5)(7,5.5)
                  \drawline(7,5.5)(7.00001, 2.5000025) \drawline(7.00001, 2.5000025)(9,3)
\end{picture}
\caption{The red curve is f1, the green curve is f2 and the blue curve is $\rm{f1} + \rm{f3}$ using the first rule and
    the black curve is $\rm{f1} + \rm{f3}$ using the second rule.
    \label{sum_f1_f3}}
\end{center}
\end{figure}


Another rule one could implement would effectively add a point with $y = 0$ just below the first point of $f3$,
one just above the last point of $f3$, one at $x = 1$ and one at $x = 9$ to yield an $f3'$ as
\begin{verbatim}
        f3' = (1,0), (2.99999,0), (3,1), (7,3), (7.00001,0), (9,0).
\end{verbatim}
and the sum $f1 + f3$ would then be
\begin{verbatim}
        (1,1), (2.99999,1.4999975, (3,2.5), (7,5.5), (7.00001, 2.5000025), (9,3)
\end{verbatim}
The sum resulting from this latter rule is shown as the black curve in Fgure~\ref{sum_f1_f3}. This rule looks much better and is.
However, when designing the \highlight{ptwXYPoints} model, this rule was also rejected as it would require the 
\highlight{ptwXYPoints} library to know the appropriate distance below and above the end-points to add 0's.

The rule that the \highlight{ptwXYPoints} model implements is called ``mutual domain''. This rule states that the
domains of the functions operated on must be the same with one exception. This exception will now be explained. 
Let $h1$ and $h2$ be two \highlight{ptwXYPoints} instances with
the lower and upper domain limits of $h1$ being $x_{1,l}$ and $x_{1,u}$ and that of $h2$ being $x_{2,l}$ and $x_{2,u}$. 
If $x_{1,l} \ne x_{2,l}$ then the y-value for the greater lower-x-limit must be 0. For example, if
$x_{1,l} > x_{2,l}$ then $h2(x_{2,l}) = 0$. If $x_{1,u} \ne x_{2,u}$ then the y-value for the lesser
upper-x-limit must be 0. For example, if $x_{1,u} < x_{2,u}$ then $h1(x_{1,u}) = 0$. This rule works because
the \highlight{ptwXYPoints} model assumes that if the y-value is 0 at the lower limit, then it is 0 for all
$x$ less than the lower limit. Likewise if the y-value is 0 at the upper limit, then it is 0 for all
$x$ greater than the upper limit.

If $f4$ and $f5$ have mutual domains, and $f4$ and $f6$ have mutual domains, the it is not guaranteed that $f5$ and $f6$
have mutual domains. As an example, let
\begin{verbatim}
        f4 = (1,4), (8,4)
        f5 = (3,0), (8,2)
        f6 = (4,3), (6,0).
\end{verbatim}
Then, $f4$ and $f5$ have mutual domains and so do $f5$ and $f6$. However, $f4$ and $f6$ do not have mutual domains.
Because of this fact, the function \highlight{ptwXY\_groupThreeFunctions} has to check the domains of \highlight{ptwXY1} to 
\highlight{ptwXY2}, \highlight{ptwXY1} to \highlight{ptwXY3} and \highlight{ptwXY2} to \highlight{ptwXY3}. Acutally,
\highlight{ptwXY\_groupThreeFunctions} first limits the domains of \highlight{ptwXY1}, \highlight{ptwXY2} and
\highlight{ptwXY3} to that of \highlight{groupBoundaries} first.
\begin{figure}
\setlength{\unitlength}{1 cm}
\begin{center}
\begin{picture}(10, 5)(0, 0)
    \color{Black}
% yAxis
    \drawline(0, 0)(0, 5) \drawline(10, 0)(10, 5)
    \put( -0.3,  0 ){0}
    \put( -0.3,  0.9 ){1} \drawline(0, 1)(.2, 1)
    \put( -0.3,  1.9 ){2} \drawline(0, 2)(.2, 2)
    \put( -0.3,  2.9 ){3} \drawline(0, 3)(.2, 3)
    \put( -0.3,  3.9 ){4} \drawline(0, 4)(.2, 4)
    \put( -0.3,  4.9 ){5} \drawline(0, 5)(.2, 5)

% xAxis
    \drawline(0, 0)(10, 0)  \drawline(0, 5)(10, 5)
    \put( -0.1,  -0.4 ){0}
    \put(  0.9,  -0.4 ){1} \put(1, 0){\line( 0, 1 ){.2}}
    \put(  1.9,  -0.4 ){2} \put(2, 0){\line( 0, 1 ){.2}}
    \put(  2.9,  -0.4 ){3} \put(3, 0){\line( 0, 1 ){.2}}
    \put(  3.9,  -0.4 ){4} \put(4, 0){\line( 0, 1 ){.2}}
    \put(  4.9,  -0.4 ){5} \put(5, 0){\line( 0, 1 ){.2}}
    \put(  5.9,  -0.4 ){6} \put(6, 0){\line( 0, 1 ){.2}}
    \put(  6.9,  -0.4 ){7} \put(7, 0){\line( 0, 1 ){.2}}
    \put(  7.9,  -0.4 ){8} \put(8, 0){\line( 0, 1 ){.2}}
    \put(  8.9,  -0.4 ){9} \put(9, 0){\line( 0, 1 ){.2}}
    \put(  9.9,  -0.4 ){10} \put(10, 0){\line( 0, 1 ){.2}}

% lines
    \color{Red} \drawline(1, 4)(8, 4)
    \color{Green} \drawline(3, 0)(8, 2)
    \color{Blue} \drawline(3, 3)(6, 0)

\end{picture}
\caption{The red curve is $f4$, the green curve is $f5$ and the blue curve is $f6$. 
    The domains of $f4$ and $f5$ are mutual as are the domains of $f5$ and $f6$. However,
    the domains of $f4$ and $f6$ are not mutual.
    \label{mutualDomain_f4_f5_f6}}
\end{center}
\end{figure}


\subsubsection{Infill}
The addition of two linear functions yields another linear function. As example, the sum of $f1(x) = s1 \times x + y1$
and $f2(x) = s2 \times x + y2$ is $f1(x) + f2(x) = ( s1 + s2 ) \times x + y1 + y2$. Hence, when the function \highlight{ptwXY\_add\_ptwXY}
adds two linear-linear pointwise functions, it only needs to make a union of the x-values of the two addends to maintain accuracy.
However, the multiplication of $f1(x)$ and $f2(x)$ is not a linear function but a quadratic function (e.g., $ f1(x) \times f2(x) =
s1 \times s2 \times x^2 + ( s1 \times y2 + s2 \times y1 ) \times x + y1 \times y2$). In an attempt to maintain accuracy, the
function \highlight{ptwXY\_mul2\_ptwXY} may add additional points between the union points. For example, consider the following
linear-linear point-wise functions $f3$ and $f4$,
\begin{verbatim}
        f3 = (0,0), (1,1)
        f4 = (0,1), (1,0)
\end{verbatim}
which have the symbolic forms $f_3(x) = x$ and $f_4(x) = 1 - x$ over the domain $0 \le x \le 1$ and the symbolic
product $x ( 1 - x )$. Making a union of the x-values and evaluating the product on the x-values yields
\begin{verbatim}
    (0,0), (1,0) = f3 * f4
\end{verbatim}
which is clearly inadequate. For this example, the only way to maintain the accuracy is to add points between $x = 0$ and $x = 1$. The adding of
points in an attempt to maintain accuracy is called infilling and is done automatically by some \highlight{ptwXYPoints} functions
including \highlight{ptwXY\_mul2\_ptwXY} but not by \highlight{ptwXY\_mul\_ptwXY}.

Infilling is done by bisecting (i.e., generating the point midway between) two consecutive points and asking if 
the accuracy of the operation (e.g., mutiplication) is satisfied. If it is, the midpoint is not added. However, if 
the accurcay is not satisfied, the midpoint is added then the segments on both side of the midpoint are tested. 

In some cases, infilling can add a lot of points, more than one may like. Each \highlight{ptwXYPoints}
instance has a member called \highlight{biSectionMax} to limit bisecting. The union function sets the
\highlight{biSectionMax} of the returned \highlight{ptwXYPoints} instance, to the maximum of \highlight{biSectionMax} of
its inputted \highlight{ptwXYPoints} instances. For each initial segment of the union at most
\highlight{biSectionMax} bisections are performed. Table~\ref{infillCode1} contains a snippet of code which
demonstrates the multiplication $f3$ and $f4$, without any error checking of course, for \highlight{biSectionMax}
set to 0, 1, 2, and 3, and Figure~\ref{mul_f3_f4_infill} show the output from this code.

\begin{table}
\begin{verbatim}
int main( int argc, char **argc ) {

    double f3[4] = { 0., 0., 1., 1. }, f4[4] = { 0., 1., 1., 0. };
    double accuracy = 1e-3;
    ptwXYPoints *ptwXY3, *ptwXY4;
    nfu_status status;
    ptwXY_interpolation linlin = ptwXY_interpolationLinLin;

    ptwXY3 = ptwXY_create( linlin, 0, accuracy, 10, 10, 2, f3, &status, 0 );
    ptwXY4 = ptwXY_create( linlin, 0, accuracy, 10, 10, 2, f4, &status, 0 );

    doProduct( ptwXY3, ptwXY4, 0 );
    doProduct( ptwXY3, ptwXY4, 1 );
    doProduct( ptwXY3, ptwXY4, 2 );
    doProduct( ptwXY3, ptwXY4, 3 );

}

void doProduct( ptwXYPoints *ptwXY3, ptwXYPoints *ptwXY4, double biSection ) {

    ptwXYPoints *product;
    nfu_status status;

    ptwXY_setBiSectionMax( ptwXY3, biSection );
    product = ptwXY_mul2_ptwXY( ptwXY3, ptwXY4, &status );
}
\end{verbatim}
\caption{This table show a snippet of the code used to generate the curves in Figure~\ref{mul_f3_f4_infill}} \label{infillCode1}
\end{table}
\begin{figure}
\setlength{\unitlength}{10 cm}
\begin{center}
\begin{picture}(1, 0.9)(0, 0)
    \color{Black}
% yAxis
    \drawline(0, 0)(0, 0.9) \drawline(1.0, 0)(1.0, 0.9)
    \put( -0.07,  0 ){0}
    \put( -0.07,  0.27){0.1} \drawline(0, 0.3)(.02, 0.3)
    \put( -0.07,  0.57 ){0.2} \drawline(0, 0.6)(.02, 0.6)
    \put( -0.07,  0.87 ){0.3} \drawline(0, 0.9)(.02, 0.9)

% xAxis
    \drawline(0, 0)(1.0, 0)  \drawline(0, 0.9)(1.0, 0.9)
    \put( -0.01,  -0.04 ){0}
    \put(  0.09,  -0.04 ){0.1} \put(0.1, 0){\line( 0, 1 ){.02}}
    \put(  0.19,  -0.04 ){0.2} \put(0.2, 0){\line( 0, 1 ){.02}}
    \put(  0.29,  -0.04 ){0.3} \put(0.3, 0){\line( 0, 1 ){.02}}
    \put(  0.39,  -0.04 ){0.4} \put(0.4, 0){\line( 0, 1 ){.02}}
    \put(  0.49,  -0.04 ){0.5} \put(0.5, 0){\line( 0, 1 ){.02}}
    \put(  0.59,  -0.04 ){0.6} \put(0.6, 0){\line( 0, 1 ){.02}}
    \put(  0.69,  -0.04 ){0.7} \put(0.7, 0){\line( 0, 1 ){.02}}
    \put(  0.79,  -0.04 ){0.8} \put(0.8, 0){\line( 0, 1 ){.02}}
    \put(  0.89,  -0.04 ){0.9} \put(0.9, 0){\line( 0, 1 ){.02}}
    \put(  0.99,  -0.04 ){1.0} \put(1.0, 0){\line( 0, 1 ){.02}}

% lines
    \color{Black}

        \drawline(0.0, 0.0)(0.01, 0.0297)
        \drawline(0.01, 0.0297)(0.02, 0.0588)
        \drawline(0.02, 0.0588)(0.03, 0.0873)
        \drawline(0.03, 0.0873)(0.04, 0.1152)
        \drawline(0.04, 0.1152)(0.05, 0.1425)
        \drawline(0.05, 0.1425)(0.06, 0.1692)
        \drawline(0.06, 0.1692)(0.07, 0.1953)
        \drawline(0.07, 0.1953)(0.08, 0.2208)
        \drawline(0.08, 0.2208)(0.09, 0.2457)
        \drawline(0.09, 0.2457)(0.1, 0.27)
        \drawline(0.1, 0.27)(0.11, 0.2937)
        \drawline(0.11, 0.2937)(0.12, 0.3168)
        \drawline(0.12, 0.3168)(0.13, 0.3393)
        \drawline(0.13, 0.3393)(0.14, 0.3612)
        \drawline(0.14, 0.3612)(0.15, 0.3825)
        \drawline(0.15, 0.3825)(0.16, 0.4032)
        \drawline(0.16, 0.4032)(0.17, 0.4233)
        \drawline(0.17, 0.4233)(0.18, 0.4428)
        \drawline(0.18, 0.4428)(0.19, 0.4617)
        \drawline(0.19, 0.4617)(0.2, 0.48)
        \drawline(0.2, 0.48)(0.21, 0.4977)
        \drawline(0.21, 0.4977)(0.22, 0.5148)
        \drawline(0.22, 0.5148)(0.23, 0.5313)
        \drawline(0.23, 0.5313)(0.24, 0.5472)
        \drawline(0.24, 0.5472)(0.25, 0.5625)
        \drawline(0.25, 0.5625)(0.26, 0.5772)
        \drawline(0.26, 0.5772)(0.27, 0.5913)
        \drawline(0.27, 0.5913)(0.28, 0.6048)
        \drawline(0.28, 0.6048)(0.29, 0.6177)
        \drawline(0.29, 0.6177)(0.3, 0.63)
        \drawline(0.3, 0.63)(0.31, 0.6417)
        \drawline(0.31, 0.6417)(0.32, 0.6528)
        \drawline(0.32, 0.6528)(0.33, 0.6633)
        \drawline(0.33, 0.6633)(0.34, 0.6732)
        \drawline(0.34, 0.6732)(0.35, 0.6825)
        \drawline(0.35, 0.6825)(0.36, 0.6912)
        \drawline(0.36, 0.6912)(0.37, 0.6993)
        \drawline(0.37, 0.6993)(0.38, 0.7068)
        \drawline(0.38, 0.7068)(0.39, 0.7137)
        \drawline(0.39, 0.7137)(0.4, 0.72)
        \drawline(0.4, 0.72)(0.41, 0.7257)
        \drawline(0.41, 0.7257)(0.42, 0.7308)
        \drawline(0.42, 0.7308)(0.43, 0.7353)
        \drawline(0.43, 0.7353)(0.44, 0.7392)
        \drawline(0.44, 0.7392)(0.45, 0.7425)
        \drawline(0.45, 0.7425)(0.46, 0.7452)
        \drawline(0.46, 0.7452)(0.47, 0.7473)
        \drawline(0.47, 0.7473)(0.48, 0.7488)
        \drawline(0.48, 0.7488)(0.49, 0.7497)
        \drawline(0.49, 0.7497)(0.5, 0.75)
        \drawline(0.5, 0.75)(0.51, 0.7497)
        \drawline(0.51, 0.7497)(0.52, 0.7488)
        \drawline(0.52, 0.7488)(0.53, 0.7473)
        \drawline(0.53, 0.7473)(0.54, 0.7452)
        \drawline(0.54, 0.7452)(0.55, 0.7425)
        \drawline(0.55, 0.7425)(0.56, 0.7392)
        \drawline(0.56, 0.7392)(0.57, 0.7353)
        \drawline(0.57, 0.7353)(0.58, 0.7308)
        \drawline(0.58, 0.7308)(0.59, 0.7257)
        \drawline(0.59, 0.7257)(0.6, 0.72)
        \drawline(0.6, 0.72)(0.61, 0.7137)
        \drawline(0.61, 0.7137)(0.62, 0.7068)
        \drawline(0.62, 0.7068)(0.63, 0.6993)
        \drawline(0.63, 0.6993)(0.64, 0.6912)
        \drawline(0.64, 0.6912)(0.65, 0.6825)
        \drawline(0.65, 0.6825)(0.66, 0.6732)
        \drawline(0.66, 0.6732)(0.67, 0.6633)
        \drawline(0.67, 0.6633)(0.68, 0.6528)
        \drawline(0.68, 0.6528)(0.69, 0.6417)
        \drawline(0.69, 0.6417)(0.7, 0.63)
        \drawline(0.7, 0.63)(0.71, 0.6177)
        \drawline(0.71, 0.6177)(0.72, 0.6048)
        \drawline(0.72, 0.6048)(0.73, 0.5913)
        \drawline(0.73, 0.5913)(0.74, 0.5772)
        \drawline(0.74, 0.5772)(0.75, 0.5625)
        \drawline(0.75, 0.5625)(0.76, 0.5472)
        \drawline(0.76, 0.5472)(0.77, 0.5313)
        \drawline(0.77, 0.5313)(0.78, 0.5148)
        \drawline(0.78, 0.5148)(0.79, 0.4977)
        \drawline(0.79, 0.4977)(0.8, 0.48)
        \drawline(0.8, 0.48)(0.81, 0.4617)
        \drawline(0.81, 0.4617)(0.82, 0.4428)
        \drawline(0.82, 0.4428)(0.83, 0.4233)
        \drawline(0.83, 0.4233)(0.84, 0.4032)
        \drawline(0.84, 0.4032)(0.85, 0.3825)
        \drawline(0.85, 0.3825)(0.86, 0.3612)
        \drawline(0.86, 0.3612)(0.87, 0.3393)
        \drawline(0.87, 0.3393)(0.88, 0.3168)
        \drawline(0.88, 0.3168)(0.89, 0.2937)
        \drawline(0.89, 0.2937)(0.9, 0.27)
        \drawline(0.9, 0.27)(0.91, 0.2457)
        \drawline(0.91, 0.2457)(0.92, 0.2208)
        \drawline(0.92, 0.2208)(0.93, 0.1953)
        \drawline(0.93, 0.1953)(0.94, 0.1692)
        \drawline(0.94, 0.1692)(0.95, 0.1425)
        \drawline(0.95, 0.1425)(0.96, 0.1152)
        \drawline(0.96, 0.1152)(0.97, 0.0873)
        \drawline(0.97, 0.0873)(0.98, 0.0588)
        \drawline(0.98, 0.0588)(0.99, 0.0297)
        \drawline(0.99, 0.0297)(1.0, 0.0)

    \color{Blue}
        \drawline(0.0, 0.0)(0.5, 0.75)
        \drawline(0.5, 0.75)(1.0, 0.0)

    \color{Green}
        \drawline(0.0, 0.0)(0.25, 0.5625)
        \drawline(0.25, 0.5625)(0.5, 0.75)
        \drawline(0.5, 0.75)(0.75, 0.5625)
        \drawline(0.75, 0.5625)(1.0, 0.0)

    \color{Red}
        \drawline(0.0, 0.0)(0.125, 0.328125)
        \drawline(0.125, 0.328125)(0.25, 0.5625)
        \drawline(0.25, 0.5625)(0.366025403784, 0.696152422707)
        \drawline(0.366025403784, 0.696152422707)(0.5, 0.75)
        \drawline(0.5, 0.75)(0.633974596216, 0.696152422707)
        \drawline(0.633974596216, 0.696152422707)(0.75, 0.5625)
        \drawline(0.75, 0.5625)(0.875, 0.328125)
        \drawline(0.875, 0.328125)(1.0, 0.0)

% biSectionMax = 4
%        \drawline(0.0, 0.0)(0.0625, 0.17578125)
%        \drawline(0.0625, 0.17578125)(0.125, 0.328125)
%        \drawline(0.125, 0.328125)(0.179128784748, 0.441124989669)
%        \drawline(0.179128784748, 0.441124989669)(0.25, 0.5625)
%        \drawline(0.25, 0.5625)(0.304923875032, 0.635835916401)
%        \drawline(0.304923875032, 0.635835916401)(0.366025403784, 0.696152422707)
%        \drawline(0.366025403784, 0.696152422707)(0.431765131169, 0.736032008028)
%        \drawline(0.431765131169, 0.736032008028)(0.5, 0.75)
%        \drawline(0.5, 0.75)(0.568234868831, 0.736032008028)
%        \drawline(0.568234868831, 0.736032008028)(0.633974596216, 0.696152422707)
%        \drawline(0.633974596216, 0.696152422707)(0.695076124968, 0.635835916401)
%        \drawline(0.695076124968, 0.635835916401)(0.75, 0.5625)
%        \drawline(0.75, 0.5625)(0.820871215252, 0.441124989669)
%        \drawline(0.820871215252, 0.441124989669)(0.875, 0.328125)
%        \drawline(0.875, 0.328125)(0.9375, 0.17578125)
%        \drawline(0.9375, 0.17578125)(1.0, 0.0)


    \color{Black}
        \linethickness{0.6mm}
        \multiput(0, 0)(.05, 0){20}{\line(1, 0){.025}}

\end{picture}
\caption{The solid black curve is the function $x ( 1 - x )$. The blue, grean and red curves are products of
$f3 \times f4$ from ptwXY\_mul2\_ptwXY for biSectionMax of 1, 2 and 3 respectively. The solid black curve and the
red curve are nearly identical. The dashed thicker black line at the bottom of the plot is the product for
biSectionMax = 0.
    \label{mul_f3_f4_infill}}
\end{center}
\end{figure}





If infilling is needed, the \highlight{biSectionMax} member of the returned \highlight{ptwXYPoints} instance is reduced by 
$\ln(l_f / l_u) / ln(2)$ where $l_u$ is the lenght of the union and $l_f$ is the final length after all bisections\footnote{This reduction is derived
by setting $2^z = l_f/l_u$ and solving for $z$.}. An \highlight{ptwXYPoints} instance's \highlight{biSectionMax} can be set using
\highlight{ptwXY\_setBiSectionMax} and got using \highlight{ptwXY\_setBiSectionMax}. A \highlight{ptwXYPoints}'s
\highlight{biSectionMax} is limited to the range 0 to \highlight{ptwXY\_maxBiSectionMax}.

\subsubsection{Safe divide}

\section{Name convention}

This section defines some of the names used in this document.

\begin{description}
    \item[point:] A point is a pair of $(x,y)$ values.
    \item[cache and array:] In this document there is a distinction between a cache and an array of a cache. A cache 
            is allocated memory used to store data. An array of a cache is a region of a cache containing valid data. As example, 
            for the primary cache described in section~\ref{TCDGArray}, points are added to the cache as needed. The current points 
            in the cache constitute the array of that cache.
\end{description}

\section{Two-cached, Dynamic-Growth Data Array} \label{TCDGArray}
Built into the \highlight{ptwXY} model is the ability to insert a point at any $x$. The supporting routines will
automatically increase the size of an internal data cache if needed to accommodate a new $x$ value. However, to make adding
and deleting points potentially more efficient, the \highlight{ptwXY} model has two data caches, dubbed primary and secondary.
In the primary cache, data are stored in a C array in ascending order which allows for quick accessing. However, inserting a new
$x$ value at any place other than the end of the array can be slow as it requires moving all $x$ values that are greater than the new value
up one element in the array.  To over come this, a newly added $x$ value
is inserted into the secondary cache if: 1) the value cannot be inserted after the last element of the primary array\footnote{A value can
only be inserted after the last element of the primary array if its $x$ is greater than the current maximum $x$ value and there is room in
the primary cache.} and 2) space is available in the secondary cache. The secondary cache is a static, linked list. Here, static means that
the elements of the linked list are allocated during setup so there is no overhead associated with allocating or freeing
elements of the linked list later. Typically, re-allocation
of the memory of the primary cache is only required when a new $x$ value cannot be inserted into either cache.

There are four parameters, two for each cache, that describe the current state of the caches. Each cache has a size which is the 
amount, in units of an element of that cache, of memory allocated for the cache and a length which is the amount, in units of an 
element of that cache, of the cache that is currently used (i.e., the size of the array of that cache).

The initial size of the two caches is set either through the routine \highlight{ptwXY\_new} or \highlight{ptwXY\_\-setup} via their
\highlight{primarySize} and \highlight{secondarySize} arguments. The size of the primary and secondary caches can be directly altered after
they have been created via the
routines \highlight{ptwXY\_reallo\-cate\-Points} and \highlight{ptwXY\_\-reallo\-cate\-Overflow\-Points} respectively. 
In general these last two routines should
not be called by the users unless they know that the a cache is woefully too small.

The routine \highlight{pwtXY\_coalescePoints} can be called to transfer all secondary points into the primary cache.

\section{ptwXYPoints's C structs, macros and enums}
The following definitions are defined in the C header file "ptwXY.h".

\subsection{ptwXYPoints}
The \highlight{ptwXYPoints} type is defined as:
\begin{verbatim}
typedef
    struct ptwXYPoints_s {
        nfu_status status;
        ptwXY_interpolation interpolation;
        int userFlag;
        double biSectionMax;
        double accuracy;
        double minFractional_dx;
        int64_t length;
        int64_t allocatedSize;
        int64_t overflowLength;
        int64_t overflowAllocatedSize;
        int64_t mallocFailedSize;
        ptwXYOverflowPoint lessThanEqualXPoint, greaterThanXPoint;
        ptwXYOverflowPoint overflowHeader;
        ptwXYPoint *points;
        ptwXYOverflowPoint *overflowPoints;
    } ptwXYPoints;
\end{verbatim}

The \highlight{ptwXYPoint} type is defined as:
\begin{verbatim}
typedef
    struct ptwXYPoint_s {
        double x, y;
    } ptwXYPoint;
\end{verbatim}

The type \highlight{ptwXYOverflowPoint} will not be described here as it is not used as an argument in any routine and
its members in \highlight{ptwXYPoints} should not be accessed user codes.

\subsection{C macros}
This section lists some of the C macros defined in "ptwXY.h".

\subsection{Interpolation} \label{interpolationSection}
For an $x$ value that is within the domain of a \highlight{ptwXYPoints} object but not one of its points, the \highlight{ptwXYPoints} 
routines interpolate, as instructed by the member \highlight{interpolation}, to obtain the $y$ value. Interpolation types are
defined using the type \highlight{ptwXY\_interpolation} which is defined as:
\begin{verbatim}
typedef enum ptwXY_interpolation_e { 
    ptwXY_interpolationLinLin,  /* x and y linear. */
    ptwXY_interpolationLinLog,  /* x linear and y logarithmic. */
    ptwXY_interpolationLogLin,  /* x logarithmic and y linear. */
    ptwXY_interpolationLogLog,  /* x and y logarithmic. */
    ptwXY_interpolationFlat,    /* see below */
    ptwXY_interpolationOther    /* see below */
} ptwXY_interpolation;
\end{verbatim}
The latter two interpolation types have many restrictions. For ptwXY\_interpolationFlat the $y$ for $x_i \le x < x_{i+1}$ is $y_i$.
This type is good for storing histogram type data.
Many of the functions in the \highlight{ptwXY} library cannot handle the flat interpolation and return the error
nfu\_invalidInterpolation via their nfu\_status argument. The interpolation type ptwXY\_interpolationOther
allows the use of ptwXY storage type for data that does not fit into one the other defined interpolation types. Most
functions cannot handle the other interpolation and also return the error nfu\_invalidInterpolation.

\subsection{Data types}
Currently, the \highlight{ptwXY} model only supports a point as an $(x, y)$ pair. 

\subsection{Miscellaneous types}
The routine \highlight{ptwXY\_getPointsAroundX} is used by other routines to determine
where an $x$ value fits into a \highlight{ptwXYPoints} object. The return value of this
routine is of type \highlight{ptwXY\_less\-Equal\-GreaterX} which is defined as:
\begin{verbatim}
typedef enum ptwXY_lessEqualGreaterX_e {
    ptwXY_lessEqualGreaterX_empty,          /* The object has no points. */
    ptwXY_lessEqualGreaterX_lessThan,       /* The x < xMin. */
    ptwXY_lessEqualGreaterX_equal,          /* x = x_i. */
    ptwXY_lessEqualGreaterX_between,        /* x_i < x < x_i+1. */
    ptwXY_lessEqualGreaterX_greater         /* The x > xMax. */
} ptwXY_lessEqualGreaterX;
\end{verbatim}
Here, xMin and xMax are the minimum and maximum $x$ values of the \highlight{ptwXYPoints} object, and
$x_i$ and $x_{i+1}$ are the $(i-1)^{th}$ and $i^{th}$ $x$ values of the \highlight{ptwXYPoints} object respectively.

\section{Routines}

\subsection{Core}
This section decribes all the routines in the file "ptwXY\_core.c".

\subsubsection{ptwXY\_new} \label{ptwXYnewSec}
This routine allocates memory for a new \highlight{ptwXYPoints} object and initializes it by calling \highlight{ptwXY\-\_setup}.
\setargumentNameLengths{secondarySize}
\CallingC{ptwXYPoints *ptwXY\_new(}{ptwXY\_interpolation interpolation,
    \addArgument{double biSectionMax,}
    \addArgument{double accuracy,}
    \addArgument{int64\_t primarySize,}
    \addArgument{int64\_t secondarySize,}
    \addArgument{fnu\_status *status ),}
    \addArgument{int userFlag );}}
    \argumentBox{interpolation}{The type of interpolation to use.}
    \argumentBox{biSectionMax}{The maximum disection allowed.}
    \argumentBox{accuracy}{The interpolation accuracy of the data.}
    \argumentBox{primarySize}{Initial size of the primary cache.}
    \argumentBox{secondarySize}{Initial size of the secondary cache.} 
    \argumentBox{status}{On return, the status value.}
    \argumentBox{userFlag}{An user defined integer value not used by any ptwXY function.}
    \vskip 0.05 in \noindent
If this routine fails, NULL is returned.

\subsubsection{ptwXY\_setup}
This routine initializes a \highlight{ptwXYPoints} object and must be called for a \highlight{ptwXYPoints} object
before that object can be used by any other routine in this package.
\setargumentNameLengths{secondarySize}
\CallingC{fnu\_status ptwXY\_setup(}{ptwXYPoints *ptwXY,
    \addArgument{ptwXY\_interpolation interpolation,}
    \addArgument{double biSectionMax,}
    \addArgument{double accuracy,}
    \addArgument{int64\_t primarySize,}
    \addArgument{int64\_t secondarySize );},
    \addArgument{int userFlag );}}
    \argumentBox{ptwXY}{A pointer to a \highlight{ptwXYPoints} object to initialize.}
    \argumentBox{interpolation}{The type of interpolation to use.}
    \argumentBox{biSectionMax}{The maximum disection allowed.}
    \argumentBox{accuracy}{The interpolation accuracy of the data.}
    \argumentBox{primarySize}{Initial size of the primary cache.}
    \argumentBox{secondarySize}{Initial size of the secondary cache.} 
    \argumentBox{userFlag}{An user defined integer value not used by any ptwXY function.}
    \vskip 0.05 in \noindent
The primary and secondary caches are allocated with routines \highlight{ptwXY\_reallocatePoints} and 
\highlight{ptwXY\_reallocateOverflowPoints} respectively.

\subsubsection{ptwXY\_create}
This routines combines \highlight{ptwXY\_new} and \highlight{ptwXY\_setXYData}.
\CallingC{ptwXYPoints *ptwXY\_create(}{ptwXY\_interpolation interpolation, 
    \addArgument{double biSectionMax,}
    \addArgument{double accuracy,}
    \addArgument{int64\_t primarySize,}
    \addArgument{int64\_t secondarySize,}
    \addArgument{int64\_t length,}
    \addArgument{double *xy ),}
    \addArgument{fnu\_status *status,}
    \addArgument{int userFlag );}}
    \argumentBox{interpolation}{The type of interpolation to use.}
    \argumentBox{biSectionMax}{The maximum disection allowed.}
    \argumentBox{accuracy}{The interpolation accuracy of the data.}
    \argumentBox{primarySize}{Initial size of the primary cache.}
    \argumentBox{secondarySize}{Initial size of the secondary cache.} 
    \argumentBox{length}{The number of points in xy.}
    \argumentBox{xy}{The new points given as $x_0, y_0, x_1, y_1, \; \ldots, \; x_n, y_n$ where n = length - 1.}
    \argumentBox{status}{On return, the status value.}
    \argumentBox{userFlag}{An user defined integer value not used by any ptwXY function.}
    \vskip 0.05 in \noindent
If this routine fails, NULL is returned.

\subsubsection{ptwXY\_createFrom\_Xs\_Ys}
This routines is like \highlight{ptwXY\_create} except the x and y data are given in separate arrays.
\CallingC{ptwXYPoints *ptwXY\_createFrom\_Xs\_Ys(}{ptwXY\_interpolation interpolation, 
    \addArgument{double biSectionMax,}
    \addArgument{double accuracy,}
    \addArgument{int64\_t primarySize,}
    \addArgument{int64\_t secondarySize,}
    \addArgument{int64\_t length,}
    \addArgument{double *Xs ),}
    \addArgument{double *Ys ),}
    \addArgument{fnu\_status *status,}
    \addArgument{int userFlag );}}
    \argumentBox{interpolation}{The type of interpolation to use.}
    \argumentBox{biSectionMax}{The maximum disection allowed.}
    \argumentBox{accuracy}{The interpolation accuracy of the data.}
    \argumentBox{primarySize}{Initial size of the primary cache.}
    \argumentBox{secondarySize}{Initial size of the secondary cache.} 
    \argumentBox{length}{The number of points in xy.}
    \argumentBox{Xs}{The new x points given as $x_0, x_1, \; \ldots, \; x_n$ where n = length - 1.}
    \argumentBox{Ys}{The new y points given as $y_0, y_1, \; \ldots, \; y_n$ where n = length - 1.}
    \argumentBox{status}{On return, the status value.}
    \argumentBox{userFlag}{An user defined integer value not used by any ptwXY function.}
    \vskip 0.05 in \noindent
If this routine fails, NULL is returned.

\subsubsection{ptwXY\_copy}
This routine clears the points in \highlight{dest} and then copies the points from \highlight{src} into \highlight{dest}.
The \highlight{src} object is not modified.
\setargumentNameLengths{dest}
\CallingC{fnu\_status ptwXY\_copy(}{ptwXYPoints *dest,
    \addArgument{ptwXYPoints *src );}
    }
    \argumentBox{dest}{A pointer to the destination \highlight{ptwXYPoints} object.}
    \argumentBox{src}{A pointer to the source \highlight{ptwXYPoints} object.}

\subsubsection{ptwXY\_clone}
This routine creates a new \highlight{ptwXYPoints} object and sets its points to the points in its first argument.
\CallingC{ptwXYPoints *ptwXY\_clone(}{ptwXYPoints *ptwXY, 
    \addArgument{fnu\_status *status );}}
    \argumentBox{ptwXY}{A pointer to the \highlight{ptwXYPoints} object.}
    \argumentBox{status}{On return, the status value.}
    \vskip 0.05 in \noindent
If an error occurs, NULL is returned.

\subsubsection{ptwXY\_slice}
This routine creates a new \highlight{ptwXYPoints} object and sets its points to the points from index \highlight{index1} inclusive 
to \highlight{index2} exclusive of \highlight{ptwXY}.
\setargumentNameLengths{secondarySize}
\CallingC{ptwXYPoints *ptwXY\_slice(}{ptwXYPoints *ptwXY, 
    \addArgument{int64\_t index1,}
    \addArgument{int64\_t index2,}
    \addArgument{int64\_t secondarySize,}
    \addArgument{fnu\_status *status );}}
    \argumentBox{ptwXY}{A pointer to the \highlight{ptwXYPoints} object.}
    \argumentBox{index1}{The lower index.}
    \argumentBox{index2}{The upper index.}
    \argumentBox{secondarySize}{Initial size of the secondary cahce.}
    \argumentBox{status}{On return, the status value.}
    \vskip 0.05 in \noindent
If an error occurs, NULL is returned.

\subsubsection{ptwXY\_xSlice}
This routine creates a new \highlight{ptwXYPoints} object and sets its points to the points from the points between the domain
\highlight{xMin} and \highlight{xMax} of \highlight{ptwXY}. If \highlight{fill} is true, points at xMin and xMax are added if
not in the inputted \highlight{ptwXY}.
\setargumentNameLengths{secondarySize}
\CallingC{ptwXYPoints *ptwXY\_xSlice(}{ptwXYPoints *ptwXY, 
    \addArgument{double xMin,}
    \addArgument{double xMax,}
    \addArgument{int64\_t secondarySize,}
    \addArgument{int fill,}
    \addArgument{fnu\_status *status );}}
    \argumentBox{ptwXY}{A pointer to the \highlight{ptwXYPoints} object.}
    \argumentBox{xMax}{The lower domain value.}
    \argumentBox{xMax}{The upper domain value.}
    \argumentBox{secondarySize}{Initial size of the secondary cahce.}
    \argumentBox{fill}{Initial size of the secondary cahce.}
    \argumentBox{status}{On return, the status value.}
    \vskip 0.05 in \noindent
If an error occurs, NULL is returned.

\subsubsection{ptwXY\_xMinSlice}
This routine creates a new \highlight{ptwXYPoints} object and sets its points to the points from the points between the domain
\highlight{xMin} to the end of \highlight{ptwXY}. If \highlight{fill} is true, point at xMin is added if
not in the inputted \highlight{ptwXY}.
\setargumentNameLengths{secondarySize}
\CallingC{ptwXYPoints *ptwXY\_xMinSlice(}{ptwXYPoints *ptwXY, 
    \addArgument{double xMin,}
    \addArgument{int64\_t secondarySize,}
    \addArgument{int fill,}
    \addArgument{fnu\_status *status );}}
    \argumentBox{ptwXY}{A pointer to the \highlight{ptwXYPoints} object.}
    \argumentBox{xMin}{The lower domain value.}
    \argumentBox{secondarySize}{Initial size of the secondary cahce.}
    \argumentBox{fill}{Initial size of the secondary cahce.}
    \argumentBox{status}{On return, the status value.}
    \vskip 0.05 in \noindent
If an error occurs, NULL is returned.

\subsubsection{ptwXY\_xMaxSlice}
This routine creates a new \highlight{ptwXYPoints} object and sets its points to the points from the points between the domain of
the beginning of \highlight{ptwXY} to xMax. If \highlight{fill} is true, point at xMax is added if
not in the inputted \highlight{ptwXY}.
\setargumentNameLengths{secondarySize}
\CallingC{ptwXYPoints *ptwXY\_xMaxSlice(}{ptwXYPoints *ptwXY, 
    \addArgument{double xMax,}
    \addArgument{int64\_t secondarySize,}
    \addArgument{int fill,}
    \addArgument{fnu\_status *status );}}
    \argumentBox{ptwXY}{A pointer to the \highlight{ptwXYPoints} object.}
    \argumentBox{xMax}{The upper domain value.}
    \argumentBox{secondarySize}{Initial size of the secondary cahce.}
    \argumentBox{fill}{Initial size of the secondary cahce.}
    \argumentBox{status}{On return, the status value.}
    \vskip 0.05 in \noindent
If an error occurs, NULL is returned.

\subsubsection{ptwXY\_getUserFlag}
This routine returns the value of \highlight{ptwXY}'s userFlag member.
\CallingC{int ptwXY\_getUserFlag(}{ptwXYPoints *ptwXY );}
    \argumentBox{ptwXY}{A pointer to the \highlight{ptwXYPoints} object.}

\subsubsection{ptwXY\_setUserFlag}
This routine sets the value of the \highlight{ptwXY}'s userFlag member to userFlag.
\CallingC{void ptwXY\_setUserFlag(}{ptwXYPoints *ptwXY,
    \addArgument{int userFlag);}}
    \argumentBox{ptwXY}{A pointer to the \highlight{ptwXYPoints} object.}
    \argumentBox{userFlag}{The value to set ptwXY's userFlag to.}

\subsubsection{ptwXY\_getAccuracy}
This routine returns the value of \highlight{ptwXY}'s accuracy member.
\CallingC{double ptwXY\_getAccuracy(}{ptwXYPoints *ptwXY );}
    \argumentBox{ptwXY}{A pointer to the \highlight{ptwXYPoints} object.}

\subsubsection{ptwXY\_setAccuracy}
This routine sets the value of the \highlight{ptwXY}'s accuracy member to accuracy.
\CallingC{double ptwXY\_setAccuracy(}{ptwXYPoints *ptwXY,
    \addArgument{double accuracy );}}
    \argumentBox{ptwXY}{A pointer to the \highlight{ptwXYPoints} object.}
    \argumentBox{accuracy}{The value to set ptwXY's accuracy to.}
Becuase the range of accuracy is limited, the actual value set may be different then the 
argument accuracy. The actual value set in ptwXY is returned.

\subsubsection{ptwXY\_getBiSectionMax}
This routine returns the value of \highlight{ptwXY}'s biSectionMax member.
\CallingC{double ptwXY\_getBiSectionMax(}{ptwXYPoints *ptwXY );}
    \argumentBox{ptwXY}{A pointer to the \highlight{ptwXYPoints} object.}

\subsubsection{ptwXY\_setBiSectionMax}
This routine sets the value of the \highlight{ptwXY}'s biSectionMax member to biSectionMax.
\CallingC{double ptwXY\_setBiSectionMax(}{ptwXYPoints *ptwXY,
    \addArgument{double biSectionMax );}}
    \argumentBox{ptwXY}{A pointer to the \highlight{ptwXYPoints} object.}
    \argumentBox{biSectionMax}{The value to set ptwXY's biSectionMax to.}
Becuase the range of biSectionMax is limited, the actual value set may be different then the 
argument biSectionMax. The actual value set in ptwXY is returned.

\subsubsection{ptwXY\_reallocatePoints}
This routine changes the size of the primary cache.
\setargumentNameLengths{forceSmallerResize}
\CallingC{fnu\_status ptwXY\_reallocatePoints(}{ptwXYPoints *ptwXY,
    \addArgument{int64\_t size,}
    \addArgument{int forceSmallerResize );}}
    \argumentBox{ptwXY}{A pointer to the \highlight{ptwXYPoints} object.}
    \argumentBox{size}{The desired size of the primary cache.}
    \argumentBox{forceSmallerResize}{If true (i.e. non-zero) and size is smaller than the current size, the primary cache
        is resized. Otherwise, the primary cache is only reduced if the inputted size is significantly smaller than the current size.}
    \vskip 0.05 in \noindent
The actual memory allocated is the maximum of {\tt size}, the current length of the primary cache and \highlight{ptwXY\_minimumSize}.

\subsubsection{ptwXY\_reallocateOverflowPoints}
This routine changes the size of the secondary cache.
\setargumentNameLengths{ptwXY}
\CallingC{fnu\_status ptwXY\_reallocateOverflowPoints(}{ptwXYPoints *ptwXY,
    \addArgument{int64\_t size );}}
    \argumentBox{ptwXY}{A pointer to the \highlight{ptwXYPoints} object.}
    \argumentBox{size}{The desired size of the secondary cache.}
    \vskip 0.05 in \noindent
The actual memory allocated is the maximum of {\tt size} and \highlight{ptwXY\_minimumOverflowSize}. The function
\highlight{ptwXY\_coalescePoints} is called if the current length of the secondary cache is greater than the inputted size.

\subsubsection{ptwXY\_coalescePoints}
This routine adds the points from the secondary cache to the primary cache and then removes the points from the secondary cache. If the
argument \highlight{newPoint} is not-NULL it is also added to the primary cache.
\setargumentNameLengths{forceSmallerResize}
\CallingC{fnu\_status ptwXY\_coalescePoints(}{ptwXYPoints *ptwXY,
    \addArgument{int64\_t size,}
    \addArgument{ptwXYPointsPoint *newPoint,}
    \addArgument{int forceSmallerResize );}}
    \argumentBox{ptwXY}{A pointer to the \highlight{ptwXYPoints} object.}
    \argumentBox{size}{The desired size of the primary cache.}
    \argumentBox{newPoint}{If not NULL, an additional point to add.}
    \argumentBox{forceSmallerResize}{If true (i.e. non-zero) and size is smaller than the current size, the primary cache
        is resized. Otherwise, the primary cache is only reduced if the new size is significantly smaller than the current size.}
    \vskip 0.05 in \noindent
The actual memory allocated is the maximum of {\tt size}, the new length of the \highlight{ptwXY} object and \highlight{ptwXY\_minimumSize}.

\subsubsection{ptwXY\_simpleCoalescePoints}
This routine is a simple wrapper for \highlight{ptwXY\_coalescePoints} when only coalescing of the existing points is needed.
\CallingC{fnu\_status ptwXY\_simpleCoalescePoints(}{ptwXYPoints *ptwXY );}
    \argumentBox{ptwXY}{A pointer to the \highlight{ptwXYPoints} object.}

\subsubsection{ptwXY\_clear}
This routine removes all points from a \highlight{ptwXYPoints} object but does not free any allocated memory. Upon return, the
length of the \highlight{ptwXYPoints} object is zero.
\setargumentNameLengths{ptwXY}
\CallingC{fnu\_status ptwXY\_clear(}{ptwXYPoints *ptwXY );}
    \argumentBox{ptwXY}{A pointer to the \highlight{ptwXYPoints} object.}

\subsubsection{ptwXY\_release}
This routine frees all the internal memory allocated for a \highlight{ptwXYPoints} object.
\setargumentNameLengths{ptwXY}
\CallingC{fnu\_status ptwXY\_release(}{ptwXYPoints *ptwXY );}
    \argumentBox{ptwXY}{A pointer to the \highlight{ptwXYPoints} object.}

\subsubsection{ptwXY\_free}
This routine calls \highlight{ptwXY\_release} and then calls free on \highlight{ptwXY}. 
\CallingC{fnu\_status ptwXY\_free(}{ptwXYPoints *ptwXY );}
    \argumentBox{ptwXY}{A pointer to the \highlight{ptwXYPoints} object.}
    \vskip 0.05 in \noindent
Any \highlight{ptwXYPoints} object allocated using \highlight{ptwXY\_new} should be freed calling \highlight{ptwXY\_free}.
Once this routine is called, the \highlight{ptwXYPoints} object should never be used.

\subsubsection{ptwXY\_length}
This routine returns the length (i.e., number of points in the primary and secondary caches) for a \highlight{ptwXY} object.
\CallingC{int64\_t ptwXY\_length(}{ptwXYPoints *ptwXY );}
    \argumentBox{ptwXY}{A pointer to the \highlight{ptwXYPoints} object.}

\subsubsection{ptwXY\_getNonOverflowLength}
This routine returns the length of the primary caches (note, this is not its size).
\CallingC{int64\_t ptwXY\_getNonOverflowLength(}{ptwXYPoints *ptwXY );}
    \argumentBox{ptwXY}{A pointer to the \highlight{ptwXYPoints} object.}

\subsubsection{ptwXY\_setXYData}
This routine replaces the current points in a \highlight{ptwXY} object with a new set of points.
\CallingC{fnu\_status ptwXY\_setXYData(}{ptwXYPoints *ptwXY,
    \addArgument{int64\_t length,}
    \addArgument{double *xy );}}
    \argumentBox{ptwXY}{A pointer to the \highlight{ptwXYPoints} object.}
    \argumentBox{length}{The number of points in xy.}
    \argumentBox{xy}{The new points given as $x_0, y_0, x_1, y_1, \; \ldots, \; x_n, y_n$ where n = length - 1.}

\subsubsection{ptwXY\_setXYDataFromXsAndYs}
This routines is like \highlight{ptwXY\_setXYData} except the x and y data are given in separate arrays.
\CallingC{fnu\_status ptwXY\_setXYDataFromXsAndYs(}{ptwXYPoints *ptwXY,
    \addArgument{int64\_t length,}
    \addArgument{double *Xs,}
    \addArgument{double *Ys );}}
    \argumentBox{ptwXY}{A pointer to the \highlight{ptwXYPoints} object.}
    \argumentBox{length}{The number of points in xy.}
    \argumentBox{Xs}{The new x points given as $x_0, x_1, \; \ldots, \; x_n$ where n = length - 1.}
    \argumentBox{Ys}{The new y points given as $y_0, y_1, \; \ldots, \; y_n$ where n = length - 1.}

\subsubsection{ptwXY\_deletePoints}
This routine removes all the points from index \highlight{i1} inclusive to index \highlight{i2} exclusive. Indexing is 0 based.
\CallingC{fnu\_status ptwXY\_deletePoints(}{ptwXYPoints *ptwXY,
    \addArgument{int64\_t i1,}
    \addArgument{int64\_t i2 );}}
    \argumentBox{ptwXY}{A pointer to the \highlight{ptwXYPoints} object.}
    \argumentBox{i1}{The lower index.}
    \argumentBox{i2}{The upper index.}
    \vskip 0.05 in \noindent
As example, if an \highlight{ptwXY} object contains the points (1.2, 4), (1.3, 5), (1.6, 6), (1.9, 3) (2.0, 6), (2.1, 4) 
and (2.3, 1). Then calling \highlight{ptwXY\_deletePoints} with i1 = 2 and i2 = 4 removes the points (1.6, 6) and (1.9, 3).
The indices i1 and i2 must satisfy the relationship ( 0 $\le$ i1 $\le$ i2 $\le n$ ) where $n$ is the length of the
\highlight{ptwXY} object; otherwise, no modification is done to the \highlight{ptwXY} object
and the error \highlight{nfu\_badIndex} is returned.

\subsubsection{ptwXY\_getPointAtIndex}
This routine checks that the index argument is valid, and if it is, this routine returns the result 
of \highlight{ptwXY\_getPointAtIndex\_Unsafely}. Otherwise, NULL is returned.
\CallingC{ptwXYPoint *ptwXY\_getPointAtIndex(}{ptwXYPoints *ptwXY,
    \addArgument{int64\_t index );}}
    \argumentBox{ptwXY}{A pointer to the \highlight{ptwXYPoints} object.}
    \argumentBox{index}{The index of the point to return.}

\subsubsection{ptwXY\_getPointAtIndex\_Unsafely}
This routine returns the point at index. This routine does not check if index is valid and 
thus is not intended for general use. Instead, see \highlight{ptwXY\_getPointAtIndex} for a general use version of this routine.
\CallingCLimited{ptwXYPoint *ptwXY\_getPointAtIndex\_Unsafely(}{ptwXYPoints *ptwXY,
    \addArgument{int64\_t index );}}
    \argumentBox{ptwXY}{A pointer to the \highlight{ptwXYPoints} object.}
    \argumentBox{index}{The index of the point to return.}

\subsubsection{ptwXY\_getXYPairAtIndex}
This routine calls \highlight{ptwXY\_getPointAtIndex} and if the index is valid it returns the point's x and y values via the
arguments *x and *y. Otherwise, *x and *y are unaltered and an error signal is returned.
\CallingC{ptwXYPoint *ptwXY\_getPairAtIndex(}{ptwXYPoints *ptwXY,
    \addArgument{int64\_t index,}
    \addArgument{double *x,}
    \addArgument{double *y );}}
    \argumentBox{ptwXY}{A pointer to the \highlight{ptwXYPoints} object.}
    \argumentBox{index}{The index of the point to return.}
    \argumentBox{*x}{The point's x value is returned in this argument.}
    \argumentBox{*y}{The point's y value is returned in this argument.}

\subsubsection{ptwXY\_getPointsAroundX}
This routine sets the \highlight{lessThanEqualXPoint} and \highlight{greaterThanXPoint} members of the 
\highlight{ptwXY} object to the two points that bound a point $x$.
\CallingCLimited{ptwXY\_lessEqualGreaterX ptwXY\_getPointsAroundX(}{ptwXYPoints *ptwXY,
    \addArgument{double x );}}
    \argumentBox{ptwXY}{A pointer to the \highlight{ptwXYPoints} object.}
    \argumentBox{x}{The $x$ value.}
    \vskip 0.05 in \noindent
If the \highlight{ptwXY} object is empty then the return value is \highlight{ptwXY\_less\-Equal\-GreaterX\_\-empty}.
If $x$ is less than xMin, then \highlight{ptwXY\_less\-Equal\-GreaterX\_\-less\-Than} is return.
If $x$ is greater than xMax, then \highlight{ptwXY\_less\-Equal\-GreaterX\_\-greater\-Than} is return. If $x$ corresponds to a
point in the \highlight{ptwXY} object then \highlight{ptwXY\_\-less\-Equal\-GreaterX\_\-equal} is returned. Otherwise, 
\highlight{ptwXY\_\-less\-Equal\-GreaterX\_\-between} is returned.

\subsubsection{ptwXY\_getValueAtX}
This routine gets the $y$ value at $x$, interpolating if necessary.
\CallingC{fnu\_status ptwXY\_getValueAtX(}{ptwXYPoints *ptwXY,
    \addArgument{double x,}
    \addArgument{double *y );}}
    \argumentBox{ptwXY}{A pointer to the \highlight{ptwXYPoints} object.}
    \argumentBox{x}{The $x$ value.}
    \argumentBox{y}{Upon return, contains the $y$ value.}
    \vskip 0.05 in \noindent
If the x value is outside the domain of the \highlight{ptwXY} object, $y$ is set to zero and the returned value
is \highlight{nfu\_X\-Outside\-Domain}.

\subsubsection{ptwXY\_setValueAtX}
This routine sets the point at $x$ to $y$, if $x$ does not corresponds to a
point in the \highlight{ptwXY} object then a new point is added.
\CallingC{fnu\_status ptwXY\_setValueAtX(}{ptwXYPoints *ptwXY,
    \addArgument{double x,}
    \addArgument{double y );}}
    \argumentBox{ptwXY}{A pointer to the \highlight{ptwXYPoints} object.}
    \argumentBox{x}{The $x$ value.}
    \argumentBox{y}{The $y$ value.}
    \vskip 0.05 in \noindent

\subsubsection{ptwXY\_setXYPairAtIndex}
This routine sets the $x$ and $y$ values at index.
\CallingC{fnu\_status ptwXY\_setXYPairAtIndex(}{ptwXYPoints *ptwXY,
    \addArgument{int64\_t index}
    \addArgument{double x,}
    \addArgument{double y );}}
    \argumentBox{ptwXY}{A pointer to the \highlight{ptwXYPoints} object.}
    \argumentBox{index}{The index of the point to set.}
    \argumentBox{x}{The $x$ value.}
    \argumentBox{y}{The $y$ value.}
    \vskip 0.05 in \noindent
If index is invalid, \highlight{nfu\_badIndex} is returned. If the $x$ value is not valid for index (i.e. $x \le x_{\rm index-1}$ 
or $x \ge x_{\rm index+1}$) then \highlight{nfu\_badIndexForX} is return.

\subsubsection{ptwXY\_getSlopeAtX}
This routine calculates the slope at the point $x$ assuming linear-linear interpolation. That is, for $x_i < x < x_{i+1}$,
the slope is $( y_{i+1} - y_i ) / ( x_{i+1} - x_i )$. If $x = x_j$ is the point in \highlight{ptwXY} at
index $j$ then for side = `+', $i = j$ is used in the above slope equation. Else, if side = `-', $i = j-1$ is used in the above slope equation.
\CallingC{fnu\_status ptwXY\_getSlopeAtX(}{ptwXYPoints *ptwXY,
    \addArgument{double x,}
    \addArgument{const char side,}
    \addArgument{double *slope );}}
    \argumentBox{ptwXY}{A pointer to the \highlight{ptwXYPoints} object.}
    \argumentBox{index}{The index of the point to set.}
    \argumentBox{x}{The $x$ value.}
    \argumentBox{y}{The $y$ value.}
    \vskip 0.05 in \noindent
If side is neither '-' or '+', the error \highlight{nfu\_badInput} is returned.

\subsubsection{ptwXY\_getXMinAndFrom --- Not for general use}
This routine returns the xMin value and indicates whether the minimum value resides in the primary 
or secondary cache.
\setargumentNameLengths{dataFrom}
\CallingCLimited{double ptwXY\_getXMinAndFrom(}{ptwXYPoints *ptwXY,
    \addArgument{ptwXY\_dataFrom *dataFrom );}}
    \argumentBox{ptwXY}{A pointer to the \highlight{ptwXYPoints} object.}
    \argumentBox{dataFrom}{The output of this argument indicates which cache the minimum value resides in.}
    \vskip 0.05 in \noindent
The return value from this routine is xMin. If there are no data in the \highlight{ptwXYPoints} object, then \highlight{dataFrom} is set
to \highlight{ptwXY\_dataFrom\_Unknown}. Otherwise, it is set to \highlight{ptwXY\_data\-From\_Points} or \highlight{ptwXY\_dataFrom\_Overflow} if the 
minimum value is in the primary or secondary cache respectively.

\subsubsection{ptwXY\_getXMin}
This routine returns the xMin value returned by \highlight{ptwXY\_getXMinAndFrom}. The calling routine should check that the
\highlight{ptwXYPoints} object contains at least one point (i.e., that the length is greater than 0). If the length is 0, the return value is
undefined.
\CallingC{double ptwXY\_getXMin(}{ptwXYPoints *ptwXY );}
    \argumentBox{ptwXY}{A pointer to the \highlight{ptwXYPoints} object.}

\subsubsection{ptwXY\_getXMaxAndFrom --- Not for general use}
This routine returns the xMax value and indicates whether the maximum value resides in the primary or secondary cache.
\CallingCLimited{double ptwXY\_getXMaxAndFrom(}{ptwXYPoints *ptwXY,
    \addArgument{ptwXY\_dataFrom *dataFrom );}}
    \argumentBox{ptwXY}{A pointer to the \highlight{ptwXYPoints} object.}
    \argumentBox{dataFrom}{The output of this argument indicates which cache the maximum value resides in.}
    \vskip 0.05 in \noindent
The return value from this routine is xMax. If there are no data in the \highlight{ptwXYPoints} object, then \highlight{dataFrom} is set
to \highlight{ptwXY\_dataFrom\_Unknown}. Otherwise, it is set to \highlight{ptwXY\_data\-From\_Points} or \highlight{ptwXY\_dataFrom\_Overflow} if the 
maximum value is in the primary or secondary cache respectively.

\subsubsection{ptwXY\_getXMax}
This routine returns the xMax value returned by \highlight{ptwXY\_getXMinAndFrom}. The calling routine should check that the
\highlight{ptwXYPoints} object contains at least one point (i.e., that the length is greater than 0). If the length is 0, the return value is
undefined.
\CallingC{double ptwXY\_getXMax(}{ptwXYPoints *ptwXY );}
    \argumentBox{ptwXY}{A pointer to the \highlight{ptwXYPoints} object.}

\subsubsection{ptwXY\_getYMin}
This routine returns the minimum y value in \highlight{ptwXY}.
\CallingC{double ptwXY\_getYMin(}{ptwXYPoints *ptwXY );}
    \argumentBox{ptwXY}{A pointer to the \highlight{ptwXYPoints} object.}

\subsubsection{ptwXY\_getYMax}
This routine returns the maximum y value in \highlight{ptwXY}.
\CallingC{double ptwXY\_getYMax(}{ptwXYPoints *ptwXY );}
    \argumentBox{ptwXY}{A pointer to the \highlight{ptwXYPoints} object.}

\subsubsection{ptwXY\_initialOverflowPoint --- Not for general use}
This routine initializes a point in the secondary cache.
\CallingCLimited{void ptwXY\_initialOverflowPoint(\hskip -1. in}{
    \addArgument{ptwXYOverflowPoint *overflowPoint,}
    \addArgument{ptwXYOverflowPoint *prior,}
    \addArgument{ptwXYOverflowPoint *next );}}
    \argumentBox{ptwXY}{A pointer to the \highlight{ptwXYPoints} object.}
    \argumentBox{prior}{The prior point in the linked list.}
    \argumentBox{next}{The next point in the linked list.}

\subsection{Methods}
This section decribes all the functions in the file ``ptwXY\_method.c''.

\subsubsection{ptwXY\_clip}
This function clips the y-values of \highlight{ptwXY} to be within the range \highlight{rangeMin} and \highlight{rangeMax}.
Points will be added to insure that the curve within the \highlight{rangeMin} and \highlight{rangeMax} is not altered.
\setargumentNameLengths{allocatedSize}
\CallingC{fnu\_status ptwXY\_clip(}{statusMessageReporting *smr,
    \addArgument{ptwXYPoints *ptwXY,}
    \addArgument{double rangeMin,}
    \addArgument{double rangeMax );}}
    \argumentBox{smr}{The \highlight{statusMessageReporting} instance to record errors.}
    \argumentBox{ptwXY}{A pointer to the \highlight{ptwXYPoints} object.}
    \argumentBox{rangeMin}{All y-values in \highlight{ptwXY} will be greater than or equal to this value.}
    \argumentBox{rangeMax}{All y-values in \highlight{ptwXY} will be less than or equal to this value.}

\subsubsection{ptwXY\_thicken}
This function thicken the points in \highlight{ptwXY} by adding points as determined by the input parameters.
\setargumentNameLengths{sectionSubdivideMax}
\CallingC{fnu\_status ptwXY\_thicken(}{statusMessageReporting *smr,
    \addArgument{ptwXYPoints *ptwXY,}
    \addArgument{int sectionSubdivideMax,}
    \addArgument{double dDomainMax,}
    \addArgument{double fDomainMax );}}
    \argumentBox{smr}{The \highlight{statusMessageReporting} instance to record errors.}
    \argumentBox{ptwXY}{A pointer to the \highlight{ptwXYPoints} object.}
    \argumentBox{sectionSubdivideMax}{The maximum number of points to add between two initial consecutive points.}
    \argumentBox{dDomainMax}{The desired maximum absolute x step between consecutive points.}
    \argumentBox{fDomainMax}{The desired maximum relative x step between consecutive points.}
This function adds points so that $x_{j+1} - x_j \le \highlight{dDoaminMax}$ and $x_{j+1} / x_j \le \highlight{fDomainMax}$ but will never add
more then \highlight{sectionSubdivideMax} points between any of the orginal points. If \highlight{sectionSubdivideMax} $<$ 1
or \highlight{dDomainMax} $<$ 0 or \highlight{fDomainMax} $<$ 1, the error \highlight{nfu\_badInput} is return.

\subsubsection{ptwXY\_thin}
This function returns a clone of \highlight{ptwXY} with its points thinned (i.e., removed) while maintaining interpolation 
\highlight{accuracy} with \highlight{ptwXY}.
\setargumentNameLengths{accuracy}
\CallingC{ptwXPoints *ptwXY\_thin(}{statusMessageReporting *smr,
    \addArgument{ptwXYPoints *ptwXY,}
    \addArgument{double accuracy );}}
    \argumentBox{smr}{The \highlight{statusMessageReporting} instance to record errors.}
    \argumentBox{ptwXY}{A pointer to the \highlight{ptwXYPoints} object.}
    \argumentBox{accuracy}{The accuracy of the thinned \highlight{ptwXYPoints} object.}
    \vskip 0.05 in \noindent

\subsubsection{ptwXY\_thinDomain}
This function returns a clone of \highlight{ptwXYPoints} whose x-values are those of \highlight{ptwXY} but thinned so that
\begin{equation}
    x[i+1] - x[i] \ge 0.5 \times \rm{epsilon} \times (x[i+1] + x[i]) \ \ \ .
\end{equation}
\setargumentNameLengths{accuracy}
\CallingC{ptwXPoints *ptwXY\_thinDomain(}{statusMessageReporting *smr,
    \addArgument{ptwXYPoints *ptwXY,}
    \addArgument{double epsilon );}}
    \argumentBox{smr}{The \highlight{statusMessageReporting} instance to record errors.}
    \argumentBox{ptwXY}{A pointer to the \highlight{ptwXYPoints} object.}
    \argumentBox{epsilon}{The epsilon to thin the x-values to.}
    \vskip 0.05 in \noindent

\subsubsection{ptwXY\_trim}
This function removes all extra 0.'s at the beginning and end of \highlight{ptwXY}.
\CallingC{fnu\_status ptwXY\_trim(}{statusMessageReporting *smr,
    \addArgument{ptwXYPoints *ptwXY );}}
    \argumentBox{smr}{The \highlight{statusMessageReporting} instance to record errors.}
    \argumentBox{ptwXY}{A pointer to the \highlight{ptwXYPoints} object.}
    \vskip 0.05 in \noindent
If \highlight{ptwXYPoints} starts (ends) with more than two 0.'s then all intermediary are removed.

\subsubsection{ptwXY\_union}
This function creates a new \highlight{ptwXY} instance whose x-values are the union of \highlight{ptwXY1}'s and \highlight{ptwXY2}'s 
x-values. The domains of \highlight{ptwXY1} and \highlight{ptwXY2} do not have to be mutual.
\setargumentNameLengths{unionOptions}
\CallingC{ptwXYPoints *ptwXY\_union(}{statusMessageReporting *smr,
    \addArgument{ptwXYPoints *ptwXY1,}
    \addArgument{ptwXYPoints *ptwXY2,}
    \addArgument{int unionOptions );}}
    \argumentBox{smr}{The \highlight{statusMessageReporting} instance to record errors.}
    \argumentBox{ptwXY1}{A pointer to a \highlight{ptwXYPoints} object.}
    \argumentBox{ptwXY2}{A pointer to a \highlight{ptwXYPoints} object.}
    \argumentBox{unionOptions}{Specifies options (see below).}
    \vskip 0.05 in \noindent
If an error occurs, NULL is returned. The default behavior of this function can be altered by setting bits
in the argument \highlight{unionOptions} . Currently, there are two bits, set via the C marcos
\highlight{ptwXY\_union\_fill} and \highlight{ptwXY\_union\_trim},
that alter \highlight{ptwXY\_union}'s behavior. The macro \highlight{ptwXY\_\-union\_\-fill} causes all y-values of the new \highlight{ptwXYPoints} object
to be filled via the y-values of \highlight{ptwXY1}; otherwise, the y-values are all zero. 
Normally, the new \highlight{ptwXYPoints} object's x domain spans all x-values
in both \highlight{ptwXY1} and \highlight{ptwXY2}. The macro \highlight{ptwXY\_\-union\_\-trim} limits the x domain to the common x domain
of \highlight{ptwXY1} and \highlight{ptwXY2}.

The returned \highlight{ptwXYPoints} object will always contain no points in the \highlight{overflowPoints} region.

\subsubsection{ptwXY\_scaleOffsetXAndY}
This function scales and offset the x-values and y-values.
\CallingC{nfu\_status ptwXY\_scaleOffsetXAndY(}{statusMessageReporting *smr,
    \addArgument{ptwXYPoints *ptwXY1,}
    \addArgument{double domainScale,}
    \addArgument{double domainOffset,}
    \addArgument{double rangeScale,}
    \addArgument{double rangeOffset );}}
    \argumentBox{smr}{The \highlight{statusMessageReporting} instance to record errors.}
    \argumentBox{domainScale}{The scale for the x-values.}
    \argumentBox{domainOffset}{The offset for the x-values.}
    \argumentBox{rangeScale}{The scale for the y-values.}
    \argumentBox{rangeOffset}{The offset for the y-values.}


\subsection{Unitary operators}
This section decribes all the routines in the file "ptwXY\_unitaryOperators.c".

\subsubsection{ptwXY\_abs}
This routine applies the math absolute operation to every y-value in \highlight{ptwXY}.
\CallingC{fnu\_status ptwXY\_abs(}{ptwXYPoints *ptwXY );}
    \argumentBox{ptwXY}{A pointer to the \highlight{ptwXYPoints} object.}

\subsubsection{ptwXY\_neg}
This routine applies the math negate operation to every y-value in \highlight{ptwXY}.
\CallingC{fnu\_status ptwXY\_neg(}{ptwXYPoints *ptwXY );}
    \argumentBox{ptwXY}{A pointer to the \highlight{ptwXYPoints} object.}

\subsection{Binary operators}
This section decribes all the routines in the file "ptwXY\_binaryOperators.c".

\subsubsection{ptwXY\_slopeOffset}
This routine applies the math operation ( $y_i$ = slope $\times$ $y_i$ + offset ) to the y-values of \highlight{ptwXY}.
\setargumentNameLengths{offset}
\CallingC{fnu\_status ptwXY\_slopeOffset(}{ptwXYPoints *ptwXY,
    \addArgument{double slope,}
    \addArgument{double offset );}}
    \argumentBox{ptwXY}{A pointer to the \highlight{ptwXYPoints} object.}
    \argumentBox{slope}{The slope.}
    \argumentBox{offset}{The offset.}
    \vskip 0.05 in \noindent

\subsubsection{ptwXY\_add\_double}
This routine applies the math operation ( $y_i$ = $y_i$ + offset ) to the y-values of \highlight{ptwXY}.
\CallingC{fnu\_status ptwXY\_add\_double(}{ptwXYPoints *ptwXY,
    \addArgument{double offset );}}
    \argumentBox{ptwXY}{A pointer to the \highlight{ptwXYPoints} object.}
    \argumentBox{offset}{The offset.}

\subsubsection{ptwXY\_sub\_doubleFrom}
This routine applies the math operation ( $y_i$ = $y_i$ - offset ) to the y-values of \highlight{ptwXY}.
\CallingC{fnu\_status ptwXY\_sub\_double(}{ptwXYPoints *ptwXY,
    \addArgument{double offset );}}
    \argumentBox{ptwXY}{A pointer to the \highlight{ptwXYPoints} object.}
    \argumentBox{offset}{The offset.}

\subsubsection{ptwXY\_sub\_fromDouble}
This routine applies the math operation ( $y_i$ = offset - $y_i$ ) to the y-values of \highlight{ptwXY}.
\CallingC{fnu\_status ptwXY\_sub\_fromDouble(}{ptwXYPoints *ptwXY,
    \addArgument{double offset );}}
    \argumentBox{ptwXY}{A pointer to the \highlight{ptwXYPoints} object.}
    \argumentBox{offset}{The offset.}

\subsubsection{ptwXY\_mul\_double}
This routine applies the math operation ( $y_i$ = slope $\times$ $y_i$ ) to the y-values of \highlight{ptwXY}.
\CallingC{fnu\_status ptwXY\_mul\_double(}{ptwXYPoints *ptwXY,
    \addArgument{double slope );}}
    \argumentBox{ptwXY}{A pointer to the \highlight{ptwXYPoints} object.}
    \argumentBox{slope}{The slope.}

\subsubsection{ptwXY\_div\_doubleFrom}
This routine applies the math operation ( $y_i$ = $y_i$ / divisor ) to the y-values of \highlight{ptwXY}.
\CallingC{fnu\_status ptwXY\_div\_doubleFrom(}{ptwXYPoints *ptwXY,
    \addArgument{double divisor );}}
    \argumentBox{ptwXY}{A pointer to the \highlight{ptwXYPoints} object.}
    \argumentBox{divisor}{The divisor.}
If \highlight{divisor} is zero, the error \highlight{nfu\_divByZero} is returned.

\subsubsection{ptwXY\_div\_fromDouble}
This routine applies the math operation ( $y_i$ = dividend / $y_i$ ) to the y-values of \highlight{ptwXY}.
\setargumentNameLengths{dividend}
\CallingC{fnu\_status ptwXY\_div\_fromDouble(}{ptwXYPoints *ptwXY,
    \addArgument{double dividend );}}
    \argumentBox{ptwXY}{A pointer to the \highlight{ptwXYPoints} object.}
    \argumentBox{dividend}{The dividend.}
    \vskip 0.05 in \noindent
This routine does not handle safe division (see Section~\ref{ptwXYdivptwXY}). One way to do safe division is
to use the routine \highlight{ptwXY\_valueTo\_ptwXY} to convert the \highlight{dividend} value to a \highlight{ptwXYPoints} object and
then use \highlight{ptwXY\_div\_ptwXY}.

\subsubsection{ptwXY\_mod}
\setargumentNameLengths{ptwXY}
This routine gives the remainer of $y_i$ divide by $m$. That is, it set \highlight{ptwXY}'s y-values to 
\begin{equation}
    y_i = {\rm mod}( y_i, m ) \ \ \ \ .
\end{equation}
\setargumentNameLengths{pythonMod}
\CallingC{fnu\_status ptwXY\_mod(}{ptwXYPoints *ptwXY,
    \addArgument{double m,}
    \addArgument{int pythonMod );}}
    \argumentBox{ptwXY}{A pointer to the \highlight{ptwXYPoints} object.}
    \argumentBox{m}{The modulus.}
    \argumentBox{pythonMod}{Controls whether the Python or C form of mod is implemented.}
    \vskip 0.05 in \noindent
Python's and C's mod functions act differently for negative values. If \highlight{pythonMod} then the Python form is executed;
otherwise, the C form is executed.

\subsubsection{ptwXY\_binary\_ptwXY}
This routine creates a new \highlight{ptwXYPoints} object from the union of \highlight{ptwXY1} and \highlight{ptwXY2} and then
applies the math operation
\begin{equation}
    y_i(x_i) = s_1 \times y_1(x_i) + s_2 \times y_2(x_i) + s_{12} \times y_1(x_i) \times y_2(x_i)
\end{equation}
to the new object. Here ($x_i,y_i$) is a point in the new object, $y_1(x_i)$ is \highlight{ptwXY1}'s y-value at $x_i$ and 
$y_2(x_i)$ is \highlight{ptwXY2}'s y-value at $x_i$.
This routine is used internally to add, subtract and multiply two \highlight{ptwXYPoints} objects. For example, addition is performed
by setting $s_1$ and $s_2$ to 1. and $s_{12}$ to 0.
\setargumentNameLengths{ptwXY1}
\CallingC{ptwXYPoints *ptwXY\_binary\_ptwXY(}{ptwXYPoints *ptwXY1,
    \addArgument{ptwXYPoints *ptwXY2,}
    \addArgument{double s1,}
    \addArgument{double s2,}
    \addArgument{double s12,}
    \addArgument{fnu\_status *status );}}
    \argumentBox{ptwXY1}{A pointer to a \highlight{ptwXYPoints} object.}
    \argumentBox{ptwXY2}{A pointer to a \highlight{ptwXYPoints} object.}
    \argumentBox{s1}{The value $s_1$.}
    \argumentBox{s2}{The value $s_2$.}
    \argumentBox{s12}{The value $s_{12}$.}
    \argumentBox{status}{On return, the status value.}

\subsubsection{ptwXY\_add\_ptwXY}
This routine adds two \highlight{ptwXYPoints} objects and returns the result as a new \highlight{ptwXYPoints} object
(i.e., it calls ptwXY\_binary\_ptwXY with $s_1 = s_2 = 1.$ and $s_{12} = 0.$).
\CallingC{ptwXYPoints *ptwXY\_add\_ptwXY(}{ptwXYPoints *ptwXY1,
    \addArgument{ptwXYPoints *ptwXY2,}
    \addArgument{fnu\_status *status );}}
    \argumentBox{ptwXY1}{A pointer to a \highlight{ptwXYPoints} object.}
    \argumentBox{ptwXY2}{A pointer to a \highlight{ptwXYPoints} object.}
    \argumentBox{status}{On return, the status value.}

\subsubsection{ptwXY\_sub\_ptwXY}
This routine subtracts one \highlight{ptwXYPoints} objects from another, and returns the result as a new \highlight{ptwXY} object
(i.e., it calls ptwXY\_binary\_ptwXY with $s_1 = 1.$, $s_2 = -1.$ and $s_{12} = 0.$).
\CallingC{ptwXYPoints *ptwXY\_sub\_ptwXY(}{ptwXYPoints *ptwXY1,
    \addArgument{ptwXYPoints *ptwXY2,}
    \addArgument{fnu\_status *status );}}
    \argumentBox{ptwXY1}{A pointer to a \highlight{ptwXYPoints} object which is the minuend.}
    \argumentBox{ptwXY2}{A pointer to a \highlight{ptwXYPoints} object which is the subtrahend.}
    \argumentBox{status}{On return, the status value.}

\subsubsection{ptwXY\_mul\_ptwXY}
This routine multiplies two \highlight{ptwXYPoints} objects and returns the result as a new \highlight{ptwXY} object
(i.e., it calls ptwXY\_binary\_ptwXY with $s_1 = s_2 = 0.$ and $s_{12} = 1.$).
\CallingC{ptwXYPoints *ptwXY\_mul\_ptwXY(}{ptwXYPoints *ptwXY1,
    \addArgument{ptwXYPoints *ptwXY2,}
    \addArgument{fnu\_status *status );}}
    \argumentBox{ptwXY1}{A pointer to a \highlight{ptwXYPoints} object.}
    \argumentBox{ptwXY2}{A pointer to a \highlight{ptwXYPoints} object.}
    \argumentBox{status}{On return, the status value.}

\subsubsection{ptwXY\_mul2\_ptwXY}
This routine multiplies two \highlight{ptwXYPoints} objects and returns the result as a new \highlight{ptwXY} object.
Unlike \highlight{ptwXY\_mul\_ptwXY}, this routine will infill to obtain the desired accuracy.
\CallingC{ptwXYPoints *ptwXY\_mul2\_ptwXY(}{ptwXYPoints *ptwXY1,
    \addArgument{ptwXYPoints *ptwXY2,}
    \addArgument{fnu\_status *status );}}
    \argumentBox{ptwXY1}{A pointer to a \highlight{ptwXYPoints} object.}
    \argumentBox{ptwXY2}{A pointer to a \highlight{ptwXYPoints} object.}
    \argumentBox{status}{On return, the status value.}

\subsubsection{ptwXY\_div\_ptwXY} \label{ptwXYdivptwXY}
This routine divides two \highlight{ptwXYPoints} objects and returns the result as a new \highlight{ptwXY} object.
\setargumentNameLengths{safeDivide}
\CallingC{ptwXYPoints *ptwXY\_div\_ptwXY(}{ptwXYPoints *ptwXY1,
    \addArgument{ptwXYPoints *ptwXY2,}
    \addArgument{fnu\_status *status,;}
    \addArgument{int safeDivide );}}
    \argumentBox{ptwXY1}{A pointer to a \highlight{ptwXYPoints} object.}
    \argumentBox{ptwXY2}{A pointer to a \highlight{ptwXYPoints} object.}
    \argumentBox{status}{On return, the status value.}
    \argumentBox{safeDivide}{If true safe division is performed.}

\subsection{Functions}
This section decribes all the routines in the file "ptwXY\_functions.c".

\subsubsection{ptwXY\_pow}
\setargumentNameLengths{ptwXY}
This routine applies the math operation $y_i = y_i^{p}$ to the y-values of \highlight{ptwXY}.
\CallingC{fnu\_status ptwXY\_pow(}{ptwXYPoints *ptwXY,
    \addArgument{double p );}}
    \argumentBox{ptwXY}{A pointer to the \highlight{ptwXYPoints} object.}
    \argumentBox{p}{The exponent.}
    \vskip 0.05 in \noindent
This routine infills to maintain the initial accuracy.

\subsubsection{ptwXY\_exp}
\setargumentNameLengths{ptwXY}
This routine applies the math operation $y_i = \exp( a \, y_i )$ to the y-values of \highlight{ptwXY}.
\CallingC{fnu\_status ptwXY\_exp(}{ptwXYPoints *ptwXY,
    \addArgument{double a );}}
    \argumentBox{ptwXY}{A pointer to the \highlight{ptwXYPoints} object.}
    \argumentBox{a}{The exponent coefficient.}
    \vskip 0.05 in \noindent
This routine infills to maintain the initial accuracy.

\subsubsection{ptwXY\_convolution}
This routine returns the convolution of \highlight{ptwXY1} and \highlight{ptwXY2}.
\setargumentNameLengths{status}
\CallingC{ptwXYPoints *ptwXY\_convolution(}{ptwXYPoints *ptwXY1,
    \addArgument{ptwXYPoints *ptwXY2,}
    \addArgument{nfu\_status *status,}
    \addArgument{int mode );}}
    \argumentBox{ptwXY1}{A pointer to a \highlight{ptwXYPoints} object.}
    \argumentBox{ptwXY2}{A pointer to a \highlight{ptwXYPoints} object.}
    \argumentBox{status}{On return, the status value.}
    \argumentBox{mode}{Flag to determine the initial x-values for calculating the convolutions.}
    \vskip 0.05 in \noindent
User should set \highlight{mode} to 0. 

\subsection{Interpolation}
This section decribes all the functions in the file ``ptwXY\_interpolation.c''.

\subsubsection{ptwXY\_interpolatePoint}
\setargumentNameLengths{interpolation}
This function interpolates an $x$ value between the points (x1,y1) and (x2,y2) to obtain its $y$ value
for the requested \highlight{interpolation}.
\CallingC{fnu\_status ptwXY\_interpolatePoint(}{statusMessageReporting *smr,
    \addArgument{ptwXY\_interpolation interpolation,}
    \addArgument{double x,}
    \addArgument{double *y,}
    \addArgument{double x1,}
    \addArgument{double y1,}
    \addArgument{double x2,}
    \addArgument{double y2 );}}
    \argumentBox{smr}{The \highlight{statusMessageReporting} instance to record errors.}
    \argumentBox{interpolation}{Type of interpolation to perform (see Section~\ref{interpolationSection}).}
    \argumentBox{x}{The $x$ value at which the $y$ value is desired.}
    \argumentBox{x1}{The $x$ value of the first point.}
    \argumentBox{y1}{The $y$ value of the first point.}
    \argumentBox{x2}{The $x$ value of the second point.}
    \argumentBox{y2}{The $y$ value of the second point.}
    \vskip 0.05 in \noindent
If the interpolation flag is invalid or ( x1 $>$ x2 ) then \highlight{nfu\_invalid\-Interpolation} is returned. 
If logarithm interpolation is requested for an axis, and one of the input values for that axis is less than or equal to 0., 
then \highlight{nfu\_invalid\-Interpolation} is also returned. If interpolation is \highlight{ptwXY\_interpolationOther} then
\highlight{nfu\_otherInterpolation} is returned.

\subsubsection{ptwXY\_flatInterpolationToLinear}
This function returns a linear-linear interpolated representation of \highlight{ptwXY}.
\setargumentNameLengths{upperEps}
\CallingC{ptwXYPoints *ptwXY\_flatInterpolationToLinear(}{statusMessageReporting *smr,
    \addArgument{ptwXYPoints *ptwXY,}
    \addArgument{double lowerEps,}
    \addArgument{double upperEps );}}
    \argumentBox{smr}{The \highlight{statusMessageReporting} instance to record errors.}
    \argumentBox{ptwXY}{A pointer to a \highlight{ptwXYPoints} object.}
    \argumentBox{lowerEps}{The amount to adjust every interior point down in x.}
    \argumentBox{upperEps}{The amount to adjust every interior point up in x}
    \vskip 0.05 in \noindent
For every interior point (i.e., $(x_i,y_i)$ for $0 < i < n - 1$ where n is the number of points), two points may be added.
The positions of these points depend on \highlight{lowerEps} and \highlight{upperEps} as follows:
\begin{description}
    \item[lowerEps ==  0 and upperEps ==  0:] This condition is not allowed. status is set to \highlight{nfu\_bad\-Input} and NULL is returned.
        This condition is also returned if either \highlight{lowerEps} or \highlight{upperEps} is negative.
    \item[lowerEps $>$ 0 and upperEps ==  0:] At each interior point $(x_i,y_i)$ the two points $(x_m,y_{i-1})$ and $(x_i,y_i)$ are set.
    \item[lowerEps ==  0 and upperEps $>$ 0:] At each interior point $(x_i,y_i)$ the two points $(x_i,y_{i-1})$ and $(x_p,y_i)$ are set.
    \item[lowerEps $>$ 0 and upperEps $>$ 0:] At each interior point $(x_i,y_i)$, this point is removed and the two 
        points $(x_m,y_{i-1})$ and $(x_p,y_i)$ are set.
\end{description}
where $x_m$ and $x_p$ are given in Table~\ref{flatInterpolationToLinear}.
\begin{table}
\begin{center}
\begin{tabular}{|c|l|l|}  \hline
                & $x_m$                    & $x_p$                          \\ \hline \hline
    $x_i <  0$  & $x_i ( 1 + \epsilon_l )$ & $x_p = x_i ( 1 - \epsilon_p )$ \\ \hline
    $x_i == 0$  & $ -\epsilon_l $          & $ \epsilon_p $                 \\ \hline
    $x_i >  0$  & $x_i ( 1 - \epsilon_l )$ & $x_p = x_i ( 1 + \epsilon_p )$ \\ \hline
\end{tabular}
\end{center}
\caption{The value of $x_m$ and $x_p$ used to adjust interior points in \highlight{ptwXY\_fla-Interpolation\-To\-Linear}. 
    Here, $ \epsilon_l = $ \highlight{lowerEps} and $ \epsilon_p = $ \highlight{upperEps}. \label{flatInterpolationToLinear}}
\end{table}

\subsubsection{ptwXY\_toOtherInterpolation}
This function returns \highlight{ptwXY} converted to interpolation \highlight{interpolation}.
\setargumentNameLengths{interpolation}
\CallingC{ptwXYPoints *ptwXY\_toOtherInterpolation(}{statusMessageReporting *smr,
    \addArgument{ptwXYPoints *ptwXY,}
    \addArgument{ptwXY\_interpolation interpolation,}
    \addArgument{double accuracy );}}
    \argumentBox{smr}{The \highlight{statusMessageReporting} instance to record errors.}
    \argumentBox{ptwXY}{A pointer to a \highlight{ptwXYPoints} object.}
    \argumentBox{interpolation}{The interpolation to convert to.}
    \argumentBox{accuracy}{The accuracy of the conversion.}
    \vskip 0.05 in \noindent
Currently, \highlight{interpolation} can only be \highlight{ptwXY\_\-interpolation\-LinLin}.

\subsubsection{ptwXY\_toUnitbase}
This function returns a unit-based version of \highlight{ptwXY}.
\setargumentNameLengths{scaleRange}
\CallingC{ptwXYPoints *ptwXY\_toUnitbase(}{statusMessageReporting *smr,
    \addArgument{ptwXYPoints *ptwXY,}
    \addArgument{int scaleRange );}}
    \argumentBox{smr}{The \highlight{statusMessageReporting} instance to record errors.}
    \argumentBox{ptwXY}{A pointer to the \highlight{ptwXYPoints} object.}
    \argumentBox{scaleRange}{The y-values are not scaled if this is 0.}
    \vskip 0.05 in \noindent
Unitbasing maps the domain to 0 to 1 by scaling each x-value as 
\begin{equation}
    x_i = ( x_i - x_0 ) / ( x_{n-1} - x_0 )
\end{equation}
and if \highlight{scaleRange} is not 0, scaling each y-value as
\begin{equation}
    y_i = y_i \times ( x_{n-1} - x_0 ) \ \ \ . 
\end{equation}
Unitbasing is most useful for pdf's.

\subsubsection{ptwXY\_fromUnitbase}
This function undoes the unit base mapping done by \highlight{ptwXY\_toUnitbase}.
\setargumentNameLengths{scaleRange}
\CallingC{ptwXYPoints *ptwXY\_fromUnitbase(}{statusMessageReporting *smr,
    \addArgument{ptwXYPoints *ptwXY,}
    \addArgument{double domainMin,}
    \addArgument{double domainMax,}
    \addArgument{int scaleRange );}}
    \argumentBox{smr}{The \highlight{statusMessageReporting} instance to record errors.}
    \argumentBox{ptwXY}{A pointer to the \highlight{ptwXYPoints} object.}
    \argumentBox{domainMin}{The lower domain for the returned \highlight{ptwXYPoints} instances.}
    \argumentBox{domainMax}{The upper domain for the returned \highlight{ptwXYPoints} instances.}
    \argumentBox{scaleRange}{The y-values are not scaled if this is 0.}
    \vskip 0.05 in \noindent
Each x-value is scaled as 
\begin{equation}
    x_i = ( {\rm domainMax} - {\rm domainMin} ) \times x_i + {\rm domainMin}
\end{equation}
and if \highlight{scaleRange} is not 0, each y-value is scaled as 
\begin{equation}
y_i = y_i / ( {\rm domainMax} - {\rm domainMin} ) \ \ \ \ 
\end{equation}

\subsubsection{ptwXY\_unitbaseInterpolate}
This function returns a \highlight{ptwXYPoints} instance that is the unit-base interpolation of \highlight{ptwXY1} at $w_1$
and \highlight{ptwXY2} at $w_2$ at the w-value $w$.
\setargumentNameLengths{interpolation}
\CallingC{ptwXYPoints *ptwXY\_unitbaseInterpolate(}{statusMessageReporting *smr,
    \addArgument{double w,}
    \addArgument{double w1,}
    \addArgument{ptwXYPoints *ptwXY1,}
    \addArgument{double w2,}
    \addArgument{ptwXYPoints *ptwXY2,}
    \addArgument{scaleRange );}}
    \argumentBox{smr}{The \highlight{statusMessageReporting} instance to record errors.}
    \argumentBox{w}{The w-value to interpole to.}
    \argumentBox{w1}{The lower w-value}
    \argumentBox{ptwXY1}{A pointer to a \highlight{ptwXYPoints} object at w1.}
    \argumentBox{w2}{The upper w-value}
    \argumentBox{ptwXY2}{A pointer to a \highlight{ptwXYPoints} object at w2.}
    \argumentBox{scaleRange}{The y-values are not scaled if this is 0.}
    \vskip 0.05 in \noindent

\subsection{Integration}
This section decribes all the routines in the file "ptwXY\_integration.c".

\subsubsection{ptwXY\_f\_integrate}
This routine returns the integral bewteen two points.
\setargumentNameLengths{interpolation}
\CallingC{nfu\_status ptwXY\_f\_integrate(}{ptwXYPoints *ptwXY,
    \addArgument{ptwXY\_interpolation interpolation,}
    \addArgument{double x1,}
    \addArgument{double y1,}
    \addArgument{double x2,}
    \addArgument{double y2,}
    \addArgument{double *value );}}
    \argumentBox{ptwXY}{A pointer to a \highlight{ptwXYPoints} object.}
    \argumentBox{interpolation}{The interpolation bewteen the two points.}
    \argumentBox{x2}{The x-value of the lower point.}
    \argumentBox{y2}{The y-value of the lower point}
    \argumentBox{x2}{The x-value of the upper point.}
    \argumentBox{y2}{The y-value of the upper point}
    \argumentBox{value}{On return, the value of the integral.}
    \vskip 0.05 in \noindent

\subsubsection{ptwXY\_integrate}
This routine returns the integral of \highlight{ptwXY} from \highlight{xMin} to \highlight{xMax}.
\setargumentNameLengths{ptwXY}
\CallingC{ptwXPoints *ptwXY\_integrate(}{ptwXYPoints *ptwXY,
    \addArgument{double xl,}
    \addArgument{double xu,}
    \addArgument{nfu\_status *status );}}
    \argumentBox{ptwXY}{A pointer to the \highlight{ptwXYPoints} object.}
    \argumentBox{xl}{The lower limit of integration.}
    \argumentBox{xu}{The upperlimit of integration.}
    \argumentBox{status}{On return, the status value.}
    \vskip 0.05 in \noindent
The return value is $\int_{\rm xl}^{\rm xu} f(x) dx$.

\subsubsection{ptwXY\_integrateDomain}
This routine returns the integral of \highlight{ptwXY} over its domain.
\setargumentNameLengths{ptwXY}
\CallingC{ptwXPoints *ptwXY\_integrateDomain(}{ptwXYPoints *ptwXY,
    \addArgument{nfu\_status *status );}}
    \argumentBox{ptwXY}{A pointer to the \highlight{ptwXYPoints} object.}
    \argumentBox{status}{On return, the status value.}
    \vskip 0.05 in \noindent
The return value is $\int f(x) dx$ over the domain of \highlight{ptwXY}.

\subsubsection{ptwXY\_normalize}
This routine multiplies each y-value of \highlight{ptwXY} by a constant so that its integral is then normalized to 1.
\setargumentNameLengths{ptwXY}
\CallingC{ptwXPoints *ptwXY\_normalize(}{ptwXYPoints *ptwXY );}
    \argumentBox{ptwXY}{A pointer to the \highlight{ptwXYPoints} object.}
    \vskip 0.05 in \noindent

\subsubsection{ptwXY\_integrateDomainWithWeight\_x}
This routine returns the integral of \highlight{ptwXY} weighted by x over its domain.
\setargumentNameLengths{ptwXY}
\CallingC{ptwXPoints *ptwXY\_integrateDomainWithWeight\_x(}{ptwXYPoints *ptwXY,
    \addArgument{nfu\_status *status );}}
    \argumentBox{ptwXY}{A pointer to the \highlight{ptwXYPoints} object.}
    \argumentBox{status}{On return, the status value.}
    \vskip 0.05 in \noindent
The return value is $\int x f(x) dx$ over the domain of \highlight{ptwXY}.

\subsubsection{ptwXY\_integrateWithWeight\_x}
This routine returns the integral of \highlight{ptwXY} weighted by x from xMin to xMax.
\setargumentNameLengths{ptwXY}
\CallingC{ptwXPoints *ptwXY\_integrateWithWeight\_x(}{ptwXYPoints *ptwXY,
    \addArgument{double xMin,}
    \addArgument{double xMax,}
    \addArgument{nfu\_status *status );}}
    \argumentBox{ptwXY}{A pointer to the \highlight{ptwXYPoints} object.}
    \argumentBox{xMin}{The lower limit of the integration.}
    \argumentBox{xMax}{The upper limit of the integration.}
    \argumentBox{status}{On return, the status value.}
    \vskip 0.05 in \noindent
The return value is $\int_{xMin}^{xMax} x f(x) dx$ over the domain of \highlight{ptwXY}.

\subsubsection{ptwXY\_integrateDomainWithWeight\_sqrt\_x}
This routine returns the integral of \highlight{ptwXY} weighted by $\sqrt{ x }$ over its domain.
\setargumentNameLengths{ptwXY}
\CallingC{ptwXPoints *ptwXY\_integrateDomainWithWeight\_sqrt\_x(}{ptwXYPoints *ptwXY,
    \addArgument{nfu\_status *status );}}
    \argumentBox{ptwXY}{A pointer to the \highlight{ptwXYPoints} object.}
    \argumentBox{status}{On return, the status value.}
    \vskip 0.05 in \noindent
The return value is $\int \sqrt{ x } f(x) dx$ over the domain of \highlight{ptwXY}.

\subsubsection{ptwXY\_integrateWithWeight\_sqrt\_x}
This routine returns the integral of \highlight{ptwXY} weighted by x from xMin to xMax.
\setargumentNameLengths{ptwXY}
\CallingC{ptwXPoints *ptwXY\_integrateWithWeight\_sqrt\_x(}{ptwXYPoints *ptwXY,
    \addArgument{double xMin,}
    \addArgument{double xMax,}
    \addArgument{nfu\_status *status );}}
    \argumentBox{ptwXY}{A pointer to the \highlight{ptwXYPoints} object.}
    \argumentBox{xMin}{The lower limit of the integration.}
    \argumentBox{xMax}{The upper limit of the integration.}
    \argumentBox{status}{On return, the status value.}
    \vskip 0.05 in \noindent
The return value is $\int_{xMin}^{xMax} \sqrt{ x } f(x) dx$ over the domain of \highlight{ptwXY}.

\subsubsection{ptwXY\_groupOneFunction}
This routine integrates \highlight{ptwXY} between each pair of consecutive points in \highlight{groupBoundaries} and
returns each integral's value as an element of the returned \highlight{ptwXPoints}.
\setargumentNameLengths{groupBoundaries}
\CallingC{ptwXPoints *ptwXY\_groupOneFunction(}{ptwXYPoints *ptwXY,
    \addArgument{ptwXPoints *groupBoundaries,}
    \addArgument{ptwXY\_group\_normType normType,}
    \addArgument{ptwXPoints *norm,}
    \addArgument{nfu\_status *status );}}
    \argumentBox{ptwXY}{A pointer to the \highlight{ptwXYPoints} object.}
    \argumentBox{groupBoundaries}{A list of x-values.}
    \argumentBox{normType}{The type of normalization to apply to integration.}
    \argumentBox{norm}{A list of normalizations to be applied when normType is \highlight{ptwXY\-\_group\-\_normType\-\_norm}.}
    \argumentBox{status}{On return, the status value.}
    \vskip 0.05 in \noindent
Let \highlight{groupBoundaries} contain $n$ x-values with $x_i < x_{i+1}$. The returned \highlight{ptwXPoints} will contain $n-1$
values $I_i$ such that
\begin{equation}
    I_i = { 1 \over n_i } \int_{x_i}^{x_{i+1}} f(x) dx
\end{equation}
where $n_i$ is determined by \highlight{normType} as,
\begin{description}
    \item[ptwXY\_group\_normType\_none:] $n_i = 1$.
    \item[ptwXY\_group\_normType\_dx:] $n_i = x_{i+1} - x_i$.
    \item[ptwXY\_group\_normType\_norm:] $n_i = $ the $(i-1)^{th}$ element of norm.
\end{description}

\subsubsection{ptwXY\_groupTwoFunctions}
This routine integrates the product of \highlight{ptwXY1} and \highlight{ptwXY2} between each pair of consecutive points in \highlight{groupBoundaries} and
returns each integral's value as an element of the returned \highlight{ptwXPoints}.
\setargumentNameLengths{groupBoundaries}
\CallingC{ptwXPoints *ptwXY\_groupTwoFunctions(}{ptwXYPoints *ptwXY1,
    \addArgument{ptwXYPoints *ptwXY2,}
    \addArgument{ptwXPoints *groupBoundaries,}
    \addArgument{ptwXY\_group\_normType normType,}
    \addArgument{ptwXPoints *norm,}
    \addArgument{nfu\_status *status );}}
    \argumentBox{ptwXY}{A pointer to the \highlight{ptwXYPoints} object.}
    \argumentBox{groupBoundaries}{A list of x-values.}
    \argumentBox{normType}{The type of normalization to apply to integration.}
    \argumentBox{norm}{A list of normalizations to be applied when normType is \highlight{ptwXY\-\_group\-\_normType\-\_norm}.}
    \argumentBox{status}{On return, the status value.}
    \vskip 0.05 in \noindent
Let \highlight{groupBoundaries} contain $n$ x-values with $x_i < x_{i+1}$. The returned \highlight{ptwXPoints} will contain $n-1$
values $I_i$ such that
\begin{equation}
    I_i = { 1 \over n_i } \int_{x_i}^{x_{i+1}} f(x) \; g(x) dx
\end{equation}
where $n_i$ is determined by \highlight{normType} as,
\begin{description}
    \item[ptwXY\_group\_normType\_none:] $n_i = 1$.
    \item[ptwXY\_group\_normType\_dx:] $n_i = x_{i+1} - x_i$.
    \item[ptwXY\_group\_normType\_norm:] $n_i = $ the $(i-1)^{th}$ element of norm.
\end{description}

\subsubsection{ptwXY\_groupThreeFunctions}
This routine integrates the product \highlight{ptwXY1}, \highlight{ptwXY2} and \highlight{ptwXY3} between each pair of 
consecutive points in \highlight{groupBoundaries} and
returns each integral's value as an element of the returned \highlight{ptwXPoints}.
\setargumentNameLengths{groupBoundaries}
\CallingC{ptwXPoints *ptwXY\_groupThreeFunctions(}{ptwXYPoints *ptwXY1,
    \addArgument{ptwXYPoints *ptwXY2,}
    \addArgument{ptwXYPoints *ptwXY3,}
    \addArgument{ptwXPoints *groupBoundaries,}
    \addArgument{ptwXY\_group\_normType normType,}
    \addArgument{ptwXPoints *norm,}
    \addArgument{nfu\_status *status );}}
    \argumentBox{ptwXY}{A pointer to the \highlight{ptwXYPoints} object.}
    \argumentBox{groupBoundaries}{A list of x-values.}
    \argumentBox{normType}{The type of normalization to apply to integration.}
    \argumentBox{norm}{A list of normalizations to be applied when normType is \highlight{ptwXY\-\_group\-\_normType\-\_norm}.}
    \argumentBox{status}{On return, the status value.}
    \vskip 0.05 in \noindent
Let \highlight{groupBoundaries} contain $n$ x-values with $x_i < x_{i+1}$. The returned \highlight{ptwXPoints} will contain $n-1$
values $I_i$ such that
\begin{equation}
    I_i = \int_{x_i}^{x_{i+1}} { f(x) g(x) h(x) \over n_i } dx
\end{equation}
where $n_i$ is determined by \highlight{normType} as,
\begin{description}
    \item[ptwXY\_group\_normType\_none:] $n_i = 1$.
    \item[ptwXY\_group\_normType\_dx:] $n_i = x_{i+1} - x_i$.
    \item[ptwXY\_group\_normType\_norm:] $n_i = $ the $(i-1)^{th}$ element of norm.
\end{description}

\subsection{Convenient}
This section decribes all the routines in the file "ptwXY\_convenient.c".

\subsubsection{ptwXY\_getXArray}
This routine returns, as an \highlight{ptwXPoints}, the list of x values in \highlight{ptwXY}. The returned object is allocated
by \highlight{ptwXY\_getXArray} and must be freed by the user.
\CallingC{ptwXPoints *ptwXY\_getXArray(}{ptwXYPoints *ptwXY,
    \addArgument{fnu\_status *status );}}
    \argumentBox{ptwXY}{A pointer to the \highlight{ptwXYPoints} object.}
    \argumentBox{*status}{The status.}
Returns NULL if an error occured.

\subsubsection{ptwXY\_dullEdges}
This function insures that the y-values at the end-points of  \highlight{ptwXY} are 0. This can be usefull for making
sure two \highlight{ptwXYPoints} instances have mutual domains.
\setargumentNameLengths{positiveXOnly}
\CallingC{nfu\_status ptwXY\_dullEdges(}{ptwXYPoints *ptwXY,
    \addArgument{double lowerEps,}
    \addArgument{double upperEps,}
    \addArgument{int positiveXOnly );}}
    \argumentBox{ptwXY}{A pointer to the \highlight{ptwXYPoints} object.}
    \argumentBox{lowerEps}{The amount to adjust the first points.}
    \argumentBox{upperEps}{The amount to adjust the last points.}
    \argumentBox{positiveXOnly}{The next point in the linked list.}
The description here will mainly focuses on the dulling of the low point of ptwXY, the upper point's dulling is similar.
Let $\epsilon_l =$ lowerEps, $x_0$ and $y_0$ be the first point of ptwXY and $x_1$ and $y_1$ be the second point of ptwXY.
Also, if $x_0 \ne 0$ then let $\Delta x = |\epsilon_l| x_0$ otherwise let $\Delta x = |\epsilon_l|$.
Then, the points around $x_0$ are modified only if lowerEps $\ne 0$ and $y_0 \ne 0$.
The dulling of the lower edge can have one of the four outcomes listed here,
\begin{eqnarray}
                        & x_0, 0    & \hskip .5 in x_p, y_p            \hskip .5 in x_1, y_1 \hskip .5 in {\rm outcome \ 1} \nonumber \\
                        & x_0, 0    & \hskip .5 in \hphantom{x_p, y_p} \hskip .5 in x_1, y_1 \hskip .5 in {\rm outcome \ 2} \nonumber \\
    x_m, 0 \hskip .5 in & x_0, y'_0 & \hskip .5 in x_p, y_p            \hskip .5 in x_1, y_1 \hskip .5 in {\rm outcome \ 3} \nonumber \\
    x_m, 0 \hskip .5 in & x_0, y'_0 & \hskip .5 in \hphantom{x_p, y_p} \hskip .5 in x_1, y_1 \hskip .5 in {\rm outcome \ 4} \nonumber
\end{eqnarray}
In all outcomes, the lower point now has $y = 0$.
The point is added at $x_p = x_0 + \Delta x$ with $y = f(x_p)$ only if $x_0 + 2 \Delta x < x_2$.
If the point at $x_m = x_0 - \Delta x$ is not added, then $y_0$ is set to 0 as shown in outcomes 1 and 2.
The point $x_m$ is not added if $\epsilon_l > 0$, or positiveXOnly is true and $x_m < 0$ and $x_0 \ge 0$.

The dulling of the upper edge can have one of the four outcomes listed here,
\begin{eqnarray}
    x_{k-1}, y_{k-1} \hskip .5 in x_m, y_m \hskip .5 in            & x_k, 0 & \hskip .5 in  \hphantom{x_p, 0} \hskip .5 in {\rm outcome \ 1} \nonumber \\
    x_{k-1}, y_{k-1} \hskip .5 in \hphantom{x_m, y_m} \hskip .5 in & x_k, 0 & \hskip .5 in  \hphantom{x_p, 0} \hskip .5 in {\rm outcome \ 2} \nonumber \\
    x_{k-1}, y_{k-1} \hskip .5 in x_m, y_m \hskip .5 in            & x_k, y_k & \hskip .5 in x_p, 0 \hskip .5 in {\rm outcome \ 3} \nonumber \\
    x_{k-1}, y_{k-1} \hskip .5 in \hphantom{x_m, y_m} \hskip .5 in & x_k, y_k & \hskip .5 in x_p, 0 \hskip .5 in {\rm outcome \ 4} \nonumber
\end{eqnarray}
where $k$ is the index of the last point.

\subsubsection{ptwXY\_mergeClosePoints}
Removes and/or moves points so that no two consecutive points are too close to others.
\CallingC{fnu\_status ptwXY\_mergeClosePoints(}{ptwXYPoints *ptwXY,
    \addArgument{double epsilon);}}
    \argumentBox{ptwXY}{A pointer to the \highlight{ptwXYPoints} object.}
    \argumentBox{epsilon}{The minimum relative spacing desired.}
Points are removed and/or moved so the $x_{i+1} - x_i \le \highlight{epsilon} \times ( x_i + x_{i+1} ) / 2$.

\subsubsection{ptwXY\_intersectionWith\_ptwX}
This routine returns an \highlight{ptwXYPoints} instance whose x-values are the intersection of \highlight{ptwXY}'s and
\highlight{ptwX}'s x-values. The domains of \highlight{ptwXY} and \highlight{ptwX} do not have to be mutual.
\CallingC{ptwXY\_intersectionWith\_ptwX(}{ptwXYPoints *ptwXY,
    \addArgument{ptwXPoints *ptwX,}
    \addArgument{nfu\_status *status );}}
    \argumentBox{ptwXY}{A pointer to the \highlight{ptwXYPoints} object.}
    \argumentBox{ptwX}{A pointer to the \highlight{ptwXPoints} object.}
    \argumentBox{status}{On return, the status value.}

\subsubsection{ptwXY\_areDomainsMutual}
This routine returns \highlight{nfu\_Okay} if \highlight{ptwXY1} and \highlight{ptwXY2} are mutual.
\CallingC{fnu\_status ptwXY\_areDomainsMutual(}{ptwXYPoints *ptwXY1,
    \addArgument{ptwXYPoints *ptwXY2 );}}
    \argumentBox{ptwXY1}{A pointer to a \highlight{ptwXYPoints} object.}
    \argumentBox{ptwXY2}{A pointer to a \highlight{ptwXYPoints} object.}
If one or both of \highlight{ptwXY1} and \highlight{ptwXY2} are empty, \highlight{nfu\_empty} is returned.
If one or both of \highlight{ptwXY1} and \highlight{ptwXY2} has only one point, \highlight{nfu\_tooFewPoints} is returned.
If the domains are not mutual, \highlight{nfu\_domainsNotMutual} is returned.

\subsubsection{ptwXY\_mutualifyDomains}
If possible and needed, this routine mutualifies the domains of \highlight{ptwXY1} and \highlight{ptwXY2} by calling 
\highlight{ptwXY\_dullEdges} on one or both of \highlight{ptwXY1} and \highlight{ptwXY2} if needed.
\setargumentNameLengths{positiveXOnly1}
\CallingC{fnu\_status ptwXY\_mutualifyDomains(}{ptwXYPoints *ptwXY1,
    \addArgument{double lowerEps1,}
    \addArgument{double upperEps1,}
    \addArgument{int positiveXOnly1,}
    \addArgument{ptwXYPoints *ptwXY2,}
    \addArgument{double lowerEps2,}
    \addArgument{double upperEps2,}
    \addArgument{int positiveXOnly2 );}}
    \argumentBox{ptwXY1}{A pointer to a \highlight{ptwXYPoints} object.}
    \argumentBox{lowerEps1}{If needed the value of \highlight{lowerEps} passed to \highlight{ptwXY\_dullEdges} when dulling \highlight{ptwXY1}.}
    \argumentBox{upperEps1}{If needed the value of \highlight{upperEps} passed to \highlight{ptwXY\_dullEdges} when dulling \highlight{ptwXY1}.}
    \argumentBox{positiveXOnly1}{The value of \highlight{positiveXOnly} passed to \highlight{ptwXY\_dullEdges} when dulling \highlight{ptwXY1}.}
    \argumentBox{ptwXY2}{A pointer to a \highlight{ptwXYPoints} object.}
    \argumentBox{lowerEps2}{If needed the value of \highlight{lowerEps} passed to \highlight{ptwXY\_dullEdges} when dulling \highlight{ptwXY2}.}
    \argumentBox{upperEps2}{If needed the value of \highlight{upperEps} passed to \highlight{ptwXY\_dullEdges} when dulling \highlight{ptwXY2}.}
    \argumentBox{positiveXOnly2}{The value of \highlight{positiveXOnly} passed to \highlight{ptwXY\_dullEdges} when dulling \highlight{ptwXY2}.}

\setargumentNameLengths{llocatedSize}
\subsubsection{ptwXY\_copyToC\_XY}
This routine copies the points from index \highlight{index1} inclusive to \highlight{index2} exclusive of \highlight{ptwXY}
into the address pointed to by \highlight{xys}.
\setargumentNameLengths{allocatedSize}
\CallingC{fnu\_status ptwXY\_copyToC\_XY(}{ptwXYPoints *ptwXY, 
    \addArgument{int64\_t index1,}
    \addArgument{int64\_t index2,}
    \addArgument{int64\_t allocatedSize,}
    \addArgument{int64\_t numberOfPoints,}
    \addArgument{double *xys );}}
    \argumentBox{ptwXY}{A pointer to the \highlight{ptwXYPoints} object.}
    \argumentBox{index1}{The lower index.}
    \argumentBox{index2}{The upper index.}
    \argumentBox{allocatedSize}{The size of the space allocated for xys in pairs of C-double.}
    \argumentBox{numberOfPoints}{The number of (x,y) points filled into *xys.}
    \argumentBox{xys}{A pointer to the space to write the data.}
    \vskip 0.05 in \noindent
The size of \highlight{xys} must be at least 2 $\times$ sizeof( double ) $\times$ \highlight{allocatedSize} bytes.
The values of \highlight{index1} and \highlight{index2} are ajusted as follows. If \highlight{index1} is less than 0, it is set to 0. Then 
if \highlight{index2} is less than \highlight{index1}, it is set to \highlight{index1}. Finally,
if \highlight{index2} is greater than the length of \highlight{ptwXY}, it is set to the length of \highlight{ptwXY}.
If \highlight{allocatedSize} is less than the number of points to be copied (i.e., \highlight{index2} - \highlight{index1} 
after \highlight{index1} and \highlight{index2} are adjusted) then \highlight{nfu\_insufficientMemory} is returned;

The returned \highlight{ptwXYPoints} object will always contain no points in the \highlight{overflowPoints} region.

\setargumentNameLengths{interpolation}
\subsubsection{ptwXY\_valueTo\_ptwXY}
This routine creates a \highlight{ptwXYPoints} object with the two points (x1,y), (x2,y) where x1 $<$ x2.
\CallingC{ptwXYPoints *ptwXY\_valueTo\_ptwXY(}{ptwXY\_interpolation interpolation, 
    \addArgument{double x1,}
    \addArgument{double x2,}
    \addArgument{double y,}
    \addArgument{fnu\_status *status );}}
    \argumentBox{interpolation}{Type of interpolation to perform (see Section~\ref{interpolationSection}).}
    \argumentBox{x1}{x value for the lower point.}
    \argumentBox{x2}{x value for the upper point.}
    \argumentBox{y}{y value for both points.}
    \argumentBox{status}{On return, the status value.}
    \vskip 0.05 in \noindent
If an error occurs, NULL is returned.

\subsubsection{ptwXY\_createGaussianCenteredSigma1}
This routine returns a \highlight{ptwXYPoints} instance of the simple Gaussian $y(x) = \exp( -x^2 / 2 )$.
\setargumentNameLengths{status}
\CallingC{ptwXYPoints *ptwXY\_createGaussianCenteredSigma1(}{double accuracy, 
    \addArgument{nfu\_status *status );}}
    \argumentBox{accuracy}{The returned points are accurate to accuracy.}
    \argumentBox{status}{On return, the status value.}
    \vskip 0.05 in \noindent
The domain ranges from $-\sqrt{ 2 \log( {\rm yMin} )}$ to $\sqrt{ 2 \log( {\rm yMin} )}$ where yMin = $10^{-10}$.

\subsubsection{ptwXY\_createGaussian}
This routine returns a \highlight{ptwXYPoints} instance of the Gaussian $y(x) = a \exp( -( x - c )^2 / ( 2 s ) )$.
\setargumentNameLengths{amplitude}
\CallingC{ptwXYPoints *ptwXY\_createGaussian(}{double accuracy,
    \addArgument{double xCenter,}
    \addArgument{double sigma,}
    \addArgument{double amplitude,}
    \addArgument{double xMin,}
    \addArgument{double xMax,}
    \addArgument{double dullEps,}
    \addArgument{nfu\_status *status );}}
    \argumentBox{accuracy}{The returned points are accurate to accuracy.}
    \argumentBox{xCenter}{The center of the Gaussian.}
    \argumentBox{sigma}{The width of the Gaussian.}
    \argumentBox{amplitude}{The amplitude of the Gaussian.}
    \argumentBox{xMin}{The lower domain of the returned Gaussian.}
    \argumentBox{xMax}{The upper lower domain of the returned Gaussian.}
    \argumentBox{dullEps}{Currently not implemented.}
    \argumentBox{status}{On return, the status value.}
    \vskip 0.05 in \noindent
In the equation $a = $ \highlight{amplitude}, $c = $ \highlight{xCenter} and $s = $ \highlight{sigma}. This routine calls 
\highlight{ptwXY\_create\-GaussianCenteredSigma1} and then scales the x and y values.

\subsection{Miscellaneous}
This section decribes all the functions in the file ``ptwXY\_misc.c''.

\subsubsection{ptwXY\_limitAccuracy}
This function returns accuracy limited between \highlight{ptwXY\_minAccuracy} and 1. Ergo, it is similar to the pseudo code
\begin{equation}
    {\rm accuracy} = {\rm min}( \ 1, \ {\rm max}( \ {\rm accuracy,}\ {\rm ptwXY\_minAccuracy} \ ) \ ) \ \ \ .
\end{equation}
\CallingC{void ptwXY\_limitAccuracy(}{ double accuracy ); }
    \argumentBox{accuracy}{The desired accuracy.}

\subsubsection{ptwXY\_update\_biSectionMax --- Not for general use}
This function is used by \highlight{ptwXY} functions to update the member \highlight{biSectionMax} base on the prior length, given
by \highlight{oldLength}, and the current length of \highlight{ptwXY}.
\CallingCLimited{void ptwXY\_update\_biSectionMax(}{ptwXYPoints *ptwXY,
    \addArgument{double oldLength );}}
    \argumentBox{ptwXY}{A pointer to the \highlight{ptwXYPoints} object.}
    \argumentBox{oldLength}{The prior length of \highlight{ptwXY}.}

\subsubsection{ptwXY\_createFromFunction}
This function creates a \highlight{ptwXYPoints} whose domain ranges from xs[0] to xs[n-1] and whose y-values
are obtained from \highlight{func}. All values of \highlight{xs} are added
and infill between them is done until either \highlight{accuracy} or \highlight{biSectionMax} is satisfied.
\setargumentNameLengths{checkForRoots}
\CallingC{ptwXYPoints *ptwXY\_createFromFunction(}{statusMessageReporting *smr,
    \addArgument{int N,}
    \addArgument{double *xs,}
    \addArgument{ptwXY\_createFromFunction\_callback func,}
    \addArgument{void *argList,}
    \addArgument{double accuracy,}
    \addArgument{int checkForRoots,}
    \addArgument{int biSectionMax );}}
    \argumentBox{smr}{The \highlight{statusMessageReporting} instance to record errors.}
    \argumentBox{N}{Number of values in \highlight{xs}}
    \argumentBox{xs}{Minimum list of x-values to add.}
    \argumentBox{func}{A function called to calculate $y(x)$.}
    \argumentBox{argList}{A pointer passed to \highlight{func}.}
    \argumentBox{accuracy}{The desired accuracy.}
    \argumentBox{checkForRoots}{If true, points where $y = 0$ are searched and added.}
    \argumentBox{biSectionMax}{Maximum number of bisections between consecutive points of \highlight{xs}.}
The function \highlight{func} is called as:
\begin{verbatim}
    nfu_status func( statusMessageReporting *smr, ptwXYPoint *point, void *argList );    .
\end{verbatim}

Often, only 2 values in \highlight{xs} are needed. However, in some cases more values help with the bisection algorithm.
For example, if the function is $y = \sin(x)$ for $0 \le x \le 2 \pi$ and \highlight{xs} only contains the points
0 and $2 \pi$, then bisection will not add the point at $\pi$ as it, like the bounding points, is 0.0. In this case,
\highlight{xs} should contain the values 0.0, $\pi$ and $2 \pi$.

\subsubsection{ptwXY\_applyFunction}
This function is used by other functions to map $y_i$ to func( $x_i$, $y_i$ ) with infilling as needed. For example,
this function is used by \highlight{ptwXY\_pow}.
\setargumentNameLengths{argList}
\CallingC{nf\_status ptwXY\_applyFunction(}{statusMessageReporting *smr,
    \addArgument{ptwXYPoints *ptwXY,}
    \addArgument{ptwXY\_applyFunction\_callback func,}
    \addArgument{void *argList );}}
    \argumentBox{smr}{The \highlight{statusMessageReporting} instance to record errors.}
    \argumentBox{ptwXY}{A pointer to the \highlight{ptwXYPoints} object.}
    \argumentBox{func}{A function called to calculate $y(x)$.}
    \argumentBox{argList}{A pointer passed to \highlight{func}.}
This function infills to maintain the initial accuracy. The function \highlight{func} is called as:
\begin{verbatim}
    nfu_status func( statusMessageReporting *smr, ptwXYPoint *point, void *argList );    .
\end{verbatim}

\subsubsection{ptwXY\_fromString}
This function creates a \highlight{ptwXYPoints} from the string of double values in \highlight{str}. There must be an even number
of string doubles in \highlight{str}.
\setargumentNameLengths{interpolationString}
\CallingC{ptwXYPoints *ptwXY\_fromString(}{statusMessageReporting *smr,
    \addArgument{char const *str,}
    \addArgument{char sep,}
    \addArgument{ptwXY\_interpolation interpolation,}
    \addArgument{char const *interpolationString,}
    \addArgument{double biSectionMax,}
    \addArgument{double accuracy,}
    \addArgument{char **endCharacter,}
    \addArgument{int useSystem\_strtod );}}
    \argumentBox{smr}{The \highlight{statusMessageReporting} instance to record errors.}
    \argumentBox{str}{The list of double values as a string.}
    \argumentBox{sep}{The separator character between each double.}
    \argumentBox{interpolation}{The interpolation of the data.}
    \argumentBox{interpolationString}{The string representation of the string.}
    \argumentBox{biSectionMax}{The biSectionMax of the returned \highlight{ptwXYPoints}.}
    \argumentBox{accuracy}{The accuracy of the returned \highlight{ptwXYPoints}.}
    \argumentBox{endCharacter}{The pointer to the character after the last character converted.}
    \argumentBox{useSystem\_strtod}{See the function \highlight{nfu\_stringToListOfDoubles}.}

\subsubsection{ptwXY\_showInteralStructure --- Not for general use}
This function writes out details of the data in a \highlight{ptwXYPoints} object, including much of the
internal data normally not useful to a user. This function is intended for debugging.
\setargumentNameLengths{printPointersAsNull}
\CallingCLimited{void ptwXY\_showInteralStructure(}{ptwXYPoints *ptwXY,
    \addArgument{FILE *f,}
    \addArgument{int printPointersAsNull );}}
    \argumentBox{ptwXY}{A pointer to the \highlight{ptwXYPoints} object.}
    \argumentBox{f}{The stream where the structure is written.}
    \argumentBox{printPointersAsNull}{If true, all pointers are printed as if their value is NULL.}

\subsubsection{ptwXY\_simpleWrite}
This function writes out the (x,y) points of the \highlight{ptwXYPoints} object to a specified stream.
\setargumentNameLengths{format}
\CallingC{void ptwXY\_simpleWrite(}{ptwXYPoints *ptwXY,
    \addArgument{FILE *f,}
    \addArgument{char *format );}}
    \argumentBox{ptwXY}{A pointer to the \highlight{ptwXYPoints} object.}
    \argumentBox{f}{The stream where the points are written.}
    \argumentBox{format}{The format specifier to use for writing an (x,y) point.}
    \vskip 0.05 in \noindent
The \highlight{format} must contain two C double specifier (e.g., \texttt{"\%12.4f \%17.7e$\backslash$n"}), one each for the x- and y-values of a point.
No line feed characters (e.g., \texttt{"$\backslash$n"}) are printed, except those in \highlight{format}.

\subsubsection{ptwXY\_simplePrint}
This function calls \highlight{ptwXY\_simpleWrite} with stdout as the output stream.
\CallingC{void ptwXY\_simplePrint(}{ptwXYPoints *ptwXY,
    \addArgument{char *format );}}
    \argumentBox{ptwXY}{A pointer to the \highlight{ptwXYPoints} object.}
    \argumentBox{format}{The format specifier to use for writing an (x,y) point.}

\section{The detail of the calculations}
The following sub-sections describe the details on some of the calculations. 
Consider two consecutive points $(x_1,y_1)$ and $(x_2,y_2)$ where $x_1 <= x <= x_2$ and $x_1 < x_2$, then interpolation is defined as

\begin{description}
    \item[Lin-lin interpolation] 
    \begin{equation}
        y = { y_2 ( x - x_1 ) + y_1 ( x_2 - x ) \over ( x_2 - x_1 ) }
    \end{equation}
    \item[Lin-log interpolation] 
    \begin{equation}
        y = y_1 \, \left( y_2 \over y_1 \right)^{x - x_1 \over x_2 - x_1}
    \end{equation}
    \item[Log-lin interpolation] 
    \begin{equation}
        y = { y_1 \log(x_2/x) + y_2 \log(x/x_1) \over \log(x_2/x_1) }
    \end{equation}
    \item[Log-log interpolation] 
    \begin{equation}
        y = y_1 \, \left( x \over x_1 \right)^{\log(y_2/y_1) \over \log(x_2/x_1)}
    \end{equation}
\end{description}

In some calculation we will need the x location for the maximum of the relative error, $( y' - y ) / y$, between the 
approximate value, $y'$, and the ``exact'' value, 
$y$. This x location occurs where the derivative of the relative error is zero:
\begin{equation}    \label{MaxXFormula}
    { d(( y' - y ) / y ) \over dx} = { d(y'/y - 1) \over dx} = {d(y'/y) \over dx} = { 1 \over y^2 } \left( y {dy' \over dx } - y' { dy \over dx } \right) = 0
\end{equation}


\subsection{Converting log-log to lin-lin}
This section describes how fudge2dmath converts a \highlight{fudge2dmathXY} object with interpolation of
\highlight{f2dmC\_interpolationLogLog} (hence called log-log) to one with interpolation of \highlight{f2dmC\_inter-pol-ation-LinLin}
(hence called lin--lin).

From Eq.~\ref{MaxXFormula} the maximum of the relative error occurs where,
\begin{equation}
    { 1 \over y } \left( {dy' \over dx } - {y' \over y} { dy \over dx } \right) = \left({ 1 \over y }\right) \left\{
        { y_2 - y_1 \over x_2 - x_1 } - \left({y' \over x}\right) {\log(y_2/y_1) \over \log(x_2/x_1)} \right\} = 0
\end{equation}
The solution is
\begin{equation}
    { x \over x_1 } = { a ( x_2 / x_1 - y_2 / y_1 ) \over ( 1 - a ) ( y_2 / y_1 - 1 ) }
\end{equation}
where $a = \log(y_2/y_1) / \log(x_2/x_1)$.


\end{document}
