\subsection{Core}
This section decribes all the routines in the file "ptwXY\_core.c".

\subsubsection{ptwXY\_new} \label{ptwXYnewSec}
This routine allocates memory for a new \highlight{ptwXYPoints} object and initializes it by calling \highlight{ptwXY\-\_setup}.
\setargumentNameLengths{secondarySize}
\CallingC{ptwXYPoints *ptwXY\_new(}{ptwXY\_interpolation interpolation,
    \addArgument{double biSectionMax,}
    \addArgument{double accuracy,}
    \addArgument{int64\_t primarySize,}
    \addArgument{int64\_t secondarySize,}
    \addArgument{fnu\_status *status ),}
    \addArgument{int userFlag );}}
    \argumentBox{interpolation}{The type of interpolation to use.}
    \argumentBox{biSectionMax}{The maximum disection allowed.}
    \argumentBox{accuracy}{The interpolation accuracy of the data.}
    \argumentBox{primarySize}{Initial size of the primary cache.}
    \argumentBox{secondarySize}{Initial size of the secondary cache.} 
    \argumentBox{status}{On return, the status value.}
    \argumentBox{userFlag}{An user defined integer value not used by any ptwXY function.}
    \vskip 0.05 in \noindent
If this routine fails, NULL is returned.

\subsubsection{ptwXY\_setup}
This routine initializes a \highlight{ptwXYPoints} object and must be called for a \highlight{ptwXYPoints} object
before that object can be used by any other routine in this package.
\setargumentNameLengths{secondarySize}
\CallingC{fnu\_status ptwXY\_setup(}{ptwXYPoints *ptwXY,
    \addArgument{ptwXY\_interpolation interpolation,}
    \addArgument{double biSectionMax,}
    \addArgument{double accuracy,}
    \addArgument{int64\_t primarySize,}
    \addArgument{int64\_t secondarySize );},
    \addArgument{int userFlag );}}
    \argumentBox{ptwXY}{A pointer to a \highlight{ptwXYPoints} object to initialize.}
    \argumentBox{interpolation}{The type of interpolation to use.}
    \argumentBox{biSectionMax}{The maximum disection allowed.}
    \argumentBox{accuracy}{The interpolation accuracy of the data.}
    \argumentBox{primarySize}{Initial size of the primary cache.}
    \argumentBox{secondarySize}{Initial size of the secondary cache.} 
    \argumentBox{userFlag}{An user defined integer value not used by any ptwXY function.}
    \vskip 0.05 in \noindent
The primary and secondary caches are allocated with routines \highlight{ptwXY\_reallocatePoints} and 
\highlight{ptwXY\_reallocateOverflowPoints} respectively.

\subsubsection{ptwXY\_create}
This routines combines \highlight{ptwXY\_new} and \highlight{ptwXY\_setXYData}.
\CallingC{ptwXYPoints *ptwXY\_create(}{ptwXY\_interpolation interpolation, 
    \addArgument{double biSectionMax,}
    \addArgument{double accuracy,}
    \addArgument{int64\_t primarySize,}
    \addArgument{int64\_t secondarySize,}
    \addArgument{int64\_t length,}
    \addArgument{double *xy ),}
    \addArgument{fnu\_status *status,}
    \addArgument{int userFlag );}}
    \argumentBox{interpolation}{The type of interpolation to use.}
    \argumentBox{biSectionMax}{The maximum disection allowed.}
    \argumentBox{accuracy}{The interpolation accuracy of the data.}
    \argumentBox{primarySize}{Initial size of the primary cache.}
    \argumentBox{secondarySize}{Initial size of the secondary cache.} 
    \argumentBox{length}{The number of points in xy.}
    \argumentBox{xy}{The new points given as $x_0, y_0, x_1, y_1, \; \ldots, \; x_n, y_n$ where n = length - 1.}
    \argumentBox{status}{On return, the status value.}
    \argumentBox{userFlag}{An user defined integer value not used by any ptwXY function.}
    \vskip 0.05 in \noindent
If this routine fails, NULL is returned.

\subsubsection{ptwXY\_createFrom\_Xs\_Ys}
This routines is like \highlight{ptwXY\_create} except the x and y data are given in separate arrays.
\CallingC{ptwXYPoints *ptwXY\_createFrom\_Xs\_Ys(}{ptwXY\_interpolation interpolation, 
    \addArgument{double biSectionMax,}
    \addArgument{double accuracy,}
    \addArgument{int64\_t primarySize,}
    \addArgument{int64\_t secondarySize,}
    \addArgument{int64\_t length,}
    \addArgument{double *Xs ),}
    \addArgument{double *Ys ),}
    \addArgument{fnu\_status *status,}
    \addArgument{int userFlag );}}
    \argumentBox{interpolation}{The type of interpolation to use.}
    \argumentBox{biSectionMax}{The maximum disection allowed.}
    \argumentBox{accuracy}{The interpolation accuracy of the data.}
    \argumentBox{primarySize}{Initial size of the primary cache.}
    \argumentBox{secondarySize}{Initial size of the secondary cache.} 
    \argumentBox{length}{The number of points in xy.}
    \argumentBox{Xs}{The new x points given as $x_0, x_1, \; \ldots, \; x_n$ where n = length - 1.}
    \argumentBox{Ys}{The new y points given as $y_0, y_1, \; \ldots, \; y_n$ where n = length - 1.}
    \argumentBox{status}{On return, the status value.}
    \argumentBox{userFlag}{An user defined integer value not used by any ptwXY function.}
    \vskip 0.05 in \noindent
If this routine fails, NULL is returned.

\subsubsection{ptwXY\_copy}
This routine clears the points in \highlight{dest} and then copies the points from \highlight{src} into \highlight{dest}.
The \highlight{src} object is not modified.
\setargumentNameLengths{dest}
\CallingC{fnu\_status ptwXY\_copy(}{ptwXYPoints *dest,
    \addArgument{ptwXYPoints *src );}
    }
    \argumentBox{dest}{A pointer to the destination \highlight{ptwXYPoints} object.}
    \argumentBox{src}{A pointer to the source \highlight{ptwXYPoints} object.}

\subsubsection{ptwXY\_clone}
This routine creates a new \highlight{ptwXYPoints} object and sets its points to the points in its first argument.
\CallingC{ptwXYPoints *ptwXY\_clone(}{ptwXYPoints *ptwXY, 
    \addArgument{fnu\_status *status );}}
    \argumentBox{ptwXY}{A pointer to the \highlight{ptwXYPoints} object.}
    \argumentBox{status}{On return, the status value.}
    \vskip 0.05 in \noindent
If an error occurs, NULL is returned.

\subsubsection{ptwXY\_slice}
This routine creates a new \highlight{ptwXYPoints} object and sets its points to the points from index \highlight{index1} inclusive 
to \highlight{index2} exclusive of \highlight{ptwXY}.
\setargumentNameLengths{secondarySize}
\CallingC{ptwXYPoints *ptwXY\_slice(}{ptwXYPoints *ptwXY, 
    \addArgument{int64\_t index1,}
    \addArgument{int64\_t index2,}
    \addArgument{int64\_t secondarySize,}
    \addArgument{fnu\_status *status );}}
    \argumentBox{ptwXY}{A pointer to the \highlight{ptwXYPoints} object.}
    \argumentBox{index1}{The lower index.}
    \argumentBox{index2}{The upper index.}
    \argumentBox{secondarySize}{Initial size of the secondary cahce.}
    \argumentBox{status}{On return, the status value.}
    \vskip 0.05 in \noindent
If an error occurs, NULL is returned.

\subsubsection{ptwXY\_xSlice}
This routine creates a new \highlight{ptwXYPoints} object and sets its points to the points from the points between the domain
\highlight{xMin} and \highlight{xMax} of \highlight{ptwXY}. If \highlight{fill} is true, points at xMin and xMax are added if
not in the inputted \highlight{ptwXY}.
\setargumentNameLengths{secondarySize}
\CallingC{ptwXYPoints *ptwXY\_xSlice(}{ptwXYPoints *ptwXY, 
    \addArgument{double xMin,}
    \addArgument{double xMax,}
    \addArgument{int64\_t secondarySize,}
    \addArgument{int fill,}
    \addArgument{fnu\_status *status );}}
    \argumentBox{ptwXY}{A pointer to the \highlight{ptwXYPoints} object.}
    \argumentBox{xMax}{The lower domain value.}
    \argumentBox{xMax}{The upper domain value.}
    \argumentBox{secondarySize}{Initial size of the secondary cahce.}
    \argumentBox{fill}{Initial size of the secondary cahce.}
    \argumentBox{status}{On return, the status value.}
    \vskip 0.05 in \noindent
If an error occurs, NULL is returned.

\subsubsection{ptwXY\_xMinSlice}
This routine creates a new \highlight{ptwXYPoints} object and sets its points to the points from the points between the domain
\highlight{xMin} to the end of \highlight{ptwXY}. If \highlight{fill} is true, point at xMin is added if
not in the inputted \highlight{ptwXY}.
\setargumentNameLengths{secondarySize}
\CallingC{ptwXYPoints *ptwXY\_xMinSlice(}{ptwXYPoints *ptwXY, 
    \addArgument{double xMin,}
    \addArgument{int64\_t secondarySize,}
    \addArgument{int fill,}
    \addArgument{fnu\_status *status );}}
    \argumentBox{ptwXY}{A pointer to the \highlight{ptwXYPoints} object.}
    \argumentBox{xMin}{The lower domain value.}
    \argumentBox{secondarySize}{Initial size of the secondary cahce.}
    \argumentBox{fill}{Initial size of the secondary cahce.}
    \argumentBox{status}{On return, the status value.}
    \vskip 0.05 in \noindent
If an error occurs, NULL is returned.

\subsubsection{ptwXY\_xMaxSlice}
This routine creates a new \highlight{ptwXYPoints} object and sets its points to the points from the points between the domain of
the beginning of \highlight{ptwXY} to xMax. If \highlight{fill} is true, point at xMax is added if
not in the inputted \highlight{ptwXY}.
\setargumentNameLengths{secondarySize}
\CallingC{ptwXYPoints *ptwXY\_xMaxSlice(}{ptwXYPoints *ptwXY, 
    \addArgument{double xMax,}
    \addArgument{int64\_t secondarySize,}
    \addArgument{int fill,}
    \addArgument{fnu\_status *status );}}
    \argumentBox{ptwXY}{A pointer to the \highlight{ptwXYPoints} object.}
    \argumentBox{xMax}{The upper domain value.}
    \argumentBox{secondarySize}{Initial size of the secondary cahce.}
    \argumentBox{fill}{Initial size of the secondary cahce.}
    \argumentBox{status}{On return, the status value.}
    \vskip 0.05 in \noindent
If an error occurs, NULL is returned.

\subsubsection{ptwXY\_getUserFlag}
This routine returns the value of \highlight{ptwXY}'s userFlag member.
\CallingC{int ptwXY\_getUserFlag(}{ptwXYPoints *ptwXY );}
    \argumentBox{ptwXY}{A pointer to the \highlight{ptwXYPoints} object.}

\subsubsection{ptwXY\_setUserFlag}
This routine sets the value of the \highlight{ptwXY}'s userFlag member to userFlag.
\CallingC{void ptwXY\_setUserFlag(}{ptwXYPoints *ptwXY,
    \addArgument{int userFlag);}}
    \argumentBox{ptwXY}{A pointer to the \highlight{ptwXYPoints} object.}
    \argumentBox{userFlag}{The value to set ptwXY's userFlag to.}

\subsubsection{ptwXY\_getAccuracy}
This routine returns the value of \highlight{ptwXY}'s accuracy member.
\CallingC{double ptwXY\_getAccuracy(}{ptwXYPoints *ptwXY );}
    \argumentBox{ptwXY}{A pointer to the \highlight{ptwXYPoints} object.}

\subsubsection{ptwXY\_setAccuracy}
This routine sets the value of the \highlight{ptwXY}'s accuracy member to accuracy.
\CallingC{double ptwXY\_setAccuracy(}{ptwXYPoints *ptwXY,
    \addArgument{double accuracy );}}
    \argumentBox{ptwXY}{A pointer to the \highlight{ptwXYPoints} object.}
    \argumentBox{accuracy}{The value to set ptwXY's accuracy to.}
Becuase the range of accuracy is limited, the actual value set may be different then the 
argument accuracy. The actual value set in ptwXY is returned.

\subsubsection{ptwXY\_getBiSectionMax}
This routine returns the value of \highlight{ptwXY}'s biSectionMax member.
\CallingC{double ptwXY\_getBiSectionMax(}{ptwXYPoints *ptwXY );}
    \argumentBox{ptwXY}{A pointer to the \highlight{ptwXYPoints} object.}

\subsubsection{ptwXY\_setBiSectionMax}
This routine sets the value of the \highlight{ptwXY}'s biSectionMax member to biSectionMax.
\CallingC{double ptwXY\_setBiSectionMax(}{ptwXYPoints *ptwXY,
    \addArgument{double biSectionMax );}}
    \argumentBox{ptwXY}{A pointer to the \highlight{ptwXYPoints} object.}
    \argumentBox{biSectionMax}{The value to set ptwXY's biSectionMax to.}
Becuase the range of biSectionMax is limited, the actual value set may be different then the 
argument biSectionMax. The actual value set in ptwXY is returned.

\subsubsection{ptwXY\_reallocatePoints}
This routine changes the size of the primary cache.
\setargumentNameLengths{forceSmallerResize}
\CallingC{fnu\_status ptwXY\_reallocatePoints(}{ptwXYPoints *ptwXY,
    \addArgument{int64\_t size,}
    \addArgument{int forceSmallerResize );}}
    \argumentBox{ptwXY}{A pointer to the \highlight{ptwXYPoints} object.}
    \argumentBox{size}{The desired size of the primary cache.}
    \argumentBox{forceSmallerResize}{If true (i.e. non-zero) and size is smaller than the current size, the primary cache
        is resized. Otherwise, the primary cache is only reduced if the inputted size is significantly smaller than the current size.}
    \vskip 0.05 in \noindent
The actual memory allocated is the maximum of {\tt size}, the current length of the primary cache and \highlight{ptwXY\_minimumSize}.

\subsubsection{ptwXY\_reallocateOverflowPoints}
This routine changes the size of the secondary cache.
\setargumentNameLengths{ptwXY}
\CallingC{fnu\_status ptwXY\_reallocateOverflowPoints(}{ptwXYPoints *ptwXY,
    \addArgument{int64\_t size );}}
    \argumentBox{ptwXY}{A pointer to the \highlight{ptwXYPoints} object.}
    \argumentBox{size}{The desired size of the secondary cache.}
    \vskip 0.05 in \noindent
The actual memory allocated is the maximum of {\tt size} and \highlight{ptwXY\_minimumOverflowSize}. The function
\highlight{ptwXY\_coalescePoints} is called if the current length of the secondary cache is greater than the inputted size.

\subsubsection{ptwXY\_coalescePoints}
This routine adds the points from the secondary cache to the primary cache and then removes the points from the secondary cache. If the
argument \highlight{newPoint} is not-NULL it is also added to the primary cache.
\setargumentNameLengths{forceSmallerResize}
\CallingC{fnu\_status ptwXY\_coalescePoints(}{ptwXYPoints *ptwXY,
    \addArgument{int64\_t size,}
    \addArgument{ptwXYPointsPoint *newPoint,}
    \addArgument{int forceSmallerResize );}}
    \argumentBox{ptwXY}{A pointer to the \highlight{ptwXYPoints} object.}
    \argumentBox{size}{The desired size of the primary cache.}
    \argumentBox{newPoint}{If not NULL, an additional point to add.}
    \argumentBox{forceSmallerResize}{If true (i.e. non-zero) and size is smaller than the current size, the primary cache
        is resized. Otherwise, the primary cache is only reduced if the new size is significantly smaller than the current size.}
    \vskip 0.05 in \noindent
The actual memory allocated is the maximum of {\tt size}, the new length of the \highlight{ptwXY} object and \highlight{ptwXY\_minimumSize}.

\subsubsection{ptwXY\_simpleCoalescePoints}
This routine is a simple wrapper for \highlight{ptwXY\_coalescePoints} when only coalescing of the existing points is needed.
\CallingC{fnu\_status ptwXY\_simpleCoalescePoints(}{ptwXYPoints *ptwXY );}
    \argumentBox{ptwXY}{A pointer to the \highlight{ptwXYPoints} object.}

\subsubsection{ptwXY\_clear}
This routine removes all points from a \highlight{ptwXYPoints} object but does not free any allocated memory. Upon return, the
length of the \highlight{ptwXYPoints} object is zero.
\setargumentNameLengths{ptwXY}
\CallingC{fnu\_status ptwXY\_clear(}{ptwXYPoints *ptwXY );}
    \argumentBox{ptwXY}{A pointer to the \highlight{ptwXYPoints} object.}

\subsubsection{ptwXY\_release}
This routine frees all the internal memory allocated for a \highlight{ptwXYPoints} object.
\setargumentNameLengths{ptwXY}
\CallingC{fnu\_status ptwXY\_release(}{ptwXYPoints *ptwXY );}
    \argumentBox{ptwXY}{A pointer to the \highlight{ptwXYPoints} object.}

\subsubsection{ptwXY\_free}
This routine calls \highlight{ptwXY\_release} and then calls free on \highlight{ptwXY}. 
\CallingC{fnu\_status ptwXY\_free(}{ptwXYPoints *ptwXY );}
    \argumentBox{ptwXY}{A pointer to the \highlight{ptwXYPoints} object.}
    \vskip 0.05 in \noindent
Any \highlight{ptwXYPoints} object allocated using \highlight{ptwXY\_new} should be freed calling \highlight{ptwXY\_free}.
Once this routine is called, the \highlight{ptwXYPoints} object should never be used.

\subsubsection{ptwXY\_length}
This routine returns the length (i.e., number of points in the primary and secondary caches) for a \highlight{ptwXY} object.
\CallingC{int64\_t ptwXY\_length(}{ptwXYPoints *ptwXY );}
    \argumentBox{ptwXY}{A pointer to the \highlight{ptwXYPoints} object.}

\subsubsection{ptwXY\_getNonOverflowLength}
This routine returns the length of the primary caches (note, this is not its size).
\CallingC{int64\_t ptwXY\_getNonOverflowLength(}{ptwXYPoints *ptwXY );}
    \argumentBox{ptwXY}{A pointer to the \highlight{ptwXYPoints} object.}

\subsubsection{ptwXY\_setXYData}
This routine replaces the current points in a \highlight{ptwXY} object with a new set of points.
\CallingC{fnu\_status ptwXY\_setXYData(}{ptwXYPoints *ptwXY,
    \addArgument{int64\_t length,}
    \addArgument{double *xy );}}
    \argumentBox{ptwXY}{A pointer to the \highlight{ptwXYPoints} object.}
    \argumentBox{length}{The number of points in xy.}
    \argumentBox{xy}{The new points given as $x_0, y_0, x_1, y_1, \; \ldots, \; x_n, y_n$ where n = length - 1.}

\subsubsection{ptwXY\_setXYDataFromXsAndYs}
This routines is like \highlight{ptwXY\_setXYData} except the x and y data are given in separate arrays.
\CallingC{fnu\_status ptwXY\_setXYDataFromXsAndYs(}{ptwXYPoints *ptwXY,
    \addArgument{int64\_t length,}
    \addArgument{double *Xs,}
    \addArgument{double *Ys );}}
    \argumentBox{ptwXY}{A pointer to the \highlight{ptwXYPoints} object.}
    \argumentBox{length}{The number of points in xy.}
    \argumentBox{Xs}{The new x points given as $x_0, x_1, \; \ldots, \; x_n$ where n = length - 1.}
    \argumentBox{Ys}{The new y points given as $y_0, y_1, \; \ldots, \; y_n$ where n = length - 1.}

\subsubsection{ptwXY\_deletePoints}
This routine removes all the points from index \highlight{i1} inclusive to index \highlight{i2} exclusive. Indexing is 0 based.
\CallingC{fnu\_status ptwXY\_deletePoints(}{ptwXYPoints *ptwXY,
    \addArgument{int64\_t i1,}
    \addArgument{int64\_t i2 );}}
    \argumentBox{ptwXY}{A pointer to the \highlight{ptwXYPoints} object.}
    \argumentBox{i1}{The lower index.}
    \argumentBox{i2}{The upper index.}
    \vskip 0.05 in \noindent
As example, if an \highlight{ptwXY} object contains the points (1.2, 4), (1.3, 5), (1.6, 6), (1.9, 3) (2.0, 6), (2.1, 4) 
and (2.3, 1). Then calling \highlight{ptwXY\_deletePoints} with i1 = 2 and i2 = 4 removes the points (1.6, 6) and (1.9, 3).
The indices i1 and i2 must satisfy the relationship ( 0 $\le$ i1 $\le$ i2 $\le n$ ) where $n$ is the length of the
\highlight{ptwXY} object; otherwise, no modification is done to the \highlight{ptwXY} object
and the error \highlight{nfu\_badIndex} is returned.

\subsubsection{ptwXY\_getPointAtIndex}
This routine checks that the index argument is valid, and if it is, this routine returns the result 
of \highlight{ptwXY\_getPointAtIndex\_Unsafely}. Otherwise, NULL is returned.
\CallingC{ptwXYPoint *ptwXY\_getPointAtIndex(}{ptwXYPoints *ptwXY,
    \addArgument{int64\_t index );}}
    \argumentBox{ptwXY}{A pointer to the \highlight{ptwXYPoints} object.}
    \argumentBox{index}{The index of the point to return.}

\subsubsection{ptwXY\_getPointAtIndex\_Unsafely}
This routine returns the point at index. This routine does not check if index is valid and 
thus is not intended for general use. Instead, see \highlight{ptwXY\_getPointAtIndex} for a general use version of this routine.
\CallingCLimited{ptwXYPoint *ptwXY\_getPointAtIndex\_Unsafely(}{ptwXYPoints *ptwXY,
    \addArgument{int64\_t index );}}
    \argumentBox{ptwXY}{A pointer to the \highlight{ptwXYPoints} object.}
    \argumentBox{index}{The index of the point to return.}

\subsubsection{ptwXY\_getXYPairAtIndex}
This routine calls \highlight{ptwXY\_getPointAtIndex} and if the index is valid it returns the point's x and y values via the
arguments *x and *y. Otherwise, *x and *y are unaltered and an error signal is returned.
\CallingC{ptwXYPoint *ptwXY\_getPairAtIndex(}{ptwXYPoints *ptwXY,
    \addArgument{int64\_t index,}
    \addArgument{double *x,}
    \addArgument{double *y );}}
    \argumentBox{ptwXY}{A pointer to the \highlight{ptwXYPoints} object.}
    \argumentBox{index}{The index of the point to return.}
    \argumentBox{*x}{The point's x value is returned in this argument.}
    \argumentBox{*y}{The point's y value is returned in this argument.}

\subsubsection{ptwXY\_getPointsAroundX}
This routine sets the \highlight{lessThanEqualXPoint} and \highlight{greaterThanXPoint} members of the 
\highlight{ptwXY} object to the two points that bound a point $x$.
\CallingCLimited{ptwXY\_lessEqualGreaterX ptwXY\_getPointsAroundX(}{ptwXYPoints *ptwXY,
    \addArgument{double x );}}
    \argumentBox{ptwXY}{A pointer to the \highlight{ptwXYPoints} object.}
    \argumentBox{x}{The $x$ value.}
    \vskip 0.05 in \noindent
If the \highlight{ptwXY} object is empty then the return value is \highlight{ptwXY\_less\-Equal\-GreaterX\_\-empty}.
If $x$ is less than xMin, then \highlight{ptwXY\_less\-Equal\-GreaterX\_\-less\-Than} is return.
If $x$ is greater than xMax, then \highlight{ptwXY\_less\-Equal\-GreaterX\_\-greater\-Than} is return. If $x$ corresponds to a
point in the \highlight{ptwXY} object then \highlight{ptwXY\_\-less\-Equal\-GreaterX\_\-equal} is returned. Otherwise, 
\highlight{ptwXY\_\-less\-Equal\-GreaterX\_\-between} is returned.

\subsubsection{ptwXY\_getValueAtX}
This routine gets the $y$ value at $x$, interpolating if necessary.
\CallingC{fnu\_status ptwXY\_getValueAtX(}{ptwXYPoints *ptwXY,
    \addArgument{double x,}
    \addArgument{double *y );}}
    \argumentBox{ptwXY}{A pointer to the \highlight{ptwXYPoints} object.}
    \argumentBox{x}{The $x$ value.}
    \argumentBox{y}{Upon return, contains the $y$ value.}
    \vskip 0.05 in \noindent
If the x value is outside the domain of the \highlight{ptwXY} object, $y$ is set to zero and the returned value
is \highlight{nfu\_X\-Outside\-Domain}.

\subsubsection{ptwXY\_setValueAtX}
This routine sets the point at $x$ to $y$, if $x$ does not corresponds to a
point in the \highlight{ptwXY} object then a new point is added.
\CallingC{fnu\_status ptwXY\_setValueAtX(}{ptwXYPoints *ptwXY,
    \addArgument{double x,}
    \addArgument{double y );}}
    \argumentBox{ptwXY}{A pointer to the \highlight{ptwXYPoints} object.}
    \argumentBox{x}{The $x$ value.}
    \argumentBox{y}{The $y$ value.}
    \vskip 0.05 in \noindent

\subsubsection{ptwXY\_setXYPairAtIndex}
This routine sets the $x$ and $y$ values at index.
\CallingC{fnu\_status ptwXY\_setXYPairAtIndex(}{ptwXYPoints *ptwXY,
    \addArgument{int64\_t index}
    \addArgument{double x,}
    \addArgument{double y );}}
    \argumentBox{ptwXY}{A pointer to the \highlight{ptwXYPoints} object.}
    \argumentBox{index}{The index of the point to set.}
    \argumentBox{x}{The $x$ value.}
    \argumentBox{y}{The $y$ value.}
    \vskip 0.05 in \noindent
If index is invalid, \highlight{nfu\_badIndex} is returned. If the $x$ value is not valid for index (i.e. $x \le x_{\rm index-1}$ 
or $x \ge x_{\rm index+1}$) then \highlight{nfu\_badIndexForX} is return.

\subsubsection{ptwXY\_getSlopeAtX}
This routine calculates the slope at the point $x$ assuming linear-linear interpolation. That is, for $x_i < x < x_{i+1}$,
the slope is $( y_{i+1} - y_i ) / ( x_{i+1} - x_i )$. If $x = x_j$ is the point in \highlight{ptwXY} at
index $j$ then for side = `+', $i = j$ is used in the above slope equation. Else, if side = `-', $i = j-1$ is used in the above slope equation.
\CallingC{fnu\_status ptwXY\_getSlopeAtX(}{ptwXYPoints *ptwXY,
    \addArgument{double x,}
    \addArgument{const char side,}
    \addArgument{double *slope );}}
    \argumentBox{ptwXY}{A pointer to the \highlight{ptwXYPoints} object.}
    \argumentBox{index}{The index of the point to set.}
    \argumentBox{x}{The $x$ value.}
    \argumentBox{y}{The $y$ value.}
    \vskip 0.05 in \noindent
If side is neither '-' or '+', the error \highlight{nfu\_badInput} is returned.

\subsubsection{ptwXY\_getXMinAndFrom --- Not for general use}
This routine returns the xMin value and indicates whether the minimum value resides in the primary 
or secondary cache.
\setargumentNameLengths{dataFrom}
\CallingCLimited{double ptwXY\_getXMinAndFrom(}{ptwXYPoints *ptwXY,
    \addArgument{ptwXY\_dataFrom *dataFrom );}}
    \argumentBox{ptwXY}{A pointer to the \highlight{ptwXYPoints} object.}
    \argumentBox{dataFrom}{The output of this argument indicates which cache the minimum value resides in.}
    \vskip 0.05 in \noindent
The return value from this routine is xMin. If there are no data in the \highlight{ptwXYPoints} object, then \highlight{dataFrom} is set
to \highlight{ptwXY\_dataFrom\_Unknown}. Otherwise, it is set to \highlight{ptwXY\_data\-From\_Points} or \highlight{ptwXY\_dataFrom\_Overflow} if the 
minimum value is in the primary or secondary cache respectively.

\subsubsection{ptwXY\_getXMin}
This routine returns the xMin value returned by \highlight{ptwXY\_getXMinAndFrom}. The calling routine should check that the
\highlight{ptwXYPoints} object contains at least one point (i.e., that the length is greater than 0). If the length is 0, the return value is
undefined.
\CallingC{double ptwXY\_getXMin(}{ptwXYPoints *ptwXY );}
    \argumentBox{ptwXY}{A pointer to the \highlight{ptwXYPoints} object.}

\subsubsection{ptwXY\_getXMaxAndFrom --- Not for general use}
This routine returns the xMax value and indicates whether the maximum value resides in the primary or secondary cache.
\CallingCLimited{double ptwXY\_getXMaxAndFrom(}{ptwXYPoints *ptwXY,
    \addArgument{ptwXY\_dataFrom *dataFrom );}}
    \argumentBox{ptwXY}{A pointer to the \highlight{ptwXYPoints} object.}
    \argumentBox{dataFrom}{The output of this argument indicates which cache the maximum value resides in.}
    \vskip 0.05 in \noindent
The return value from this routine is xMax. If there are no data in the \highlight{ptwXYPoints} object, then \highlight{dataFrom} is set
to \highlight{ptwXY\_dataFrom\_Unknown}. Otherwise, it is set to \highlight{ptwXY\_data\-From\_Points} or \highlight{ptwXY\_dataFrom\_Overflow} if the 
maximum value is in the primary or secondary cache respectively.

\subsubsection{ptwXY\_getXMax}
This routine returns the xMax value returned by \highlight{ptwXY\_getXMinAndFrom}. The calling routine should check that the
\highlight{ptwXYPoints} object contains at least one point (i.e., that the length is greater than 0). If the length is 0, the return value is
undefined.
\CallingC{double ptwXY\_getXMax(}{ptwXYPoints *ptwXY );}
    \argumentBox{ptwXY}{A pointer to the \highlight{ptwXYPoints} object.}

\subsubsection{ptwXY\_getYMin}
This routine returns the minimum y value in \highlight{ptwXY}.
\CallingC{double ptwXY\_getYMin(}{ptwXYPoints *ptwXY );}
    \argumentBox{ptwXY}{A pointer to the \highlight{ptwXYPoints} object.}

\subsubsection{ptwXY\_getYMax}
This routine returns the maximum y value in \highlight{ptwXY}.
\CallingC{double ptwXY\_getYMax(}{ptwXYPoints *ptwXY );}
    \argumentBox{ptwXY}{A pointer to the \highlight{ptwXYPoints} object.}

\subsubsection{ptwXY\_initialOverflowPoint --- Not for general use}
This routine initializes a point in the secondary cache.
\CallingCLimited{void ptwXY\_initialOverflowPoint(\hskip -1. in}{
    \addArgument{ptwXYOverflowPoint *overflowPoint,}
    \addArgument{ptwXYOverflowPoint *prior,}
    \addArgument{ptwXYOverflowPoint *next );}}
    \argumentBox{ptwXY}{A pointer to the \highlight{ptwXYPoints} object.}
    \argumentBox{prior}{The prior point in the linked list.}
    \argumentBox{next}{The next point in the linked list.}
